%*****************************************
\pdfbookmark[1]{Appendix}{appendix}
\chapter*{Appendix}\label{ch:appendix}
%*****************************************
\section*{Evaluierungsdatensatz}\label{app:evaldata}
{\normalsize
Im Folgenden werden die verwendeten Evaluationsdatensätze vorgestellt, die zur Berechnung der Kriterien in \cref{ch:results} verwendet wurden.
Die Evaluationsdatensätze sind unterteilt in Einzelfaktfragen (\cref{tab:evaldata-single}), Multifakten-Fragen (\cref{tab:evaldata-multi}) und Transferfragen (\cref{tab:evaldata-transfer}).
Während der Evaluation wurden allen Modellen dieselben Fragen gestellt.
Ihre Antworten wurden mit den hier gegebenen Antworten verglichen und ausgewertet.
Die Datensätze der Fragen, allen Antworten der Modelle, zusätzliche Evaluierungsgrafiken als auch die verwendeten Skripte
sind unter \url{https://doi.org/10.5281/zenodo.8363501} zu finden. \par}
{\footnotesize
\begin{landscape}
\begin{longtable}{p{3cm}p{1.7\textwidth}}
    \toprule
    \multicolumn{2}{c}{\textbf{Evaluierungsdatensatz (Einzelfaktfragen)}}\\
    \midrule
    Frage & In Sect. 4.2, we introduced a three-dimensional classification of activities of management of information systems.
    How would you describe the scope and tasks of the following activities of managing information systems? - Developing a strategic information management plan (e.g., this is related to strategic planning), \\
    Umformuliert & How would you describe the scope and tasks of the following activities of managing information systems: Developing a strategic information management plan \\
    Antwort & Developing a strategic information management plan: strategic planning \\
    Quelle|Anz. Antw. & Book | 1\\
    \midrule
    Frage & - Initiating projects from the strategic project portfolio, \\
    Umformuliert & How would you describe the scope and tasks of the following activities of managing information systems: Initiating projects from the strategic project portfolio \\
    Antwort & Initiating projects from the strategic project portfolio: strategic directing \\
    Quelle|Anz. Antw. & Book | 1 \\
    \midrule
    Frage & - Collection and analysis of data from user surveys on their general satisfaction with the health information system, \\
    Umformuliert & How would you describe the scope and tasks of the following activities of managing information systems: Collection and analysis of data from user surveys on their general satisfaction with the health information system \\
    Antwort & Collecting and analyzing data from user surveys on their general health information system's satisfaction: strategic monitoring \\
    Quelle|Anz. Antw. & Book | 1 \\
    \midrule
    Frage & - Planning a project to select and introduce a new CPOE system, \\
    Umformuliert & How would you describe the scope and tasks of the following activities of managing information systems: Planning a project to select and introduce a new CPOE system \\
    Antwort & Planning a project to select and introduce a new CPOE system: tactical planning \\
    Quelle|Anz. Antw. & Book | 1 \\
    \midrule
    Frage & - Executing work packages within an evaluation project of a CPOE system, \\
    Umformuliert & How would you describe the scope and tasks of the following activities of managing information systems: Executing work packages within an evaluation project of a CPOE system \\
    Antwort & Executing work packages within an evaluation project of a CPOE system: tactical directing \\
    Quelle|Anz. Antw. & Book | 1 \\
    \midrule
    Frage & - Assessment of user satisfaction with a new intensive care system, \\
    Umformuliert & How would you describe the scope and tasks of the following activities of managing information systems: Assessment of user satisfaction with a new intensive care system \\
    Antwort & Assessment of user satisfaction with a new intensive care system: tactical monitoring \\
    Quelle|Anz. Antw. & Book | 1 \\
    \midrule
    Frage & - Planning of a user service desk for a group of clinical application components, \\
    Umformuliert & How would you describe the scope and tasks of the following activities of managing information systems: Planning of a user service desk for a group of clinical application components \\
    Antwort & Planning of a user service desk for a group of clinical application components: operational planning \\
    Quelle|Anz. Antw. & Book | 1 \\
    \midrule
    Frage & - Operation of a service desk for a group of clinical application components, \\
    Umformuliert & How would you describe the scope and tasks of the following activities of managing information systems: Operation of a service desk for a group of clinical application components \\
    Antwort & Operation of a service desk for a group of clinical application components: operational directing \\
    Quelle|Anz. Antw. & Book | 1 \\
    \midrule
    Frage & - Daily monitoring of network availability and network failures.\\
    Umformuliert & How would you describe the scope and tasks of the following activities of managing information systems: Daily monitoring of network availability and network failures \\
    Antwort & Daily monitoring of network availability and network failures: operational monitoring \\
    Quelle|Anz. Antw. & Book | 1 \\
    \midrule
    Frage & Definieren Sie den Begriff ``Krankenhausinformationssystem''. \\
    Umformuliert & Define the term ``hospital information system''. \\
    Antwort & A Hospital Information system is the socio-technical subsystem of for hospitals.
    It comprises all data, information, and knowledge processing as well as the associated human or technical actors in their respective data, information, and knowledge processing roles.\\
    Quelle|Anz. Antw. & IS\_2022\_07\_18 | 1 \\
    \midrule
    Frage & Wie erhalten Apotheker:innen Zugriff auf verordnete eRezepte? \\
    Umformuliert & How do pharmacists get access to e-prescriptions? \\
    Antwort & Via the CPOE System within a hospital.
    Extern via their own pharmacists information system.\\
    Quelle|Anz. Antw. & IS\_2022\_07\_18 | 1 \\
    \midrule
    Frage & Definieren Sie den Begriff ``transinstitutionelles Gesundheitsinformationssystem''. Verwenden Sie dabei den Begriff des ``Settings''. \\
    Umformuliert & Define the term ``trans-institutional health information system''. Use the term ``setting'' for this purpose.\\
    Antwort & A transinstitutional health information system is the socio-technical subsystem which comprises all data, information, and knowledge processing as well as the associated human or technical actors in their respective data, information, and knowledge processing roles.
    It describes the information system of a health care network, in which multiple Settings are present.
    Each setting is a context this information system is used in, for example in the stationary or ambulant care, but also rehabilitation or care of the elder.\\
    Quelle|Anz. Antw. & IS\_2022\_09\_27 | 1 \\
    \midrule
    Frage & tHIS? \\
    Umformuliert & Define the term ``tHIS''. \\
    Antwort & tHIS stands for transinstitutional health information system, an information system of a health care network which consists of multiple health care settings (contexts). It comprises all data, information, and knowledge processing as well as the associated human or technical actors in their respective data, information, and knowledge processing roles.\\
    Quelle|Anz. Antw. & IS\_2022\_09\_27 | 1 \\
    \midrule
    Frage & Was ist ein Arztbrief? \\
    Umformuliert & What is a doctor's letter? \\
    Antwort & A doctor's letter is a transfer document for communication between doctors.
    It is a form of information for the referring doctor, who has arranged for a referral to a hospital or other medical treatment in the outpatient sector or for the further treating doctor who takes over the further treatment.
    The doctor's letter provides a summary overview of the patient's status at discharge, a review of the course of the disease, the therapy initiated, an interpretation of the events related to the course of the disease in the specific case, information on the classification of the disease according to ICD, OPS, ICF and possibly also DRG and recommendations for continuing therapy.\\
    Quelle|Anz. Antw. & IS\_2022\_09\_27 | 1 \\
    \midrule
    Frage & Was ist openEHR \\
    Umformuliert & What is openEHR? \\
    Antwort & openEHR is an open standard specification in health informatics that describes the management and storage, retrieval and exchange of health data in electronic health records (EHRs). In openHER, all health data for a person is stored in a ``one lifetime'', vendor-independent, person-centred EHR.
    It is completely model-driven, and separates domain semantics out from software, into models created by domain professionals.\\
    Quelle|Anz. Antw. & IS\_2022\_09\_27 | 1 \\
    \midrule
    Frage & Was ist ein Archetype? \\
    Umformuliert & What is an archetype in the context of health information systems? \\
    Antwort & An archetype is a formal specification of a clinical concept that can be used to create interoperable and reusable electronic health records.
    Archetypes are designed to be independent of any specific EHR system and are intended to be used as building blocks for EHRs.\\
    Quelle|Anz. Antw. & IS\_2022\_09\_27 | 1 \\
    \midrule
    Frage & Ist openEHR ein Standard? \\
    Umformuliert & Is openEHR a standard? \\
    Antwort & Yes, openEHR is an open standard specification in health informatics that describes the management and storage, retrieval and exchange of health data in electronic health records.\\
    Quelle|Anz. Antw. & IS\_2022\_09\_27 | 1 \\
    \midrule
    Frage & Könnte man mit Hilfe von openEHR eine (DB1, ACn, Vn)-Architektur realisieren? \\
    Umformuliert & Could one realise a DB\textasciicircum{}1, AC\textasciicircum{}n, V\textasciicircum{}n architecture with the help of openEHR? \\
    Antwort & Yes \\
    Quelle|Anz. Antw. & IS\_2022\_09\_27 | 1 \\
    \midrule
    Frage & Was ist ein Modell? \\
    Umformuliert & What is a model in the context of health information systems? \\
    Antwort & A model is a representation of a system or process that is used to help understand and predict the behavior of the system or process.
    Health information systems models can be used to help design and evaluate health information systems.
    They can also be used to help identify areas for improvement in health information systems.\\
    Quelle|Anz. Antw. & IS\_2022\_09\_27 | 1 \\
    \midrule
    Frage & Was ist ARIS? \\
    Umformuliert & What is ARIS? \\
    Antwort & The Architecture of Integrated Information Systems is a framework for describing business processes.
    It provides modeling methods and meta-structures that are comprised in information models.\\
    Quelle|Anz. Antw. & IS\_2022\_09\_27 | 1 \\
    \midrule
    Frage & Wie kann man dynamische Aspekte von IS modellieren? \\
    Umformuliert & How can dynamic aspects of Information Systems be modelled? \\
    Antwort & With a Business Process Model such as BPMN \\
    Quelle|Anz. Antw. & IS\_2022\_09\_27 | 1 \\
    \midrule
    Frage & Was ist IHE? \\
    Umformuliert & What is IHE? \\
    Antwort & IHE stands for Integrating the Healthcare Enterprise.
    It is an initiative by healthcare professionals and industry to improve the way computer systems in healthcare share information.
    IHE promotes the coordinated use of established standards such as DICOM and HL7 to address specific clinical needs in support of optimal patient care \\
    Quelle|Anz. Antw. & IS\_2022\_09\_27 | 1 \\
    \midrule
    Frage & Was tut ein Kommunikationsserver? \\
    Umformuliert & What does a communication server do  in the context of health information systems? \\
    Antwort & A communication server in the context of health information systems is a server that enables communication between different systems.
    It is responsible for routing messages between different applications and systems.
    It can also be used to manage the flow of data between different systems.\\
    Quelle|Anz. Antw. & IS\_2022\_09\_27 | 1 \\
    \midrule
    Frage & Wozu setzt man Remote Function Calls ein? \\
    Umformuliert & What are Remote Function Calls used for? \\
    Antwort & Remote Function Calls (RFC) is a communications interface based on CPI-C, but with more functions and easier for application programmers to use.
    It is the call or remote execution of a Remote Function Module in an external system.
    In the SAP system, these functions are provided by the RFC interface system.
    The RFC interface system enables function calls between two SAP systems \\
    Quelle|Anz. Antw. & IS\_2022\_09\_27 | 1 \\
    \midrule
    Frage & Wie findet ein Labormitarbeiter mit einer Fallnummer in der Hand Geburtsdatum und Geschlecht eines Patienten? \\
    Umformuliert & How does a lab worker with a case number in hand find a patient's date of birth and gender? \\
    Antwort & In the Laboratory Information System, which got its information from a communication server.\\
    Quelle|Anz. Antw. & IS\_2022\_09\_27 | 1 \\
    \midrule
    Frage & Erläutern Sie die Taxonomy für Krankenhausinformationssysteme! \\
    Umformuliert & Explain the taxonomy for hospital information systems! \\
    Antwort & The taxonomy for HIS is a classification system that is used to categorize different types of HIS based on their functionality and purpose.
    The taxonomy can be used to help healthcare organizations select the right HIS for their needs.
    There are several different taxonomies for HIS, but they generally include categories such as clinical information systems, administrative information systems, and decision support systems \\
    Quelle|Anz. Antw. & IS\_2022\_09\_27 | 1 \\
    \midrule
    Frage & Wieso haben wir meist (ACn, Vn)? \\
    Umformuliert & Why do we mostly have (AC\textasciicircum{}n, V\textasciicircum{}n)? \\
    Antwort & This allows Best of Breed Architecture choosing the best products and being independent from vendors.\\
    Quelle|Anz. Antw. & IS\_2022\_09\_27 | 1 \\
    \midrule
    Frage & Wie kann man auch bei fehlender Vollausleuchtung mit den allgegenwärtigen Funklöchern hinter metallwagen so umgehen, dass Mobile Anwendungen nicht laufend abstürzen? \\
    Umformuliert & Even in the absence of full coverage, how can you deal with the ubiquitous radio holes behind metal trolleys in such a way that mobile applications don't crash all the time? \\
    Antwort & One of this:
    - caching and local storage
    - Offline Mode
    - Background Syncing and Queuing \\
    Quelle|Anz. Antw. & IS\_2022\_09\_27 | 1 \\
    \midrule
    Frage & Welche Speichermedien gewährleisten eine Unveränderbarkeit der Daten? \\
    Umformuliert & Which storage media guarantee that the data cannot be changed? \\
    Antwort & Write once read many (WORM) data such as Optical Discs, Tape Drives, Solid State Drives or Memory cards \\
    Quelle|Anz. Antw. & IS\_2022\_09\_27 | 1 \\
    \midrule
    Frage & Ist ``Dokumentation'' eine datenverarbeitende Aufgabe bzw.
    eine enterprise function? \\
    Umformuliert & Is ``documentation'' a data processing task or an enterprise function? \\
    Antwort & A data processing task \\
    Quelle|Anz. Antw. & IS\_2022\_09\_27 | 1 \\
    \midrule
    Frage & Mit welchem Anwendungssystem wird im UKL die Anforderung von Laborleistungen  (order entry) unterstützt? \\
    Umformuliert & Which application system is used in the University Hospital to support the request for laboratory services (order entry)? \\
    Antwort & Laboratory Information System \\
    Quelle|Anz. Antw. & IS\_2022\_09\_27 | 1 \\
    \midrule
    Frage & Wie erfolgt die Kommunikation der Anforderung (order) an das LIS? \\
    Umformuliert & How is the request (order) communicated to the LIS? \\
    Antwort & It isn't. The call is made from the work place during the context integration of patient identifying data to the LIS.\\
    Quelle|Anz. Antw. & IS\_2022\_09\_27 | 1 \\
    \midrule
    Frage & Mit welchem Anwendungssystem wird im UKL ein Laborbefund (finding) angezeigt? \\
    Umformuliert & Which application system is usually used to display a laboratory finding in the University Hospital? \\
    Antwort & (COPRA, LIS direkt!) \\
    Quelle|Anz. Antw. & IS\_2022\_09\_27 | 1 \\
    \midrule
    Frage & Wie würden sie die PWE eines KIS gestalten? \\
    Umformuliert & How would they design the physical tool layer of a HIS? \\
    Antwort & Redundancy by mirror servers \\
    Quelle|Anz. Antw. & A\_2021 | 1\\
\bottomrule
\caption*{Evaluierungsdatensatz (Einzelfaktfragen)}\label{tab:evaldata-single}
\end{longtable}
\end{landscape}

\begin{landscape}
    \begin{longtable}{p{3cm}p{1.7\textwidth}}
    \toprule
    \multicolumn{2}{c}{\textbf{Evaluierungsdatensatz (Multifakten-Fragen)}}\\
    \midrule
    Frage & Was ist der Unterschied zwischen dem PVS und dem PDMS? \\
    Umformuliert & What is the difference between the PMS and the PDMS? \\
    Antwort & PMS:
    - primary focus on helping health care provides streamline their day-to-day tasks

    PDMS:
    - primary focus on capturing maintaining comprehensive and accurate patient medical data.\\
    Quelle|Anz. Antw. &  IS\_2022\_09\_27  | 2 \\
    \midrule
    Frage & Look at the entity type ``patient'' that is interpreted and updated by various functions.
    Which functions update the patient information, which functions interpret it? \\
    Umformuliert & Look at the entity type ``patient'' that is interpreted and updated by various functions to be performed by health care professionals and other staff in health care facilities.
    Which functions update the patient information, which functions interpret it? \\
    Antwort & The entity type ``patient'' is updated by the function ``patient admission.'' All other functions that are related to patient care interpret it.\\
    Quelle|Anz. Antw. &  Book  | 2 \\
    \midrule
    Frage & Look at the 3LGM\textasciicircum{}2 example in Sect. 2.15.
    Use this example to explain the meaning of the following elements: functions, entity types, application systems, non-computer-based application components, physical data processing system, and inter-layer relationships.\\
    Umformuliert & Use the provided example to explain the meaning of the following elements: functions, entity types, application systems, non-computer-based application components, physical data processing system, and inter-layer relationships.\\
    Kontext & Example: Four subfunctions of patient admission (appointment scheduling, patient identification and checking for readmitted patients, administrative admission, and visitor and information services) are supported by the patient administration system, which is a part of the ERPS.
    Medical admission and nursing admission are supported by the MDMS.
    Obtaining consent for processing of patient-related data is supported by the non-computer-based application component for patient data privacy forms.
    This application component is based on paper forms which are scanned by a clerk (see physical tool layer) and then stored in the MDMS.

    The patient administration system, which is the master application system (Sect. 3.9.1) for the entity type ``patient,'' sends the administrative patient data as a message to the MDMS.
    The MDMS can thus store this information about the entity type ``patient'' in its own database; administrative patient data that is needed to support medical admission and nursing admission as functions therefore do not have to be reentered in the MDMS.
    The entity type ``patient'' is both stored in the database systems of the ERPS and the MDMS what is represented by dashed lines between the domain layer and the logical tool layer.

    Both the patient administration system and the MDMS are run on servers at a virtualized server farm (see relationships between logical and physical tool layer). The application systems can be accessed by different end devices (patient terminal, PC, tablet PC).

    It therefore simplifies some aspects which might be relevant in other contexts.
    Another visualization of relationships between 3LGM\textasciicircum{}2 model elements is the matrix view.
    The patient administration system supports three different functions, the MDMS supports two functions, and one function is supported by the paper-based patient data privacy form system.
    The matrix view also helps to identify incomplete parts of models.
    We can see that there are no functions modeled that are supported by the financial accounting system, the human resources management system, and the material management system, which are parts of the ERPS.

    The matrix view is an alternative representation of configuration lines between functions at the domain layer and application components at the logical tool layer.
    Matrix views are also available for visualizing relations between other pairs of connected 3LGM\textasciicircum{}2 classes.\\
    Antwort & Administrative admission is an enterprise function that is supported by the patient administration system.
    One entity type that is used and updated by this function is ``patient.'' The paper-based patient data privacy form system is an example of a non-computer-based application component.
    The virtualized server farm is an example of a physical tool.
    The inter-layer relationships of this example show which functions are supported by which application system and which physical data processing system the application systems are installed on.\\
    Quelle|Anz. Antw. &  Book  | 6 \\
    \midrule
    Frage & Look at the 3LGM\textasciicircum{}2 sample model in Sect. 2.15 and try to answer the following questions.

    (a) Find examples of specialization or decomposition at the domain layer.

    (b) What is the meaning of the arrows pointing from patient identification to ``patient'' and from ``patient'' to medical admission?

    (c) What entity type that is stored in the paper-based patient data privacy form system should be added at the domain layer?

    (d) Why is the function patient admission not connected with any application system?

    (e) Which physical data processing systems are needed for the function ``obtaining patient consent for the processing of data''? \\
    Umformuliert & Answer the following questions under the given example context.
    (a) Find examples of specialization or decomposition at the domain layer.

    (b) What is the meaning of the arrows pointing from patient identification to ``patient'' and from ``patient'' to medical admission?

    (c) What entity type that is stored in the paper-based patient data privacy form system should be added at the domain layer?

    (d) Why is the function patient admission not connected with any application system?

    (e) Which physical data processing systems are needed for the function ``obtaining patient consent for the processing of data''? \\
    Kontext & Four subfunctions of patient admission (appointment scheduling, patient identification and checking for readmitted patients, administrative admission, and visitor and information services) are supported by the patient administration system, which is a part of the ERPS.
    Medical admission and nursing admission are supported by the MDMS.
    Obtaining consent for processing of patient-related data is supported by the non-computer-based application component for patient data privacy forms.
    This application component is based on paper forms which are scanned by a clerk (see physical tool layer) and then stored in the MDMS.

    The patient administration system, which is the master application system (Sect. 3.9.1) for the entity type ``patient,'' sends the administrative patient data as a message to the MDMS.
    The MDMS can thus store this information about the entity type ``patient'' in its own database; administrative patient data that is needed to support medical admission and nursing admission as functions therefore do not have to be reentered in the MDMS.
    The entity type ``patient'' is both stored in the database systems of the ERPS and the MDMS what is represented by dashed lines between the domain layer and the logical tool layer.

    Both the patient administration system and the MDMS are run on servers at a virtualized server farm (see relationships between logical and physical tool layer). The application systems can be accessed by different end devices (patient terminal, PC, tablet PC).

    It therefore simplifies some aspects which might be relevant in other contexts.
    Another visualization of relationships between 3LGM\textasciicircum{}2 model elements is the matrix view.
    The patient administration system supports three different functions, the MDMS supports two functions, and one function is supported by the paper-based patient data privacy form system.
    The matrix view also helps to identify incomplete parts of models.
    We can see that there are no functions modeled that are supported by the financial accounting system, the human resources management system, and the material management system, which are parts of the ERPS.

    The matrix view is an alternative representation of configuration lines between functions at the domain layer and application components at the logical tool layer.
    Matrix views are also available for visualizing relations between other pairs of connected 3LGM\textasciicircum{}2 classes.\\
    Antwort & (a) The function ``patient admission'' is decomposed into five subfunctions - patient admission is only complete if all the subfunctions are completed.
    The entity type ``patient'' is decomposed into four entity types - data regarding that entity type are only complete if data about all sub-entity types is complete.
    There are no examples of specialization at the domain layer of Fig. 2.11.

    (b) The function ``patient identification'' updates the entity type ``patient.'' The function ``medical admission'' uses the entity type ``patient.'' This indicates that identifying patient data are updated or created during patient admission and then used for medical admission.
        
    (c) The entity type ``privacy statement'' could be added at the domain layer.
    It would be updated by the function ``obtaining consent for processing of patient data.''
        
    (d) Patient admission is decomposed into five subfunctions with each being linked to an application component by which it is supported.
    Therefore, it could lead to an ambiguous model if the superordinated function was linked with another application component.
    The corresponding modeling rule says that only the leaf functions in a function hierarchy should be linked to application components.
        
    (e) The function ``obtaining patient consent for the processing of data'' is supported by the paper-based patient data privacy form system.
    For this, a paper record cabinet, a scanner, and a clerk handling these tools are the physical data processing systems needed for the function.\\
    Quelle|Anz. Antw. &  Book  | 5 \\
    \midrule
    Frage & Imagine a hospital information system that comprises four application systems: a PAS, an MDMS, a RIS, and a PDMS.
    The hospital is now considering the introduction of a communication server to improve data integration.
    Discuss the short-term and long-term pros and cons of this decision.
    Which syntactic and semantic standards could be used? \\
    Umformuliert & Imagine a hospital information system that comprises four application systems: a PAS, an MDMS, a RIS, and a PDMS.
    The hospital is now considering the introduction of a communication server to improve data integration.
    Discuss the short-term and long-term pros and cons of this decision.
    Which syntactic and semantic standards could be used? \\
    Antwort & Short-term advantages: The communication server can handle the communication between all four application systems, including receiving, buffering, transforming, and multicasting of messages.
    It can also be used for monitoring the communication traffic.
    The communication server thus supports data integration in heterogeneous information system architectures.

    Long-term advantages: In the resulting (ACn, CP1) architecture, new application components can easily be integrated, as only one communication interface to the communication server needs to be implemented.

    Standards: For the exchange of administrative data, HL7 V2 or V3 could be used as syntactic or semantic standard.
    For the exchange of clinical data, various communication standards can be chosen such as HL7 FHIR, DICOM for medical images, or HL7 CDA for clinical documents.\\
    Quelle|Anz. Antw. &  Book  | 3 \\
    \midrule
    Frage & The following questions can be answered by reading the text and analyzing the 3LGM\textasciicircum{}2 figures of the CityCare Example 3.11.

    (a) The EHRS and the VNA in CityCare are not linked with any function they support.
    Which function of the domain layer may (partly) be supported by these application systems? Which functions (as introduced in Sect. 3.3) that are supported by these application systems could be added at the domain layer?

    (b) In which database systems shown in the logical tool layer should the entity type ``patient'' be stored?

    (c) The MPI should receive messages containing PINs (entity type ``patient'') from all patient administration systems.
    Why is there no communication link between the MPI and the patient administration system of Ernst Jokl Hospital?

    (d) According to the matrix view, which functions are supported redundantly in CityCare? Discuss pros and cons of the functional redundancies in this scenario.
    What redundancies would you resolve and how?

    (e) Which functions in which health care facility cannot be performed anymore if ``Application Server 1 Ernst Jokl Hospital'' fails? Suggest a change to the physical tool layer that would minimize the risk of missing function support in case a single application server fails.

    (f) For the CityCare network, would it make sense to implement further profiles from IHE? Explain your decision.\\
    Umformuliert & The following questions can be answered reading the provided text.
    (a) The EHRS and the VNA in CityCare are not linked with any function they support.
    Which function of the domain layer may (partly) be supported by these application systems? Which functions (as introduced in Sect. 3.3) that are supported by these application systems could be added at the domain layer?

    (b) In which database systems shown in the logical tool layer should the entity type ``patient'' be stored?

    (c) The MPI should receive messages containing PINs (entity type ``patient'') from all patient administration systems.
    Why is there no communication link between the MPI and the patient administration system of Ernst Jokl Hospital?

    (d) According to the matrix view, which functions are supported redundantly in CityCare? Discuss pros and cons of the functional redundancies in this scenario.
    What redundancies would you resolve and how?

    (e) Which functions in which health care facility cannot be performed anymore if ``Application Server 1 Ernst Jokl Hospital'' fails? Suggest a change to the physical tool layer that would minimize the risk of missing function support in case a single application server fails.

    (f) For the CityCare network, would it make sense to implement further profiles from IHE? Explain your decision.\\
    Kontext & Four subfunctions of patient admission (appointment scheduling, patient identification and checking for readmitted patients, administrative admission, and visitor and information services) are supported by the patient administration system, which is a part of the ERPS.
    Medical admission and nursing admission are supported by the MDMS.
    Obtaining consent for processing of patient-related data is supported by the non-computer-based application component for patient data privacy forms.
    This application component is based on paper forms which are scanned by a clerk (see physical tool layer) and then stored in the MDMS.

    The patient administration system, which is the master application system (Sect. 3.9.1) for the entity type ``patient,'' sends the administrative patient data as a message to the MDMS.
    The MDMS can thus store this information about the entity type ``patient'' in its own database; administrative patient data that is needed to support medical admission and nursing admission as functions therefore do not have to be reentered in the MDMS.
    The entity type ``patient'' is both stored in the database systems of the ERPS and the MDMS what is represented by dashed lines between the domain layer and the logical tool layer.

    Both the patient administration system and the MDMS are run on servers at a virtualized server farm (see relationships between logical and physical tool layer). The application systems can be accessed by different end devices (patient terminal, PC, tablet PC).

    It therefore simplifies some aspects which might be relevant in other contexts.
    Another visualization of relationships between 3LGM\textasciicircum{}2 model elements is the matrix view.
    The patient administration system supports three different functions, the MDMS supports two functions, and one function is supported by the paper-based patient data privacy form system.
    The matrix view also helps to identify incomplete parts of models.
    We can see that there are no functions modeled that are supported by the financial accounting system, the human resources management system, and the material management system, which are parts of the ERPS.

    The matrix view is an alternative representation of configuration lines between functions at the domain layer and application components at the logical tool layer.
    Matrix views are also available for visualizing relations between other pairs of connected 3LGM\textasciicircum{}2 classes.\\
    Antwort & (a) EHR systems as comprehensive application systems combine the functionalities of MDMS, NDMS, and CPOE systems.
    The EHRS of CityCare could therefore be used for medical admission, preparation of an order, or execution of diagnostic and therapeutic procedures.
    However, each of the three health care facilities in CityCare has its own MDMS.
    Therefore, the EHRS is probably mainly used for accessing findings from the other health care facilities, for example, during medical admission.
    For the VNA, no suitable function is modeled at the domain layer.
    At the domain layer, archiving of patient information could be added which is supported by the VNA and, to some extent, also by the EHRS.

    (b) The entity type ``patient'' represents the persons who are the subject of health care.
    Information about a patient includes the PIN and other administrative data about the person.
    Each of the application systems supporting subfunctions of patient care and having an own database system stores the entity type ``patient,'' for example, the patient administration system including the MPI, the MDMS, the EHRS, and the VNA.

    (c) In Ernst Jokl Hospital, there is a star architecture at the logical tool layer, i.e., a communication server is used for the exchange of messages between application systems.
    The patient administration system of Ernst Jokl Hospital, where the PINs of Ernst Jokl Hospital are generated, sends this information in a message to the communication server.
    The communication server forwards the message to the MPI of the health care network.
    In the central MPI of the tHIS, the local patient identification numbers of the different health care facilities are linked to the unique transinstitutional patient identification number of CityCare.

    (d) Administrative admission, appointment scheduling, medical admission, order entry, patient identification, and preparation of an order are each supported by at least three application systems in the scenario.
    Pros (examples):
    - Each of the health care facilities has a functioning information system that is independent from changes or system failures in the other health care facilities.
    - The different patient administration systems and medical documentation and management systems may be better adapted to the local needs and grown structures in the single health care facilities than an application system that is used by all of them together.

    Cons (examples):
    - Three or more different application systems that support the same function cause higher costs and higher administrative effort.
    - The effort for establishing integration and interoperability are higher in functional redundant architectures which have a high number of single application systems.\\
    & Resolving these redundancies (examples):
    - One patient administration system that supports patient identification and administrative admission could be used in all health care facilities instead of three patient administration system and an MPI.
    - The central EHRS could be used as MDMS, NDMS as well as CPOE system in each of the facilities and would replace the existing local application systems.

    (e) According to the matrix view in Fig. 3.37, the MDMS of Ploetzberg Hospital is installed on application server 1 Ernst Jokl Hospital.
    Thus, if this application server 1 Ernst Jokl Hospital fails, the following functions cannot no longer be performed: appointment scheduling, medical admission, order entry, and preparation of an order (see matrix view in Fig. 3.35).

    The application systems used in CityCare should be made available by server clusters with redundant servers.
    If one server in a server cluster fails, another server can take over its task.
    Thus, there is no interruption in function support.
        
    (f) Yes, it makes sense to use further integration profiles from IHE.
    For example, IHE XDS could be used.
    The CityCare network could be established as an affinity domain with several actors that interact in a standardized way (process interoperability) to share document-level or even large binary patient data, such as findings, images, or radiology reports.
    These documents would be registered centrally in a document registry and could be retrieved by other systems.
    Depending on how the central EHRS is implemented, it could either take the role of a do \\
    Quelle|Anz. Antw. &  Book  | 6 \\
    \midrule
    Frage & In Sect. 4.8.1, we presented the structure of the strategic information management plan of Ploetzberg Hospital.
    Compare its structure to the general structure presented in Sect. 4.3.1.2, consisting of strategic goals, description of current state, assessment of current state, future state, and migration path.
    Where can you find this general structure in Ploetzberg Hospital's plan? \\
    Umformuliert & Given the structure of the strategic information management plan of Ploetzberg Hospital.
    Compare its structure to the general structure, consisting of strategic goals, description of current state, assessment of current state, future state, and migration path.
    Where can you find this general structure in Ploetzberg Hospital's plan? \\
    Antwort & - Strategic goals of the health care facility (business goals) and of management of information systems: are visible in Chaps. 1 and 2 of Ploetzberg Hospital's plan.
    - Description of the current state of the information system: are visible in Chap. 3 of Ploetzberg Hospital's plan.
    - Assessment of the current state of the information system: are visible in Chap. 4 of Ploetzberg Hospital's plan.
    - Future state of the information system: are visible in Chap. 5 of Ploetzberg Hospital's plan.
    - Migration path from the current to the planned state: are visible in Chap. 6 of Ploetzberg Hospital's plan.\\
    Quelle|Anz. Antw. &  Book  | 5 \\
    \midrule
    Frage & Look at the health information system's KPIs of Ploetzberg Hospital in Example 4.8.2. Try to figure out some of these numbers for a real hospital and compare both hospitals' KPIs in the form of a benchmarking report.
    It may help to look at the strategic information management plan of this hospital or at its website.\\
    Umformuliert & Look at the health information system's KPIs of Ploetzberg Hospital in the context.
    Try to figure out 10 of these numbers for a real hospital and compare both hospitals' KPIs in the form of a benchmarking report.\\
    Kontext & The CIO of Ploetzberg Hospital annually reports to the hospital's management about the amount, quality, and costs of information processing of the Ploetzberg Hospital information system.
    For this report, the CIO uses health information system KPIs that have been agreed on by a regional group of hospital CIOs (Table 4.3). Each year, the hospitals exchange and discuss their reports as part of a best practice benchmark with other hospitals - this comparison is not shown in the table.

    Table 4.3: Extract from the Ploetzberg Hospital health information system's benchmarking report 2024.
    KPI key performance indicator
    - KPIs for the hospital
    - Number of staff \textbar 5500
    - Number of beds \textbar 1100
    - Number of inpatient cases 40,000
    - Mean duration of stay \textbar 8.1 days
    - Hospital budget \textbar 800 million
    - KPIs for health information system's costs
    - Overall IT costs \textbar 20 million
    - IT costs per inpatient case \textbar 500
    - IT costs in relation to hospital budget \textbar 2.5\%
    - KPIs for health information system's management
    - Number of HIS staff \textbar 46
    - Number of HIS users \textbar 4800
    - Number of workstations \textbar 1350
    - Number of mobile IT tools \textbar 2500
    - HIS user per mobile IT tool \textbar 1.9
    - Number of IT problem tickets \textbar 15,500
    - Percentage of solved IT problem tickets \textbar 96\%
    - Availability of the overall HIS systems \textbar 98.5\%
    - Number of finalized strategic IT projects \textbar 13
    - Percentage of successful IT projects \textbar 76\%
    - KPIs for health information system's functionality
    - Percentage of all documents available electronically \textbar 45\%
    - Percentage of all diagnosis coded electronically \textbar 77\%
    - Functionality index of patient administration system \textbar 52\%
    - Functionality index of MDMS \textbar 87\%
    - KPIs for health information system's architecture
    - Number of computer-based application components \textbar 84
    - Percentage of standard interfaces between applications \textbar 87\%
    - Functional redundancy rate \textbar 0.44 \\
    Antwort & - KPI \textbar Ploetzberg Hospital \textbar My hospital
    - Number of HIS staff \textbar 46 \textbar 89
    - Number of HIS users \textbar 4800 \textbar 9000
    - Number of workstations \textbar 1350 \textbar 6200
    - Number of mobile IT tools \textbar 2500 \textbar 2000
    - HIS user per mobile IT tool \textbar 1.9 \textbar 4.5
    - Number of IT problem tickets \textbar 15,500 \textbar 36,250
    - Percentage of solved IT problem tickets \textbar 96\% \textbar 92\%
    - Availability of the overall HIS systems \textbar 98.5\% \textbar 96\%
    - Number of finalized strategic IT projects \textbar 13 \textbar 10
    - Percentage of successful IT projects \textbar 76\% \textbar 86\% \\
    Quelle|Anz. Antw. &  Book  | 10 \\
    \midrule
    Frage & You are asked to organize regular (e.g., every half year) quantitative user feedback on the general user satisfaction with major clinical application components of your hospital as part of health information system's monitoring.
    Which user groups would you consider? How could you gather user feedback regularly in an automatic way? Explain your choice.\\
    Umformuliert & You are asked to organize regular (e.g., every half year) quantitative user feedback on the general user satisfaction with major clinical application components of your hospital as part of health information system's monitoring.
    Which user groups would you consider? How could you gather user feedback regularly in an automatic way? Explain your choice.\\
    Antwort & User groups: physicians, nurses, technical staff (e.g., lab, radiology), and management staff - these groups are typically large health information systems user groups. I would also organize regular survey of CIS key users, as they are experts in judging the quality of the information systems.

    Organization of user feedback: (1) Health information system users are randomly invited to an automatic short and standardized survey that is displayed during CIS login.
    (2) Every half year, I would organize sounding boards (a structured approach to obtain active feedback from stakeholders) with key users and with representatives from the larger user groups to discuss recent challenges with the CIS and opportunities for improvements.\\
    Quelle|Anz. Antw. &  Book  | 2 \\
    \midrule
    Frage & Read the following case descriptions and discuss the integration problems using the types of integration presented in Sect. 5.3.4. Which negative effects for information logistics result from the identified integration problems?

    1. A physician enters a medical diagnosis for a patient first in the medical documentation and management system (MDMS) and later, when ordering an X-ray, again in the CPOE system.

    2. The position of the patient's name and the formatting of the patient's birthdate vary between the MDMS and the CPOE system.

    3. When physicians shift from the MDMS to the CPOE system, they have to log in again and again search for the correct patient.

    4. The CPOE system and the RIS use slightly different catalogs of available radiology examinations.

    5. When physicians write the discharge letter for a patient in the MDMS, they also have to code the discharge diagnosis of a patient.
    For this coding, they have to use a feature that is only available in the patient administration system, so they have to shift to this application system.

    6. While at the patient's bedside during their ward rounds, physicians have to use several application components at the same time, such as MDMS for retrieving recent findings, the CPOE system for ordering, and the PACS for retrieving images.\\
    Umformuliert & Read the following case descriptions and discuss the integration problems using the types of integration.
    Which negative effects for information logistics result from the identified integration problems?

    1. A physician enters a medical diagnosis for a patient first in the medical documentation and management system (MDMS) and later, when ordering an X-ray, again in the CPOE system.

    2. The position of the patient's name and the formatting of the patient's birthdate vary between the MDMS and the CPOE system.

    3. When physicians shift from the MDMS to the CPOE system, they have to log in again and again search for the correct patient.

    4. The CPOE system and the RIS use slightly different catalogs of available radiology examinations.

    5. When physicians write the discharge letter for a patient in the MDMS, they also have to code the discharge diagnosis of a patient.
    For this coding, they have to use a feature that is only available in the patient administration system, so they have to shift to this application system.

    6. While at the patient's bedside during their ward rounds, physicians have to use several application components at the same time, such as MDMS for retrieving recent findings, the CPOE system for ordering, and the PACS for retrieving images.\\
    Antwort & 1. A physician enters a medical diagnosis for a patient first in the MDMS and later, when ordering an X-ray, again in the CPOE system.
    -\textgreater No data integration, resulting in reentering of data, which is time-consuming and may lead to errors and inconsistencies in the data, which has the potential for patient harm.
        
    2. The position of the patient's name and the formatting of the patient's birthdate vary between the MDMS and the CPOE system.
    -\textgreater No user interface integration, resulting in increased time effort when using various application components, increased time needed for user training, and increased risk in overlooking or misinterpreting important patient information, which has the potential for patient harm.
        
    3. When physicians shift from the MDMS to the CPOE system, they have to log in again and again search for the correct patient.
    -\textgreater No context integration, leading to an increase in time needed to shift between application systems and an increased risk for selecting the wrong patient in the second application systems, which has the potential for patient harm.
        
    4. The CPOE system and the RIS use slightly different catalogs of available radiology examinations.
    -\textgreater No semantic integration, making the exchange and reuse of patient information in both application systems challenging.
        
    5. When physicians write the discharge letter for a patient in the MDMS, they also have to code the discharge diagnosis of a patient.
    For this coding, they have to use a feature that is only available in the patient administration system, so they have to shift to this application system.
    -\textgreater No feature integration, leading to increased time needed to shift to the patient administration system.
        
    6. While being at the patient's bedside during their ward rounds, physicians have to use several application components at the same time, such as MDMS for retrieving recent findings, the CPOE system for ordering, and the PACS for retrieving images.
    -\textgreater No process integration; a process should be organized in a way that frequent change of application systems is avoided if possible.\\
    Quelle|Anz. Antw. &  Book  | 6 \\
    \midrule
    Frage & Skizzieren Sie die Fachliche Ebene und die Logische Werkzeugebene des folgenden Szenarios als 3LGM\textasciicircum{}2-Modell auf dem nächsten Blatt (9 Punkte).

        1. Die administrative Patientenaufnahme erfolgt mit dem Patientenverwaltungssystem.
    Die administrativen Patientendaten, repräsentiert durch den Objekttyp ``Patient'' werden vom Patientenverwaltungssystem über den Kommunikationsserver an das CPOE-System, das Medizinische Dokumentationssystem und das Laborinformationssystem gesendet.
    Auf Papier mitgebrachte Vorbefunde werden bei der administrativen Patientenaufnahme eingescannt und im Medizinischen Dokumentationssystem gespeichert. 

        2. Ergänzen Sie je eine Aufgabe des CPOE-Systems, des Medizinischen Dokumentationssystems und des Laborinformationssystems (inkl.
    Konfigurationslinien).

        3. Ergänzen Sie einen passenden Objekttyp sowie je zwei sinnvolle ``bearbeitet''- und ``nutzt''-Beziehungen zwischen Objekttypen und Aufgaben.\\
    Umformuliert & Given the scenario from the context, list out the used application components and tasks they perform.
    Order these in the Domain Layer and Logical Tool Layer of the 3LGM\textasciicircum{}2 model.
    Add one task each of the CPOE system, the medical documentation system and the laboratory information system
    Add a suitable object type as well as two meaningful ``updates'' and ``uses'' relationships between object types and tasks.\\
    Kontext & The administrative patient admission takes place with the patient management system.
    The administrative patient data, represented by the object type ``patient'', are sent from the patient management system via the communication server to the CPOE system, the medical documentation system and the laboratory information system.
    Preliminary findings brought in on paper are scanned during administrative patient admission and stored in the medical documentation system.\\
    Antwort & Domain Layer:
    - Patient Admission

    Logical Tool Layer:
    - Patient Management System
    - Communication Server
    - CPOE System
    - Medical Documentation System
    - Laboratory Information System


    CPOE System:
    - Task: Creation/Administration of Medication plans

    Medical Documentation System:
    - Task: Administration/Storage of Patientanamnesis

    Laboratory Information System:
    - Task: Storage/Processing of Laboratory Requests

    Object Type:
    - Patient
        - Uses: Administration/Storage of Patientanamnesis, Creation/Administration of Medication plans
        - Processed: Patient Admission \\
    Quelle|Anz. Antw. &  IS\_2022\_07\_18  | 1 \\
    \midrule
    Frage & Nennen Sie 3 Dinge, die die Hausärztin Frau Meier in ihrer Praxis benötigt, um auf die in der Telematikinfrastruktur gespeicherten Daten des Patienten Herrn Schulz zuzugreifen? \\
    Umformuliert & Name 3 things that the GP Ms Meier needs in her practice in order to access the data of the patient Mr Schulz stored in the telematics infrastructure.\\
    Antwort & - electronic health professional card / institution card
    - Access to telematic infrastructure (such as PC)
    - Insured Person's permission to access their data \\
    Quelle|Anz. Antw. &  IS\_2022\_07\_18  | 3 \\
    \midrule
    Frage & Ordnen Sie folgende Begriffe und Definitionen einander zu.

        1. Datenintegrität, Interoperabilität, Syntaktische Interoperabilität, Referentielle Integrität, Semantische Integration, Datenintegration

        2. Fähigkeit eines Anwendungssystems (AWS), Informationen mit anderen AWS auszutauschen und zu nutzen

        3. Zustand eines Informationssystems, in dem Daten, die einmal erfasst wurden, überall  verfügbar sind, wo sie benötigt werden

        4. Zustand eines Informationssystems, in dem interoperable Anwendungssysteme das gleiche Begriffssystem nutzen

        5. Fähigkeit eines Anwendungssystems, Nachrichten mit einer definierten Struktur auszutauschen

        6. Korrektheit der Daten

        7. Korrekte und eindeutige Zuordnung eines Objekts zu anderen Objekten \\
    Umformuliert & Match the following terms and definitions.
    Terms:
    data integrity, interoperability, syntactic interoperability, referential integrity, semantic integration, data integration.
    Definitions:

    1. ability of an application system (AWS) to exchange and use information with other AWSs

    2. state of an information system in which data, once captured, is available wherever it is needed

    3. state of an information system in which interoperable application systems use the same conceptual system

    4. ability of an application system to exchange messages with a defined structure

    5. correctness of data

    6. correct and unambiguous assignment of an object to other objects \\
    Antwort & Data Integrity: 5. correctness of data

    Interoperability: 1. ability of an application system (AWS) to exchange and use information with other AWSs

    Syntactic Interoperability: 4. ability of an application system to exchange messages with a defined structure

    Referential Integrity: 6. correct and unambiguous assignment of an object to other objects

    Semantic Integration: 3. state of an information system in which interoperable application systems use the same conceptual system

    Data Integration: 2. state of an information system in which data, once captured, is available wherever it is needed \\
    Quelle|Anz. Antw. &  IS\_2022\_07\_18  | 6 \\
    \midrule
    Frage & Erklären Sie für einen Interoperabilitätsstandard im Gesundheitswesen dessen Einsatzgebiet, Funktionsprinzip und die unterstützten Interoperabilitätsarten.\\
    Umformuliert & For a healthcare interoperability standard, explain its area of use, operating principle and the types of interoperability supported.\\
    Antwort & Health Level 7 Version 2:
    Area of Use:
    - Communication between Application Systems

    operating principle:
    - Message based
    - Event driven

    Types of Interoperability:
    - Syntactic Interoperability
    - Semantic Interoperability \\
    Quelle|Anz. Antw. &  IS\_2022\_07\_18  | 4 \\
    \midrule
    Frage & Vergleichen Sie HL7 V2 und HL7 FHIR.
    Erläutern Sie mindestens drei Unterschiede.\\
    Umformuliert & Compare HL7 V2 and HL7 FHIR.
    Explain at least three differences \\
    Antwort & - both used to exchange health information between application systems
    - Syntax HL7 v2: proprietary, ASCII-text
    - Syntax HL7 FHIR: XML / JSON  = easier to implement
    - Structure HL7 v2: fixed, hierarchical, fixed amount of segments and codes
    - Structure HL7 FHIR: flexible, modular, resource/element based \\
    Quelle|Anz. Antw. &  IS\_2022\_07\_18  | 3 \\
    \midrule
    Frage & Skizzieren Sie die Fachliche Ebene und die Logische Werkzeugebene des folgenden Szenarios als 3LGM\textasciicircum{}2-Modell auf dem nächsten Blatt.

        1. Die administrative Patientenaufnahme erfolgt mit dem Patientenverwaltungssystem.
    Die administrativen Patientendaten, repräsentiert durch den Objekttyp ``Patient'' werden vom Patientenverwaltungssystem über den Kommunikationsserver an das CPOE-System, das Medizinische Dokumentationssystem und das Radiologieinformationssystem gesendet.
    Auf Papier mitgebrachte Vorbefunde werden bei der administrativen Patientenaufnahme eingescannt und im Medizinischen Dokumentationssystem gespeichert.

        2. Ergänzen Sie je eine Aufgabe des CPOE-Systems, des Medizinischen Dokumentationssystems und des Radiologieinformationssystems (inkl.
    Konfigurationslinien).

        3. Ergänzen Sie einen passenden Objekttyp sowie je zwei sinnvolle ``bearbeitet''- und ``nutzt''-Beziehungen zwischen Objekttypen und Aufgaben.\\
    Umformuliert & Describe the functional level and the logical tool level of the following scenario as a 3LGM\textasciicircum{}2 model.

    1. The administrative patient admission takes place with the patient management system.
    The administrative patient data, represented by the object type ``patient'', are sent from the patient management system via the communication server to the CPOE system, the medical documentation system and the radiology information system.
    Preliminary findings brought in on paper are scanned during administrative patient admission and stored in the medical documentation system.

    2. Add one task each of the CPOE system, the medical documentation system and the radiology information system (incl.
    configuration lines).

    3. Add a suitable object type as well as two meaningful ``updates'' and ``uses'' relationships between object types and tasks.\\
    Kontext & The administrative patient admission takes place with the patient management system.
    The administrative patient data, represented by the object type ``patient'', are sent from the patient management system via the communication server to the CPOE system, the medical documentation system and the radiology information system.
    Preliminary findings brought in on paper are scanned during administrative patient admission and stored in the medical documentation system.\\
    Antwort & Domain Layer:
    - Administrative Patient Admission

    Logical Tool Layer:
    - Patient Management System
    - Communication Server
    - CPOE System
    - Medical Documentation System
    - Radiology Information System


    CPOE System:
    - Task: Creation/Administration of Medication plans

    Medical Documentation System:
    - Task: Administration/Storage of Patientanamnesis

    Radiology Information System:
    - Task: Archive/Processing of Radiological Images

    Object Type:
    - Medication Plan
        - Uses: Administration/Storage of Patientanamnesis
        - Processed: Creation/Administration of Medication plans \\
    Quelle|Anz. Antw. &  IS\_2022\_09\_27  | 1 \\
    \midrule
    Frage & Wer stellt die ``elektronische Patientenakte (ePA)'' nach § 341 SGB V zur Verfügung? Wie wird der Zugriff auf die enthaltenen medizinischen Daten geregelt? \\
    Umformuliert & Who provides the ``electronic patient file (ePA)'' according to § 341 SGB V? How is access to the medical data it contains regulated? \\
    Antwort & - provided and managed by the health insurance companies
    - Technical and organizational measures to ensure only authorized access
    - Access rights, role-based access control, documentation of access \\
    Quelle|Anz. Antw. &  IS\_2022\_09\_27  | 2 \\
    \midrule
    Frage & Was ist IHE und welchen Nutzen hat es? Erläutern Sie den Zusammenhang zwischen IHE und Interoperabilitätsstandards.\\
    Umformuliert & What is IHE and what are its benefits? Explain the relationship between IHE and interoperability standards.\\
    Antwort & IHE stands for Integrating the Healthcare Enterprise.
    It is an initiative by healthcare professionals and industry to improve the way computer systems in healthcare share information.
    IHE promotes the coordinated use of established standards such as DICOM and HL7 to address specific clinical needs in support of optimal patient care \\
    Quelle|Anz. Antw. &  IS\_2022\_09\_27  | 2 \\
    \midrule
    Frage & Was wird standardisiert? \\
    Umformuliert & What are common standards used in health information systems? \\
    Antwort & HL7v2
    CDA
    HL7 FHIR
    DICOM
    ISO/IEEE 11073
    CCOW (Clinical Context Object Workgroup)
    EDIFACT (Electronic Data Interchange for Administration, Commerce and Transport)
    SNOMED, LOINC
    openEHR
    CDISC (Clinical Data Interchange Standards Consortium) \\
    Quelle|Anz. Antw. &  IS\_2022\_09\_27  | 1 \\
    \midrule
    Frage & Was sind Daten, Informationen, Wissen? \\
    Umformuliert & What are data, information, knowledge? \\
    Antwort & Data are characters, discrete numbers, or continuous signals to be processed in information systems.
    Information is a context-specific fact about entities such as events, things, persons, processes, ideas, or concepts.
    Information is represented by data.
    Knowledge is general information about concepts in a certain (scientific or professional) domain (e.g., knowledge about diseases or therapeutic methods) at a certain time.\\
    Quelle|Anz. Antw. &  A\_2021  | 3 \\
    \midrule
    Frage & Was sind System, soziotechnisches System \\
    Umformuliert & What are system and socio-technical system? \\
    Antwort & A system is a set of persons, things, events, and their relationships forming an integrated whole.
    If a (human-made) system consists of both human and technical components, it can be called a socio-technical system.\\
    Quelle|Anz. Antw. &  A\_2021  | 2 \\
    \midrule
    Frage & Wie geht 3LGM\textasciicircum{}2 genauer? \\
    Umformuliert & How dows 3LGM\textasciicircum{}2 work? \\
    Antwort & 3LGM\textasciicircum{}2 is a three-layer graph-based metamodel for modeling (health) information systems.
    It combines function, technical and organizational aspects with certain aspects of data dn process metamodels.
    It distinguishes between the following layers:
    - Domain Layer (activities in a health care setting)
    - Logical Tool Layer (application components)
    - Physical Tool Layer (Physical data processing systems and their data transmission links) \\
    Quelle|Anz. Antw. &  A\_2021  | 3 \\
    \midrule
    Frage & Was ist eine datenverarbeitende Aufgabe (enterprise function)? Bitte nennen Sie Beispiele \\
    Umformuliert & What is a data processing task (enterprise function)? Please give examples \\
    Antwort & Enterprise functions mainly emphasize the contribution of activities to business goals.
    Examples are Administrative Admission or Patient Care.\\
    Quelle|Anz. Antw. &  A\_2021  | 2 \\
    \midrule
    Frage & Bitte erläutern Sie die Aufgabe ``Patientenaufnahme'' \\
    Umformuliert & please explain the task ``patient admission'' \\
    Antwort & Patient admission updates and uses the entity type patient
    Consists of refined subfunctions such as Nursing Admission.
    Example Activity of Patient Admission is ``Physician admits Patient \\
    Quelle|Anz. Antw. &  A\_2021  | 2 \\
    \midrule
    Frage & Nennen Sie Beispiele für Objekttypen in einem Krankenhausinformationssystem \\
    Umformuliert & Give examples of object types in a hospital information system \\
    Antwort & Patient
    Doctor
    Medicine Plan
    Medical Record \\
    Quelle|Anz. Antw. &  A\_2021  | 3 \\
    \midrule
    Frage & Bitte skizzieren Sie eine typische LWE \\
    Umformuliert & Describe the logical tool layer.\\
    Antwort & Consists of Application Systems, or in broader sense, application component as the center of interest.
    Represent certain application software products on a certain computer system.

    Connected via Message Oriented Communication using communication interfaces.
    Or Service-Oriented Communication by providing features to other application systems.\\
    Quelle|Anz. Antw. &  A\_2021  | 4 \\
    \midrule
    Frage & Bitte erläutern Sie die Begriffe Integration, Interoperabilität, Integrität. \\
    Umformuliert & Please explain the terms integration, interoperability, integrity.\\
    Antwort & Integration means that the application systems are put together in such a way that the resulting information system - as opposed to its parts - displays a new quality.
    Interoperability in general is the ability of two application systems to exchange information with each other and to use the information that has been exchanged.
    Data integrity means that data are consistent, that object identity is maintained, and that relationships between entities are correct (referential integrity) \\
    Quelle|Anz. Antw. &  A\_2021  | 3 \\
    \midrule
    Frage & Welche Arten von Integrität haben wir diskutiert? \\
    Umformuliert & What types of integrity exist in the context of health information systems? \\
    Antwort & Object Identity, Referential Integrity, Consistency \\
    Quelle|Anz. Antw. &  A\_2021  | 3 \\
    \midrule
    Frage & Welche Typen der Integration haben wir diskutiert? \\
    Umformuliert & What types of integration exist in the context of health information systems? \\
    Antwort & Data-Integration, Semantic Integration, User-Interface Integration, Context Integration, Functional Integration, Process Integration \\
    Quelle|Anz. Antw. &  A\_2021  | 6 \\
    \midrule
    Frage & Nennen Sie zwei Kommunikationsstandards für KIS? \\
    Umformuliert & Name two communication standards for HIS? \\
    Antwort & HL7, DICOM \\
    Quelle|Anz. Antw. &  A\_2021  | 2 \\
    \midrule
    Frage & Wie hängen 3LGM\textasciicircum{}2, IHE zusammen \\
    Umformuliert & How are 3LGM\textasciicircum{}2 and IHE related? \\
    Antwort & 3LGM\textasciicircum{}2 is used to model Health Information Systems, while IHE describes Standards for Health Information Systems.\\
    Quelle|Anz. Antw. &  A\_2021  | 2 \\
    \midrule
    Frage & Wie hängen HL7, DICOM zusammen \\
    Umformuliert & How are HL7, DICOM related? \\
    Antwort & Both are Communication Standards commonly used in a Health Information System \\
    Quelle|Anz. Antw. &  A\_2021  | 2 \\
    \midrule
    Frage & Wie hängen IHE - Objektidentität - tHIS zusammen? \\
    Umformuliert & How are IHE - object identity - tHIS related? \\
    Antwort & IHE provides standard for Health Information Systems to improve communication between HIS.
    tHIS describes a System that includes multiple HIS.\\
    Quelle|Anz. Antw. &  A\_2021  | 2 \\
    \midrule
    Frage & Welche Bedeutung hat die Softwareentwicklung in einem Krankenhaus? \\
    Umformuliert & What is the importance of software development in a hospital? \\
    Antwort & While software development of fully integrated application components are prohibited by law due to missing certificates, adjusting software to the Needs of each Hospital results in better communication, usage and therefore results during operation.\\
    Quelle|Anz. Antw. &  A\_2021  | 2 \\
    \midrule
    Frage & Was ist Transaktionsmanagement? Wie wird das in KIS organisiert? \\
    Umformuliert & What is transaction management? How is it organized in HIS? \\
    Antwort & Transactions management ensures ACID properties during communication.
    This is managed by the Communication Server.
    Each data entity is administered by one application component and changes are broadcast to all application components that use it.\\
    Quelle|Anz. Antw. &  A\_2021  | 2 \\
    \midrule
    Frage & Nennen Sie drei Anwendungssysteme und die jeweils unterstützten Aufgaben! \\
    Umformuliert & Name three application systems and the tasks each supports! \\
    Antwort & Patient Administration System, Patient Data Management System, Laboratory Information System, Radiology Information System, etc. \\
    Quelle|Anz. Antw. &  A\_2021  | 3 \\
    \midrule
    Frage & Wie sorgen Sie für Ausfallsicherheit in einem KIS? \\
    Umformuliert & How do you ensure fail-safety in a HIS? \\
    Antwort & Redundancy across.
    Multiple Hardware Components that mirror each other.
    Separated Infrastructure for energy.\\
    Quelle|Anz. Antw. &  A\_2021  | 3 \\
    \midrule
    Frage & Wie kann man Krankenhausinformationssysteme vergleichen? \\
    Umformuliert & How can hospital information systems be compared? \\
    Antwort & (Referenzmodelle, Taxonomy) \\
    Quelle|Anz. Antw. &  A\_2021  | 2 \\
    \bottomrule
    \caption*{Evaluierungsdatensatz (Multifakten-Fragen)}\label{tab:evaldata-multi}
\end{longtable}
\end{landscape}

\begin{landscape}
    \begin{longtable}{p{3cm}p{1.7\textwidth}}
    \toprule
    \multicolumn{2}{c}{\textbf{Evaluierungsdatensatz (Transferfragen)}}\\
    \midrule
    Frage & Look at the functions presented in Sect. 3.3.2. Now imagine a small hospital (e.g., 350 beds) and a large university medical center (e.g., 1500 beds). What are the differences between these hospitals with regard to their functions? Explain your answer.\\
    Umformuliert & Imagine a small hospital (e.g., 350 beds) and a large university medical center (e.g., 1500 beds). What are the differences between these hospitals with regard to their functions to be performed by health care professionals and other staff in health care facilities? Explain your answer.\\
    Antwort & A typical hospital needs all functions to function as expected.
    The functions to be performed by health care professionals are mostly similar in all health care facilities, independent of their size.
    Only some functions may differ.
    For example, not all health care facilities are involved in clinical research, thus their information will not need to support the function research and education.\\
    Quelle|Anz. Antw. &  Book  | 2 \\
    \midrule
    Frage & Auf einer Hersteller-Webseite heißt es: ``i.s.h.med ist das einzige vollständig in SAP for Healthcare integrierte Krankenhausinformationssystem''. Welches Begriffsverständnis liegt hier zugrunde? \\
    Umformuliert & On a manufacturer's website it says: ``i.s.h.med is the only hospital information system fully integrated in SAP for Healthcare''. What is the underlying understanding of this term? \\
    Antwort & This manufacturer sees the hospital information system as software product, while in reality a HIS includes Software, Hardware and Actors.\\
    Quelle|Anz. Antw. &  IS\_2022\_07\_18  | 1 \\
    \midrule
    Frage & Wann ist ein Modell gut? \\
    Umformuliert & When is a model good in the context of health information systems? \\
    Antwort & A good models should be able to help understand and predict the behavior of the system or process.
    It should also be able to help design and evaluate health information systems. A reference architecture can be used to support the design of a proper HIS architecture that meets the various stakeholder concerns of HISs.
    This architecture should be able to show the HIS from a different angle, suitable for various stakeholders.\\
    Quelle|Anz. Antw. &  IS\_2022\_09\_27  | 3 \\
    \midrule
    Frage & Was hat das Krankenhausinformationssystem mit einer ganzheitlichen Sicht auf den Patienten zu tun? \\
    Umformuliert & What does the hospital information system have to do with a holistic view of the patient? \\
    Antwort & A hospital information system must provide the right information about patients in the right place to the right people at the right time.
    Ideally, this means that information about the patient is also taken into account holistically across departmental and case boundaries, e.g. for the optimal treatment of multimorbid patients and for the avoidance of side effects that can occur due to known allergies and multimedication and, in the worst case, lead to death.
    Additional costs, effort and patient treatment due to superfluous multiple examinations are avoided.\\
    Quelle|Anz. Antw. &  IS\_2022\_09\_27  | 2 \\
    \midrule
    Frage & Welche Kommunikation erfolgt unmittelbar nach der Aufnahme des Patienten Alfred Winter? \\
    Umformuliert & What communication takes place immediately after the admission of the patient Alfred Winter? \\
    Antwort & Admission is divided into administrative, medical and nursing admission.
    Via the (possibly existing) communication server the changed/newly added information are then broadcasted to related application components such as the patient managements system.\\
    Quelle|Anz. Antw. &  IS\_2022\_09\_27  | 3 \\
    \midrule
    Frage & Look at the functions listed in Sect. 3.3.2. Look at the relationships between the functions and the different health care professional groups (physicians, nurses, administrative staff, others) working in hospitals and medical offices.
    Select one health care professional group and describe which functions are most important for this group.\\
    Umformuliert & Look at the relationships between functions to be performed by health care professionals and other staff in health care facilities and the different health care professional groups (physicians, nurses, administrative staff, others) working in hospitals and medical offices.
    Select one health care professional group and describe which functions are most important for this group.\\
    Antwort & Physicians: Important functions are medical admission, decision-making and patient information, planning and organization of patient treatment, order entry, execution of diagnostic and therapeutic procedures, coding of diagnoses and procedures, and medical discharge and medical discharge summary writing.
    Nurses: Important functions are nursing admission, decision-making and patient information, planning and organization of patient treatment, order entry, execution of nursing procedures, and nursing discharge and nursing discharge summary writing.
    Administrative staff: Important functions are patient identification, administrative admission, and administrative discharge and billing.\\
    Quelle|Anz. Antw. &  Book  | 3 \\
    \midrule
    Frage & Read Examples 5.5.1 and determine which methods for collecting data (as described in Sects. 5.4.3 and 5.4.4) have been used.\\
    Umformuliert & Given the Example in the Context, determine which methods for collecting data have been used.\\
    Kontext & \#\#\# 5.5.1 Unintended Effects of a Computerized Physician Order Entry Nearly Hard-Stop Alert
    The introduction of application systems may have unintended effects.
    The careful evaluation of impact and unintended effects of application systems is thus an important task of management of information systems.
    We will now have a look at an example of an evaluation study that showed some unintended effects of CPOE systems.
    Table 5.1 presents the abstract of an RCT on automatic alerts in a CPOE system.
    The authors analyzed whether the so-called hard-stop alert can reduce unwanted drug-drug interactions.
    Such a ``hard-stop alerts'' appears on the screen to alert the physician about potential problems associated with a particular prescription and blocks the clinician's order from further execution to avert potentially serious reactions.

    Table 5.1: Abstract from ``Unintended Effects of a Computerized Physician Order Entry Nearly Hard-Stop Alert'' [5]
    - Background: The effectiveness of CPOE systems has been modest, largely because clinicians frequently override electronic alerts
    - Methods: To evaluate the effectiveness of a nearly ``hard-stop'' CPOE system prescribing alert intended to reduce concomitant orders for warfarin and trimethoprim-sulfamethoxazole, a randomized clinical trial was conducted at two academic medical centers in Philadelphia, Pennsylvania. A total of 1981 clinicians were assigned to either an intervention group receiving a nearly hard-stop alert or a control group receiving the standard practice.
    The study duration was August 9, 2006, through February 13, 2007
    - Results: The proportion of desired responses (i.e., not reordering the alert-triggering drug within 10 min of firing) was 57.2\% (111 of 194 hard-stop alerts) in the intervention group and 13.5\% (20 of 148) in the control group (adjusted odds ratio, 0.12; 95\% confidence interval, 0.045-0.33). However, the study was terminated early because of four unintended consequences identified among patients in the intervention group: a delay of treatment with trimethoprim-sulfamethoxazole in two patients and a delay of treatment with warfarin in another two patients
    - Conclusions: An electronic hard-stop alert as part of an inpatient CPOE system seemed to be extremely effective in changing prescribing habits.
    However, this intervention precipitated clinically important treatment delays in four patients who needed immediate drug therapy.
    These results illustrate the importance of formal evaluation and monitoring for unintended consequences of programmatic interventions intended to improve prescribing habits

    The study was designed as a quantitative, explanatory field study that was conducted as an RCT.
    The study found that these alerts can help to reduce the number of alert-triggering orders.
    But it also found that the hard-stop alert led to clinically important treatment delays in four patients.\\
    Antwort & Study ``Unintended Effects of a Computerized Physician Order Entry Nearly Hard-stop Alert'': The effectiveness of a nearly ``hard-stop'' alert was evaluated in a field study.
    The data was collected via analysis of the prescriptions in the CPOE systems.
    The overall data collection method is thus a quantitative observation of available data.\\
    Quelle|Anz. Antw. &  Book  | 1 \\
    \midrule
    Frage & Read Examples 5.5.2 and determine which methods for collecting data (as described in Sects. 5.4.3 and 5.4.4) have been used.\\
    Umformuliert & Given the Example in the Context, determine which methods for collecting data have been used.\\
    Kontext & \#\#\# 5.5.2 Clinical Decision Support for Worker Health: A Five-Site Qualitative Needs Assessment in Primary Care Setting
    Besides evaluating the effect of an intervention, evaluation may also try to understand reasons for successful or unsuccessful implementation of an application system.
    For these kinds of questions, qualitative studies are often chosen.
    Table 5.2 presents the abstract of such a qualitative study.
    The authors analyzed need, barriers, and facilitators for clinical decision support (CDS) in primary care.
    The study was performed as a qualitative, exploratory field study.

    Table 5.2: Abstract from ``Clinical Decision Support for Worker Health: A Five-Site Qualitative Needs Assessment in Primary Care Settings.'' [6]

    - Background: Although patients who work and have related health issues are usually first seen in primary care, providers in these settings do not routinely ask questions about work.
    Guidelines to help manage such patients are rarely used in primary care.
    Electronic health record systems (EHRS) with worker health CDS tools have potential for assisting these practices
    - Objective: This study aimed to identify the need for and barriers and facilitators related to implementation of CDS tools for the clinical management of working patients in a variety of primary care settings
    - Methods: We used a qualitative design that included analysis of interview transcripts and observational field notes from 10 clinics in five organizations
    - Results: We interviewed 83 providers, staff members, managers, informatics and IT experts, and leaders and spent 35 h observing.
    We identified eight themes in four categories related to CDS for worker health (operational issues, usefulness of proposed CDS, effort and time-related issues, and topic-specific issues). These categories were classified as facilitators or barriers to the use of the CDS tools.
    Facilitators related to operational issues include current technical feasibility and new work patterns associated with the coordinated care model.
    Facilitators concerning usefulness include users' need for awareness and evidence-based tools, appropriateness of the proposed CDS for their patients, and the benefits of population health data.
    Barriers that are effort-related include the additional time the proposed CDS might take as well as other pressing organizational priorities.
    Barriers that are topic-specific include sensitive issues related to health and work and the complexities of information about work
    - Conclusion: We discovered several themes not previously described that can guide future CDS development: technical feasibility of the proposed CDS within a commercial electronic health record (EHR), the sensitive nature of some CDS content, and the need to assist the entire health care team in managing worker health

    The authors found several factors that may hinder or foster the use of CDS in primary care.
    The results of this multi-center study can now be used to implement CDS in commercial application software products for primary care.\\
    Antwort & Study ``Clinical Decision Support for Worker Health: A Five-Site Qualitative Needs Assessment in Primary Care Setting'': data were collected via interviews and qualitative observations.\\
    Quelle|Anz. Antw. &  Book  | 1 \\
    \midrule
    Frage & In which of the health care settings above will the function medical admission need to be supported? \\
    Umformuliert & In which health care settings will the function medical admission need to be supported? \\
    Antwort & The function ``medical admission'' is relevant in several health care and research settings.
    It comprises the provision of forms for documenting medical history, documenting diagnoses, and scanning documents from referring physician and other sources of information about the medical history.
    It is obvious that this function needs to be supported in hospitals, nursing homes, ambulatory nursing organizations, and medical offices.
    Yet it is often also necessary in research settings, for example, when a person is recruited for a clinical trial and their data are entered into an EDC system.
    Furthermore, therapeutic offices need this function for documentation purposes, as do rehabilitation facilities and-to a limited extent-wellness or sports facilities.
    For personal environments, medical admission also plays a role, especially in telecare situations or when prevention measures are conducted, respectively.\\
    Quelle|Anz. Antw. &  Book  | 4 \\
    \midrule
    Frage & An welcher Stelle im Szenario aus Aufgabe 2 könnte der Dienst ``KIM'' aus der Telematikinfrastruktur Abhilfe schaffen? \\
    Umformuliert & At which point in the scenario from the context could the ``KIM'' service from the telematics infrastructure provide a remedy? \\
    Kontext & The administrative patient admission is done with the patient management system.
    The administrative patient data, represented by the object type ``patient'', are sent from the patient management system via the communication server to the CPOE system, the medical documentation system and the laboratory information system.
    Preliminary findings brought in on paper are scanned during administrative patient admission and stored in the medical documentation system.
    There also exists the CPOE-System, medical documentation system and the laboratory information system.\\
    Antwort & Preliminary findings brought in on paper replaced by the KIM service, which provides secure, digital transmission.\\
    Quelle|Anz. Antw. &  IS\_2022\_07\_18  | 1 \\
    \midrule
    Frage & Was ist ein Krankenhausinformationssystem?
            2.1 Warum ist diese Definition wichtig? \\
    Umformuliert & Why is the definition of a hospital information system important? \\
    Antwort & In most cases, human actors are not included in this definition while being essential for a working HIS.\\
    Quelle|Anz. Antw. &  IS\_2022\_09\_27  | 1 \\
    \midrule
    Frage & Consider a recent health-related situation you were involved in.
    Which life situation does it correspond to and what was your role in this life situation? List some of the requirements you had in this role and in this life situation.\\
    Umformuliert & Consider a recent health-related situation one could be involved in.
    Which life situation does it correspond to and what was your role in this life situation? List some of the requirements you had in this role and in this life situation.\\
    Antwort & My father was admitted to the hospital after suddenly showing symptoms of numbness in the left arm, confusion, and trouble seeing while at home.
    We called the ambulance, and after a short examination, the ambulance team took him to the nearest hospital for further diagnosis and treatment.My father was admitted to the hospital after suddenly showing symptoms of numbness in the left arm, confusion, and trouble seeing while at home.
    We called the ambulance, and after a short examination, the ambulance team took him to the nearest hospital for further diagnosis and treatment.
    This situation corresponds to an emergency life situation. I participated in this situation as a close relative.
    My urgent requirements were to know which hospital my father was taken to and to obtain more information on the suspected diagnosis (here: stroke) and the next steps of diagnosis and therapy.\\
    Quelle|Anz. Antw. &  Book  | 5 \\
    \midrule
    Frage & Imagine that a physician is given the following information about his patient, Mr. Russo: ``Diagnosis: hypertension.
    Last blood pressure measurement: 160/100 mmHg.'' Use this example to discuss the difference between ``data,'' ``information,'' and ``knowledge''! \\
    Umformuliert & Imagine that a physician is given the following information about his patient, Mr. Russo: ``Diagnosis: hypertension.
    Last blood pressure measurement: 160/100 mmHg.'' Use this example to discuss the difference between ``data,'' ``information,'' and ``knowledge''! \\
    Antwort & ``160,'' ``100,'' ``hypertension,'' and ``blood pressure'' represent data that cannot be interpreted without knowledge about the context.
    The information is that Mr. Russo has been diagnosed with hypertension and that his last blood pressure is 160/100 mmHg.
    The medical knowledge embedded in this example is that a blood pressure of 160/100 mmHg indicates hypertension that should be treated.\\
    Quelle|Anz. Antw. &  Book  | 3 \\
    \midrule
    Frage & Consider the requirements of various stakeholders when it comes to health information systems supporting various life situations.
    Can you imagine situations where the requirements of two stakeholder groups differ or even contradict each other? What does this imply when building health information systems? \\
    Umformuliert & Consider the requirements of various stakeholders when it comes to health information systems supporting various life situations.
    Can you imagine situations where the requirements of two stakeholder groups differ or even contradict each other? What does this imply when building health information systems? \\
    Antwort & While a patient is being treated for an acute disease, the requirements of the treating physicians and nurses as well as of the patient and relatives may differ.
    For example, patient and relatives want to be kept informed of ongoing diagnostic outcomes (e.g., lab values) as soon as possible.
    However, physicians and nurses may want to discuss the findings with the patient in person to avoid causing unnecessary confusion and stress in the patient.
    Therefore, the health information system must be able to provide detailed information to physicians and nurses, but it must be able to only present confirmed information to the patient (e.g., via a patient portal). \\
    Quelle|Anz. Antw. &  Book  | 2 \\
    \midrule
    Frage & Look up some information on the nervous system of the human body.
    Then try to identify subsystems of the nervous system.
    In the same way, can you also describe subsystems of the system ``hospital''? \\
    Umformuliert & Whe comparing the nervous system of the human body to the system ``hospital'' the following subsystems can be identified and described: \\
    Antwort & The nervous system comprises two main categories of cells: neurons and glial cells.
    Neurons communicate with each other via synapses and thus form their own subsystem.
    Glial cells form another subsystem that provides support and nutrition to the neurons.

    The hospital can be understood as a system comprising at least two subsystems: the subsystem where clinical care takes place and the subsystem where management takes place.
    The clinical subsystem can again be split into several subsystems, such as inpatient area, outpatient area, and specialized diagnostic or therapeutic areas.
    The inpatient area itself can be divided into various subsystems, each represented by one ward.
    The way I define the subsystems of a hospital depends on the questions or intentions I have.\\
    Quelle|Anz. Antw. &  Book  | 2 \\
    \midrule
    Frage & Imagine a situation in which a physician speaks with Mr. Russo at the patient's bedside.
    The physician looks up Mr. Russo's recent blood pressure measurement and ongoing medication, decides to increase the level of one medication, and explains this to Mr. Russo.
    Use this example to discuss the meaning of ``information and knowledge logistics.'' What in this example indicates the right information, the right place, the right people, the right form, and the right decision? What could happen if an information system does not support high-quality information and knowledge logistics? \\
    Umformuliert & Imagine a situation in which a physician speaks with Mr. Russo at the patient's bedside.
    The physician looks up Mr. Russo's recent blood pressure measurement and ongoing medication, decides to increase the level of one medication, and explains this to Mr. Russo.
    Use this example to discuss the meaning of ``information and knowledge logistics.'' What in this example indicates the right information, the right place, the right people, the right form, and the right decision? What could happen if an information system does not support high-quality information and knowledge logistics? \\
    Antwort & The physician wants to have access to the right information (the most recent blood pressure) at the right time (when talking to Mr. Russo) at the right place (at the patient's bedside) in the right form (hopefully the blood pressure is provided in an easy-to-grasp, visual way) so that he can make the right decision (here: to decide on the level of a certain medication).
    If the information system does not support this, the physician may obtain an incorrect or outdated blood pressure measurement, or he may misinterpret it, thereby coming to a decision that is suboptimal for the patient.\\
    Quelle|Anz. Antw. &  Book  | 2 \\
    \midrule
    Frage & During a night shift, a nurse uses the patient administration system to conduct the administrative patient admission.
    The nurse then uses the NDMS to plan nursing care.
    Now consider the types of integration presented in Sect. 3.8 and discuss how this nurse would recognize a high (or low) level of data integration, semantic integration, user interface integration, context integration, feature integration, and process integration.\\
    Umformuliert & During a night shift, a nurse uses the patient administration system to conduct the administrative patient admission.
    The nurse then uses the NDMS to plan nursing care.
    Now consider the types of integration and discuss how this nurse would recognize a high (or low) level of data integration, semantic integration, user interface integration, context integration, feature integration, and process integration.\\
    Antwort & Data integration would be considered high when the nurse documents patient administrative data only once in the patient administration system and then can use this data in the NDMS.

    Semantic integration would be considered high when the nurse documents a nursing diagnosis using a standardized terminology (such as NANDA) and when this standardized diagnosis is then understood by the NDMS that may, for example, suggest a standard nursing care plan for this patient based on this diagnosis.

    User interface integration would be considered high when the user interfaces of both application systems look sufficiently similar, which reduces the risk of data entry or data interpretation errors.
    For example, in both application systems, the names of the patients are always displayed at the same place, the birthdates are presented in standardized form, and colors that are used to highlight important information are used in the same way.

    Context integration would be considered high when the user context and the patient context is preserved when the nurse shifts from one application system to the other.
    The nurse thus would not have to repeat user login or the selection of the patient in the second application system.

    Feature integration would be considered high when only the patient administration system offers the needed administrative features (such as documentation of patient address). The nurse would be able to call up these features from within the NDMS.

    Process integration would be considered high if both application systems work together in a highly integrated way so that the process of patient admission and nursing care planning from the point of view of the nurse is supported in an efficient way.\\
    Quelle|Anz. Antw. &  Book  | 6 \\
    \midrule
    Frage & Imagine you are the CIO of a hospital in which almost no computer-based tools are used.
    One of the hospital's goals is to support health care professionals in their daily tasks by offering up-to-date patient information at their workplace.
    Which main goals for management of information systems could you define based on this information? Which functions should be prioritized to be supported by new application systems? What could a strategic project portfolio and a migration plan for the next 5 years look like? \\
    Umformuliert & Imagine you are the CIO of a hospital in which almost no computer-based tools are used.
    One of the hospital's goals is to support health care professionals in their daily tasks by offering up-to-date patient information at their workplace.
    Which main goals for management of information systems could you define based on this information? Which functions should be prioritized to be supported by new application systems? What could a strategic project portfolio and a migration plan for the next 5 years look like? \\
    Antwort & Goals: efficient and high-quality information logistics to support patient care.

    Functions: patient administration and all functions related to patient care (Sect. 3.3.2.1).
    Project portfolio and migration plan:

    - Year 1: Introduction of a patient administration system.

    - Year 2: Introduction of a CIS, an LIS and an RIS.

    - Year 3: Introduction of a DAS and a PACS.

    - Year 4: Introduction of an OMS and of a PDMS.

    - Year 5: Introduction of a DWS and of a patient portal.

    Please note: This is a simplified solution.
    Other solutions may be valid, too.
    In case the different application systems are meant to come from different vendors, an integration technology such as a communication server needs to be implemented.\\
    Quelle|Anz. Antw. &  Book  | 5 \\
    \midrule
    Frage & Imagine you are the CIO and have to select the three most relevant indicators for the quality of your information system at your hospital: Which would you select? You can look at the examples in Sect. 4.8.2 to get ideas.
    Explain your choice.\\
    Umformuliert & Imagine you are the CIO and have to select the three most relevant indicators for the quality of your information system at your hospital: Which would you select?
    Explain your choice.\\
    Antwort & Several solutions are possible here.
    One possible solution:

    1. HIS user per mobile IT tool: Efficient information logistics everywhere (e.g., at the patient's bedside) requires enough mobile IT tools.

    2. Number of application systems: I would strive for an integrated information system and reduce the number of application systems in the long run in order to reduce integration problems.

    3. HIS budget in relation to the overall hospital budget: Sufficient funding is the precondition for high-quality and well-integrated information system and the necessary competent IT staff.\\
    Quelle|Anz. Antw. &  Book  | 3 \\
    \midrule
    Frage & Information systems managers can be partly compared to architects.
    Read the following statement and discuss similarities and differences between information system architects and building architects [8]:

    ``We are architects.
    [...] We have designed numerous buildings, used by many people.
    [...] We know what users want.
    We know their complaints: buildings that get in the way of the things they want to do.
    [...] We also know the users' joy of relaxing, working, learning, buying, manufacturing, and worshipping in buildings which were designed with love and care as well as function in mind.
    [...] We are committed to the belief that buildings can help people to do their jobs or may impede them and that good buildings bring joy as well as efficiency.'' \\
    Umformuliert & Information systems managers can be partly compared to architects.
    Read the following statement and discuss similarities and differences between information system architects and building architects:

    ``We are architects.
    [...] We have designed numerous buildings, used by many people.
    [...] We know what users want.
    We know their complaints: buildings that get in the way of the things they want to do.
    [...] We also know the users' joy of relaxing, working, learning, buying, manufacturing, and worshipping in buildings which were designed with love and care as well as function in mind.
    [...] We are committed to the belief that buildings can help people to do their jobs or may impede them and that good buildings bring joy as well as efficiency.'' \\
    Antwort & Health information managers can indeed be compared with architects.
    Health information managers design information systems that are used by many different user groups.
    Health information managers regularly monitor the quality of information systems to obtain feedback and to improve the information system.
    Health information managers understand that information systems support many different functions for many different user groups within health care facilities.
    Health information managers make sure that the application systems are user-friendly and support working processes in an efficient way.
    Health information managers understand that an information system serves the overall goal of a health care facility and ultimately serves the need of the patients.\\
    Quelle|Anz. Antw. &  Book  | 6 \\
    \midrule
    Frage & A clinical researcher at Ploetzberg Hospital has won a grant to set up a register for patients who have received a knee endoprosthesis.
    Disease registers are research databases for collecting data about a specific disease, aiming for full coverage of the respective patient collective.
    The aim of a knee endoprosthesis registry is to collect longitudinal data to find out which type of endoprosthesis works best over time.
    The researcher wants to integrate data from patient-reported outcome questionnaires, findings from inpatient or outpatient visits at the hospital, and results from laboratory examinations.
    Which entity types need to be integrated and from which application components do they come? Devise a plan how you would set up a sustainable research architecture, i.e., an architecture that also could be used in other research settings and for different disease or research entities, considering Sect. 6.6. \\
    Umformuliert & A clinical researcher at Ploetzberg Hospital has won a grant to set up a register for patients who have received a knee endoprosthesis.
    Disease registers are research databases for collecting data about a specific disease, aiming for full coverage of the respective patient collective.
    The aim of a knee endoprosthesis registry is to collect longitudinal data to find out which type of endoprosthesis works best over time.
    The researcher wants to integrate data from patient-reported outcome questionnaires, findings from inpatient or outpatient visits at the hospital, and results from laboratory examinations.
    Which entity types need to be integrated and from which application components do they come? Devise a plan how you would set up a sustainable research architecture, i.e., an architecture that also could be used in other research settings and for different disease or research entities, considering Sect. 6.6. \\
    Antwort & The following entity types have to be integrated: patient, person, diagnosis, finding, health record, medical procedure, patient record, self-gathered symptoms, material, medical device, classification, nomenclature.

    Application components to be integrated depend on local settings and implementation but will likely include: patient administration system, MDMS, LIS, OMS, PDMS, and self-diagnosis systems (e.g., an app for collecting patient-reported outcome data) or patient portals.

    A research architecture for setting up multiple registries might include a DWS for research that is fed via ETL processes from the above-mentioned application components and can be tapped for data in different use cases or research scenarios.
    Finally, an open platform architecture would enable reuse of patient data in various research contexts.\\
    Quelle|Anz. Antw. &  Book  | 3 \\
    \midrule
    Frage & Was ist an dieser Projektbeschreibung falsch?
                - Problem: Auf dem Software-Markt existiert eine Vielzahl unterschiedlicher Informationssysteme, welche im Bereich der Krankenhaushygiene eingesetzt werden können.
    Der Funktionsumfang und der Einsatzbereich der Informationssysteme ist sehr vielfältig.
                Ziel: Ziel ist eine Systematik zur Erfassung von Informationssystemen und eine systematische Auflistung aller am Markt verfügbarer Informationssysteme im Bereich der Krankenhaushygiene.\\
    Umformuliert & What is wrong with this project description?
    Problem: There are many different information systems on the software market which can be used in the field of hospital hygiene.
    The range of functions and the field of application of the information systems is very diverse.
    Goal: The goal is to create a system for recording information systems and a systematic listing of all information systems available on the market in the field of hospital hygiene.\\
    Antwort & Information Systems are not equal to Software products.
    The underlying definition here is wrong and does not include human actors as an essential part.\\
    Quelle|Anz. Antw. &  A\_2021  | 1 \\
    \midrule
    Frage & Nennen Sie ein Beispiel für fehlende Datenintegration in der Pflege \\
    Umformuliert & Give an example of lack of data integration in care \\
    Antwort & Patient Admission, Daily Checks and Physical Documentation lead to redundant data management.\\
    Quelle|Anz. Antw. &  A\_2021  | 1 \\
    \bottomrule
    \caption*{Evaluierungsdatensatz (Transferfragen)}\label{tab:evaldata-transfer}
\end{longtable}
\end{landscape} \par}