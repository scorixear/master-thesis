Frage & Was ist der Unterschied zwischen dem PVS und dem PDMS? \\
Umformuliert & What is the difference between the PMS and the PDMS? \\
Antwort & PMS:
- primary focus on helping health care provides streamline their day-to-day tasks

PDMS:
- primary focus on capturing mainaining comprehensive and accurate patient medical data. \\
Quelle & IS\_2022\_09\_27 \\
Anz. Antw. & 2 \\
\midrule
Frage & Look at the entity type "patient" that is interpreted and updated by various functions. Which functions update the patient information, which functions interpret it? \\
Umformuliert & Look at the entity type "patient" that is interpreted and updated by various functions to be performed by health care professionals and other staff in health care facilities. Which functions update the patient information, which functions interpret it? \\
Antwort & The entity type "patient" is updated by the function "patient admission." All other functions that are related to patient care interpret it. \\
Quelle & Book \\
Anz. Antw. & 2 \\
\midrule
Frage & Look at the 3LGM\^2 example in Sect. 2.15.
Use this example to explain the meaning of the following elements: functions, entity types, application systems, non-computer-based application components, physical data processing system, and inter-layer relationships. \\
Umformuliert & Use the provided example to explain the meaning of the following elements: functions, entity types, application systems, non-computer-based application components, physical data processing system, and inter-layer relationships. \\
Kontext & Example: Four subfunctions of patient admission (appointment scheduling, patient identification and checking for readmitted patients, administrative admission, and visitor and information services) are supported by the patient administration system, which is a part of the ERPS.
Medical admission and nursing admission are supported by the MDMS.
Obtaining consent for processing of patient-related data is supported by the non-computer-based application component for patient data privacy forms.
This application component is based on paper forms which are scanned by a clerk (see physical tool layer) and then stored in the MDMS.

The patient administration system, which is the master application system (Sect. 3.9.1) for the entity type "patient," sends the administrative patient data as a message to the MDMS.
The MDMS can thus store this information about the entity type "patient" in its own database; administrative patient data that is needed to support medical admission and nursing admission as functions therefore do not have to be reentered in the MDMS.
The entity type "patient" is both stored in the database systems of the ERPS and the MDMS what is represented by dashed lines between the domain layer and the logical tool layer.

Both the patient administration system and the MDMS are run on servers at a virtualized server farm (see relationships between logical and physical tool layer). The application systems can be accessed by different end devices (patient terminal, PC, tablet PC).

It therefore simplifies some aspects which might be relevant in other contexts.
Another visualization of relationships between 3LGM\^2 model elements is the matrix view.
The patient administration system supports three different functions, the MDMS supports two functions, and one function is supported by the paper-based patient data privacy form system.
The matrix view also helps to identify incomplete parts of models.
We can see that there are no functions modeled that are supported by the financial accounting system, the human resources management system, and the material management system, which are parts of the ERPS.

The matrix view is an alternative representation of configuration lines between functions at the domain layer and application components at the logical tool layer. Matrix views are also available for visualizing relations between other pairs of connected 3LGM\^2 classes. \\
Antwort & Administrative admission is an enterprise function that is supported by the patient administration system.
One entity type that is used and updated by this function is "patient." The paper-based patient data privacy form system is an example of a non-computer-based application component.
The virtualized server farm is an example of a physical tool.
The inter-layer relationships of this example show which functions are supported by which application system and which physical data processing system the application systems are installed on. \\
Quelle & Book \\
Anz. Antw. & 6 \\
\midrule
Frage & Look at the 3LGM\^2 sample model in Sect. 2.15 and try to answer the following questions.
(a) Find examples of specialization or decomposition at the domain layer.
(b) What is the meaning of the arrows pointing from patient identification to "patient" and from "patient" to medical admission ?
(c) What entity type that is stored in the paper-based patient data privacy form system should be added at the domain layer?
(d) Why is the function patient admission not connected with any application system?
(e) Which physical data processing systems are needed for the function "obtaining patient consent for the processing of data"? \\
Umformuliert & Answer the following questions under the given example context.
(a) Find examples of specialization or decomposition at the domain layer.
(b) What is the meaning of the arrows pointing from patient identification to "patient" and from "patient" to medical admission ?
(c) What entity type that is stored in the paper-based patient data privacy form system should be added at the domain layer?
(d) Why is the function patient admission not connected with any application system?
(e) Which physical data processing systems are needed for the function "obtaining patient consent for the processing of data"? \\
Kontext & Four subfunctions of patient admission (appointment scheduling, patient identification and checking for readmitted patients, administrative admission, and visitor and information services) are supported by the patient administration system, which is a part of the ERPS.
Medical admission and nursing admission are supported by the MDMS.
Obtaining consent for processing of patient-related data is supported by the non-computer-based application component for patient data privacy forms.
This application component is based on paper forms which are scanned by a clerk (see physical tool layer) and then stored in the MDMS.

The patient administration system, which is the master application system (Sect. 3.9.1) for the entity type "patient," sends the administrative patient data as a message to the MDMS.
The MDMS can thus store this information about the entity type "patient" in its own database; administrative patient data that is needed to support medical admission and nursing admission as functions therefore do not have to be reentered in the MDMS.
The entity type "patient" is both stored in the database systems of the ERPS and the MDMS what is represented by dashed lines between the domain layer and the logical tool layer.

Both the patient administration system and the MDMS are run on servers at a virtualized server farm (see relationships between logical and physical tool layer). The application systems can be accessed by different end devices (patient terminal, PC, tablet PC).

It therefore simplifies some aspects which might be relevant in other contexts.
Another visualization of relationships between 3LGM\^2 model elements is the matrix view.
The patient administration system supports three different functions, the MDMS supports two functions, and one function is supported by the paper-based patient data privacy form system.
The matrix view also helps to identify incomplete parts of models.
We can see that there are no functions modeled that are supported by the financial accounting system, the human resources management system, and the material management system, which are parts of the ERPS.

The matrix view is an alternative representation of configuration lines between functions at the domain layer and application components at the logical tool layer. Matrix views are also available for visualizing relations between other pairs of connected 3LGM\^2 classes. \\
Antwort & (a) The function "patient admission" is decomposed into five subfunctions - patient admission is only complete if all the subfunctions are completed. The entity type "patient" is decomposed into four entity types - data regarding that entity type are only complete if data about all sub-entity types is complete. There are no examples of specialization at the domain layer of Fig. 2.11.

(b) The function "patient identification" updates the entity type "patient." The function "medical admission" uses the entity type "patient." This indicates that identifying patient data are updated or created during patient admission and then used for medical admission.
 
(c) The entity type "privacy statement" could be added at the domain layer. It would be updated by the function "obtaining consent for processing of patient data."
 
(d) Patient admission is decomposed into five subfunctions with each being linked to an application component by which it is supported. Therefore, it could lead to an ambiguous model if the superordinated function was linked with another application component. The corresponding modeling rule says that only the leaf functions in a function hierarchy should be linked to application components.
 
(e) The function "obtaining patient consent for the processing of data" is supported by the paper-based patient data privacy form system. For this, a paper record cabinet, a scanner, and a clerk handling these tools are the physical data processing systems needed for the function. \\
Quelle & Book \\
Anz. Antw. & 5 \\
\midrule
Frage & Imagine a hospital information system that comprises four application systems: a PAS, an MDMS, a RIS, and a PDMS.
The hospital is now considering the introduction of a communication server to improve data integration.
Discuss the short-term and long-term pros and cons of this decision.
Which syntactic and semantic standards could be used? \\
Umformuliert & Imagine a hospital information system that comprises four application systems: a PAS, an MDMS, a RIS, and a PDMS.
The hospital is now considering the introduction of a communication server to improve data integration.
Discuss the short-term and long-term pros and cons of this decision.
Which syntactic and semantic standards could be used? \\
Antwort & Short-term advantages: The communication server can handle the communication between all four application systems, including receiving, buffering, transforming, and multicasting of messages.
It can also be used for monitoring the communication traffic.
The communication server thus supports data integration in heterogeneous information system architectures.

Long-term advantages: In the resulting (ACn, CP1) architecture, new application components can easily be integrated, as only one communication interface to the communication server needs to be implemented.

Standards: For the exchange of administrative data, HL7 V2 or V3 could be used as syntactic or semantic standard.
For the exchange of clinical data, various communication standards can be chosen such as HL7 FHIR, DICOM for medical images, or HL7 CDA for clinical documents. \\
Quelle & Book \\
Anz. Antw. & 3 \\
\midrule
Frage & The following questions can be answered by reading the text and analyzing the 3LGM\^2 figures of the CityCare Example 3.11.
(a) The EHRS and the VNA in CityCare are not linked with any function they support. Which function of the domain layer may (partly) be supported by these application systems? Which functions (as introduced in Sect. 3.3) that are supported by these application systems could be added at the domain layer?
(b) In which database systems shown in the logical tool layer should the entity type "patient" be stored?
(c) The MPI should receive messages containing PINs (entity type "patient") from all patient administration systems. Why is there no communication link between the MPI and the patient administration system of Ernst Jokl Hospital?
(d) According to the matrix view, which functions are supported redundantly in CityCare? Discuss pros and cons of the functional redundancies in this scenario. What redundancies would you resolve and how?
(e) Which functions in which health care facility cannot be performed anymore if "Application Server 1 Ernst Jokl Hospital" fails? Suggest a change to the physical tool layer that would minimize the risk of missing function support in case a single application server fails.
(f) For the CityCare network, would it make sense to implement further profiles from IHE? Explain your decision. \\
Umformuliert & The following questions can be answered reading the provided text.
(a) The EHRS and the VNA in CityCare are not linked with any function they support. Which function of the domain layer may (partly) be supported by these application systems? Which functions (as introduced in Sect. 3.3) that are supported by these application systems could be added at the domain layer?
(b) In which database systems shown in the logical tool layer should the entity type "patient" be stored?
(c) The MPI should receive messages containing PINs (entity type "patient") from all patient administration systems. Why is there no communication link between the MPI and the patient administration system of Ernst Jokl Hospital?
(d) According to the matrix view, which functions are supported redundantly in CityCare? Discuss pros and cons of the functional redundancies in this scenario. What redundancies would you resolve and how?
(e) Which functions in which health care facility cannot be performed anymore if "Application Server 1 Ernst Jokl Hospital" fails? Suggest a change to the physical tool layer that would minimize the risk of missing function support in case a single application server fails.
(f) For the CityCare network, would it make sense to implement further profiles from IHE? Explain your decision. \\
Kontext & Four subfunctions of patient admission (appointment scheduling, patient identification and checking for readmitted patients, administrative admission, and visitor and information services) are supported by the patient administration system, which is a part of the ERPS.
Medical admission and nursing admission are supported by the MDMS.
Obtaining consent for processing of patient-related data is supported by the non-computer-based application component for patient data privacy forms.
This application component is based on paper forms which are scanned by a clerk (see physical tool layer) and then stored in the MDMS.

The patient administration system, which is the master application system (Sect. 3.9.1) for the entity type "patient," sends the administrative patient data as a message to the MDMS.
The MDMS can thus store this information about the entity type "patient" in its own database; administrative patient data that is needed to support medical admission and nursing admission as functions therefore do not have to be reentered in the MDMS.
The entity type "patient" is both stored in the database systems of the ERPS and the MDMS what is represented by dashed lines between the domain layer and the logical tool layer.

Both the patient administration system and the MDMS are run on servers at a virtualized server farm (see relationships between logical and physical tool layer). The application systems can be accessed by different end devices (patient terminal, PC, tablet PC).

It therefore simplifies some aspects which might be relevant in other contexts.
Another visualization of relationships between 3LGM\^2 model elements is the matrix view.
The patient administration system supports three different functions, the MDMS supports two functions, and one function is supported by the paper-based patient data privacy form system.
The matrix view also helps to identify incomplete parts of models.
We can see that there are no functions modeled that are supported by the financial accounting system, the human resources management system, and the material management system, which are parts of the ERPS.

The matrix view is an alternative representation of configuration lines between functions at the domain layer and application components at the logical tool layer. Matrix views are also available for visualizing relations between other pairs of connected 3LGM\^2 classes. \\
Antwort & (a) EHR systems as comprehensive application systems combine the functionalities of MDMS, NDMS, and CPOE systems. The EHRS of CityCare could therefore be used for medical admission, preparation of an order, or execution of diagnostic and therapeutic procedures. However, each of the three health care facilities in CityCare has its own MDMS. Therefore, the EHRS is probably mainly used for accessing findings from the other health care facilities, for example, during medical admission. For the VNA, no suitable function is modeled at the domain layer. At the domain layer, archiving of patient information could be added which is supported by the VNA and, to some extent, also by the EHRS.
(b) The entity type "patient" represents the persons who are the subject of health care. Information about a patient includes the PIN and other administrative data about the person. Each of the application systems supporting subfunctions of patient care and having an own database system stores the entity type "patient," for example, the patient administration system including the MPI, the MDMS, the EHRS, and the VNA.
(c) In Ernst Jokl Hospital, there is a star architecture at the logical tool layer, i.e., a communication server is used for the exchange of messages between application systems. The patient administration system of Ernst Jokl Hospital, where the PINs of Ernst Jokl Hospital are generated, sends this information in a message to the communication server. The communication server forwards the message to the MPI of the health care network. In the central MPI of the tHIS, the local patient identification numbers of the different health care facilities are linked to the unique transinstitutional patient identification number of CityCare.
(d) Administrative admission, appointment scheduling, medical admission, order entry, patient identification, and preparation of an order are each supported by at least three application systems in the scenario.
Pros (examples):
- Each of the health care facilities has a functioning information system that is independent from changes or system failures in the other health care facilities.
- The different patient administration systems and medical documentation and management systems may be better adapted to the local needs and grown structures in the single health care facilities than an application system that is used by all of them together.

Cons (examples):
- Three or more different application systems that support the same function cause higher costs and higher administrative effort.
- The effort for establishing integration and interoperability are higher in functional redundant architectures which have a high number of single application systems.

Resolving these redundancies (examples):
- One patient administration system that supports patient identification and administrative admission could be used in all health care facilities instead of three patient administration system and an MPI.
- The central EHRS could be used as MDMS, NDMS as well as CPOE system in each of the facilities and would replace the existing local application systems.

(e) According to the matrix view in Fig. 3.37, the MDMS of Ploetzberg Hospital is installed on application server 1 Ernst Jokl Hospital. Thus, if this application server 1 Ernst Jokl Hospital fails, the following functions cannot no longer be performed: appointment scheduling, medical admission, order entry, and preparation of an order (see matrix view in Fig. 3.35).

The application systems used in CityCare should be made available by server clusters with redundant servers. If one server in a server cluster fails, another server can take over its task. Thus, there is no interruption in function support.
 
(f) Yes, it makes sense to use further integration profiles from IHE. For example, IHE XDS could be used. The CityCare network could be established as an affinity domain with several actors that interact in a standardized way (process interoperability) to share document-level or even large binary patient data, such as findings, images, or radiology reports. These documents would be registered centrally in a document registry and could be retrieved by other systems. Depending on how the central EHRS is implemented, it could either take the role of a do \\
Quelle & Book \\
Anz. Antw. & 6 \\
\midrule
Frage & In Sect. 4.8.1, we presented the structure of the strategic information management plan of Ploetzberg Hospital.
Compare its structure to the general structure presented in Sect. 4.3.1.2, consisting of strategic goals, description of current state, assessment of current state, future state, and migration path.
Where can you find this general structure in Ploetzberg Hospital's plan? \\
Umformuliert & Given the structure of the strategic information management plan of Ploetzberg Hospital.
Compare its structure to the general structure, consisting of strategic goals, description of current state, assessment of current state, future state, and migration path.
Where can you find this general structure in Ploetzberg Hospital's plan? \\
Antwort & - Strategic goals of the health care facility (business goals) and of management of information systems: are visible in Chaps. 1 and 2 of Ploetzberg Hospital's plan.
- Description of the current state of the information system: are visible in Chap. 3 of Ploetzberg Hospital's plan.
- Assessment of the current state of the information system: are visible in Chap. 4 of Ploetzberg Hospital's plan.
- Future state of the information system: are visible in Chap. 5 of Ploetzberg Hospital's plan.
- Migration path from the current to the planned state: are visible in Chap. 6 of Ploetzberg Hospital's plan. \\
Quelle & Book \\
Anz. Antw. & 5 \\
\midrule
Frage & Look at the health information system's KPIs of Ploetzberg Hospital in Example 4.8.2. Try to figure out some of these numbers for a real hospital and compare both hospitals' KPIs in the form of a benchmarking report.
It may help to look at the strategic information management plan of this hospital or at its website. \\
Umformuliert & Look at the health information system's KPIs of Ploetzberg Hospital in the context. Try to figure out 10 of these numbers for a real hospital and compare both hospitals' KPIs in the form of a benchmarking report. \\
Kontext & The CIO of Ploetzberg Hospital annually reports to the hospital's management about the amount, quality, and costs of information processing of the Ploetzberg Hospital information system.
For this report, the CIO uses health information system KPIs that have been agreed on by a regional group of hospital CIOs (Table 4.3). Each year, the hospitals exchange and discuss their reports as part of a best practice benchmark with other hospitals - this comparison is not shown in the table.

Table 4.3: Extract from the Ploetzberg Hospital health information system's benchmarking report 2024.
KPI key performance indicator
- KPIs for the hospital
- Number of staff \textbar 5500
- Number of beds \textbar 1100
- Number of inpatient cases 40,000
- Mean duration of stay \textbar 8.1 days
- Hospital budget \textbar 800 million
- KPIs for health information system's costs
- Overall IT costs \textbar 20 million
- IT costs per inpatient case \textbar 500
- IT costs in relation to hospital budget \textbar 2.5\%
- KPIs for health information system's management
- Number of HIS staff \textbar 46
- Number of HIS users \textbar 4800
- Number of workstations \textbar 1350
- Number of mobile IT tools \textbar 2500
- HIS user per mobile IT tool \textbar 1.9
- Number of IT problem tickets \textbar 15,500
- Percentage of solved IT problem tickets \textbar 96\%
- Availability of the overall HIS systems \textbar 98.5\%
- Number of finalized strategic IT projects \textbar 13
- Percentage of successful IT projects \textbar 76\%
- KPIs for health information system's functionality
- Percentage of all documents available electronically \textbar 45\%
- Percentage of all diagnosis coded electronically \textbar 77\%
- Functionality index of patient administration system \textbar 52\%
- Functionality index of MDMS \textbar 87\%
- KPIs for health information system's architecture
- Number of computer-based application components \textbar 84
- Percentage of standard interfaces between applications \textbar 87\%
- Functional redundancy rate \textbar 0.44 \\
Antwort & - KPI \textbar Ploetzberg Hospital \textbar My hospital
- Number of HIS staff \textbar 46 \textbar 89
- Number of HIS users \textbar 4800 \textbar 9000
- Number of workstations \textbar 1350 \textbar 6200
- Number of mobile IT tools \textbar 2500 \textbar 2000
- HIS user per mobile IT tool \textbar 1.9 \textbar 4.5
- Number of IT problem tickets \textbar 15,500 \textbar 36,250
- Percentage of solved IT problem tickets \textbar 96\% \textbar 92\%
- Availability of the overall HIS systems \textbar 98.5\% \textbar 96\%
- Number of finalized strategic IT projects \textbar 13 \textbar 10
- Percentage of successful IT projects \textbar 76\% \textbar 86\% \\
Quelle & Book \\
Anz. Antw. & 10 \\
\midrule
Frage & You are asked to organize regular (e.g., every half year) quantitative user feedback on the general user satisfaction with major clinical application components of your hospital as part of health information system's monitoring.
Which user groups would you consider? How could you gather user feedback regularly in an automatic way? Explain your choice. \\
Umformuliert & You are asked to organize regular (e.g., every half year) quantitative user feedback on the general user satisfaction with major clinical application components of your hospital as part of health information system's monitoring.
Which user groups would you consider? How could you gather user feedback regularly in an automatic way? Explain your choice. \\
Antwort & User groups: physicians, nurses, technical staff (e.g., lab, radiology), and management staff - these groups are typically large health information systems user groups. I would also organize regular survey of CIS key users, as they are experts in judging the quality of the information systems.

Organization of user feedback: (1) Health information system users are randomly invited to an automatic short and standardized survey that is displayed during CIS login.
(2) Every half year, I would organize sounding boards (a structured approach to obtain active feedback from stakeholders) with key users and with representatives from the larger user groups to discuss recent challenges with the CIS and opportunities for improvements. \\
Quelle & Book \\
Anz. Antw. & 2 \\
\midrule
Frage & Read the following case descriptions and discuss the integration problems using the types of integration presented in Sect. 5.3.4. Which negative effects for information logistics result from the identified integration problems?
1. A physician enters a medical diagnosis for a patient first in the medical documentation and management system (MDMS) and later, when ordering an X-ray, again in the CPOE system.
2. The position of the patient's name and the formatting of the patient's birthdate vary between the MDMS and the CPOE system.
3. When physicians shift from the MDMS to the CPOE system, they have to log in again and again search for the correct patient.
4. The CPOE system and the RIS use slightly different catalogs of available radiology examinations.
5. When physicians write the discharge letter for a patient in the MDMS, they also have to code the discharge diagnosis of a patient. For this coding, they have to use a feature that is only available in the patient administration system, so they have to shift to this application system.
6. While at the patient's bedside during their ward rounds, physicians have to use several application components at the same time, such as MDMS for retrieving recent findings, the CPOE system for ordering, and the PACS for retrieving images. \\
Umformuliert & Read the following case descriptions and discuss the integration problems using the types of integration. Which negative effects for information logistics result from the identified integration problems?
1. A physician enters a medical diagnosis for a patient first in the medical documentation and management system (MDMS) and later, when ordering an X-ray, again in the CPOE system.
2. The position of the patient's name and the formatting of the patient's birthdate vary between the MDMS and the CPOE system.
3. When physicians shift from the MDMS to the CPOE system, they have to log in again and again search for the correct patient.
4. The CPOE system and the RIS use slightly different catalogs of available radiology examinations.
5. When physicians write the discharge letter for a patient in the MDMS, they also have to code the discharge diagnosis of a patient. For this coding, they have to use a feature that is only available in the patient administration system, so they have to shift to this application system.
6. While at the patient's bedside during their ward rounds, physicians have to use several application components at the same time, such as MDMS for retrieving recent findings, the CPOE system for ordering, and the PACS for retrieving images. \\
Antwort & 1. A physician enters a medical diagnosis for a patient first in the MDMS and later, when ordering an X-ray, again in the CPOE system. -\textgreater No data integration, resulting in reentering of data, which is time-consuming and may lead to errors and inconsistencies in the data, which has the potential for patient harm.
 
2. The position of the patient's name and the formatting of the patient's birthdate vary between the MDMS and the CPOE system. -\textgreater No user interface integration, resulting in increased time effort when using various application components, increased time needed for user training, and increased risk in overlooking or misinterpreting important patient information, which has the potential for patient harm.
 
3. When physicians shift from the MDMS to the CPOE system, they have to log in again and again search for the correct patient. -\textgreater No context integration, leading to an increase in time needed to shift between application systems and an increased risk for selecting the wrong patient in the second application systems, which has the potential for patient harm.
 
4. The CPOE system and the RIS use slightly different catalogs of available radiology examinations. -\textgreater No semantic integration, making the exchange and reuse of patient information in both application systems challenging.
 
5. When physicians write the discharge letter for a patient in the MDMS, they also have to code the discharge diagnosis of a patient. For this coding, they have to use a feature that is only available in the patient administration system, so they have to shift to this application system. -\textgreater No feature integration, leading to increased time needed to shift to the patient administration system.
 
6. While being at the patient's bedside during their ward rounds, physicians have to use several application components at the same time, such as MDMS for retrieving recent findings, the CPOE system for ordering, and the PACS for retrieving images. -\textgreater No process integration; a process should be organized in a way that frequent change of application systems is avoided if possible. \\
Quelle & Book \\
Anz. Antw. & 6 \\
\midrule
Frage & Skizzieren Sie die Fachliche Ebene und die Logische Werkzeugebene des folgenden Szenarios als 3LGM\^2-Modell auf dem nächsten Blatt (9 Punkte).
   1. Die administrative Patientenaufnahme erfolgt mit dem Patientenverwaltungssystem. Die administrativen Patientendaten, repräsentiert durch den Objekttyp "Patient" werden vom Patientenverwaltungssystem über den Kommunikationsserver an das CPOE-System, das Medizinische Dokumentationssystem und das Laborinformationssystem gesendet. Auf Papier mitgebrachte Vorbefunde werden bei der administrativen Patientenaufnahme eingescannt und im Medizinischen Dokumentationssystem gespeichert. 
   2. Ergänzen Sie je eine Aufgabe des CPOE-Systems, des Medizinischen Dokumentationssystems und des Laborinformationssystems (inkl. Konfigurationslinien).
   3. Ergänzen Sie einen passenden Objekttyp sowie je zwei sinnvolle "bearbeitet"- und "nutzt"-Beziehungen zwischen Objekttypen und Aufgaben. \\
Umformuliert & Given the scenario from the context, list out the used application components and tasks they perform. Order these in the Domain Layer and Logical Tool Layer of the 3LGM\^2 model.
Add one task each of the CPOE system, the medical documentation system and the laboratory information system
Add a suitable object type as well as two meaningful "updates" and "uses" relationships between object types and tasks. \\
Kontext & The administrative patient admission takes place with the patient management system. The administrative patient data, represented by the object type "patient", are sent from the patient management system via the communication server to the CPOE system, the medical documentation system and the laboratory information system. Preliminary findings brought in on paper are scanned during administrative patient admission and stored in the medical documentation system. \\
Antwort & Domain Layer:
- Patient Admission

Logical Tool Layer:
- Patient Management System
- Communication Server
- CPOE System
- Medical Documentation System
- Laboratory Information System


CPOE System:
- Task: Creation/Administration of Medication plans

Medical Documentation System:
- Task: Administration/Storage of Patientanamnesis

Laboratory Information System:
- Task: Storage/Processing of Laboratory Requests

Object Type:
- Patient
  - Uses: Administration/Storage of Patientanamnesis, Creation/Administration of Medication plans
  - Processed: Patient Admission \\
Quelle & IS\_2022\_07\_18 \\
Anz. Antw. & 11 \\
\midrule
Frage & Nennen Sie 3 Dinge, die die Hausärztin Frau Meier in ihrer Praxis benötigt, um auf die in der Telematikinfrastruktur gespeicherten Daten des Patienten Herrn Schulz zuzugreifen? \\
Umformuliert & Name 3 things that the GP Ms Meier needs in her practice in order to access the data of the patient Mr Schulz stored in the telematics infrastructure. \\
Antwort & - electronic health professional card / institution card
- Access to telematric infrastructure (such as PC)
- Insured Person's permission to acces their data \\
Quelle & IS\_2022\_07\_18 \\
Anz. Antw. & 3 \\
\midrule
Frage & Ordnen Sie folgende Begriffe und Definitionen einander zu.
   1. Datenintegrität, Interoperabilität, Syntaktische Interoperabilität, Referentielle Integrität, Semantische Integration, Datenintegration
   2. Fähigkeit eines Anwendungssystems (AWS), Informationen mit anderen AWS auszutauschen und zu nutzen
   3. Zustand eines Informationssystems, in dem Daten, die einmal erfasst wurden, überall  verfügbar sind, wo sie benötigt werden
   4. Zustand eines Informationssystems, in dem interoperable Anwendungssysteme das gleiche Begriffssystem nutzen
   5. Fähigkeit eines Anwendungssystems, Nachrichten mit einer definierten Struktur auszutauschen
   6. Korrektheit der Daten
   7. Korrekte und eindeutige Zuordnung eines Objekts zu anderen Objekten \\
Umformuliert & Match the following terms and definitions.
Terms:
data integrity, interoperability, syntactic interoperability, referential integrity, semantic integration, data integration.
Definitions:
1. ability of an application system (AWS) to exchange and use information with other AWSs
2. state of an information system in which data, once captured, is available wherever it is needed
3. state of an information system in which interoperable application systems use the same conceptual system
4. ability of an application system to exchange messages with a defined structure
5. correctness of data
6. correct and unambiguous assignment of an object to other objects \\
Antwort & Data Integrity: 5. correctness of data
Interoperability: 1. ability of an application system (AWS) to exchange and use information with other AWSs
Syntactic Interoperability: 4. ability of an application system to exchange messages with a defined structure
Referential Integrity: 6. correct and unambiguous assignment of an object to other objects
Semantic Integration: 3. state of an information system in which interoperable application systems use the same conceptual system
Data Integration: 2. state of an information system in which data, once captured, is available wherever it is needed \\
Quelle & IS\_2022\_07\_18 \\
Anz. Antw. & 6 \\
\midrule
Frage & Erklären Sie für einen Interoperabilitätsstandard im Gesundheitswesen dessen Einsatzgebiet, Funktionsprinzip und die unterstützten Interoperabilitätsarten. \\
Umformuliert & For a healthcare interoperability standard, explain its area of use, operating principle and the types of interoperability supported. \\
Antwort & Health Level 7 Version 2:
Area of Use:
- Communication between Application Systems

operating principle:
- Message based
- Event driven

Types of Interoperability:
- Syntactic Interoperability
- Semantic Interoperability \\
Quelle & IS\_2022\_07\_18 \\
Anz. Antw. & 4 \\
\midrule
Frage & Vergleichen Sie HL7 V2 und HL7 FHIR. Erläutern Sie mindestens drei Unterschiede. \\
Umformuliert & Compare HL7 V2 and HL7 FHIR. Explain at least three differences \\
Antwort & - both used to exchange health information between application systems
- Syntax HL7 v2: proprietay, ASCII-text
- Syntax HL7 FHIR: XML / JSON  = easier to implement
- Structure HL7 v2: fixed, hierarchical, fixed amount of segments and codes
- Structure HL7 FHIR: flexible, modular, resource/element based \\
Quelle & IS\_2022\_07\_18 \\
Anz. Antw. & 3 \\
\midrule
Frage & Skizzieren Sie die Fachliche Ebene und die Logische Werkzeugebene des folgenden Szenarios als 3LGM\^2-Modell auf dem nächsten Blatt.
   1. Die administrative Patientenaufnahme erfolgt mit dem Patientenverwaltungssystem. Die administrativen Patientendaten, repräsentiert durch den Objekttyp "Patient" werden vom Patientenverwaltungssystem über den Kommunikationsserver an das CPOE-System, das Medizinische Dokumentationssystem und das Radiologieinformationssystem gesendet. Auf Papier mitgebrachte Vorbefunde werden bei der administrativen Patientenaufnahme eingescannt und im Medizinischen Dokumentationssystem gespeichert.
   2. Ergänzen Sie je eine Aufgabe des CPOE-Systems, des Medizinischen Dokumentationssystems und des Radiologieinformationssystems (inkl. Konfigurationslinien).
   3. Ergänzen Sie einen passenden Objekttyp sowie je zwei sinnvolle "bearbeitet"- und "nutzt"-Beziehungen zwischen Objekttypen und Aufgaben. \\
Umformuliert & Describe the functional level and the logical tool level of the following scenario as a 3LGM\^2 model.
1. The administrative patient admission takes place with the patient management system. The administrative patient data, represented by the object type "patient", are sent from the patient management system via the communication server to the CPOE system, the medical documentation system and the radiology information system. Preliminary findings brought in on paper are scanned during administrative patient admission and stored in the medical documentation system.
2. Add one task each of the CPOE system, the medical documentation system and the radiology information system (incl. configuration lines).
3. Add a suitable object type as well as two meaningful "updates" and "uses" relationships between object types and tasks. \\
Kontext & The administrative patient admission takes place with the patient management system. The administrative patient data, represented by the object type "patient", are sent from the patient management system via the communication server to the CPOE system, the medical documentation system and the radiology information system. Preliminary findings brought in on paper are scanned during administrative patient admission and stored in the medical documentation system. \\
Antwort & Domain Layer:
- Administrative Patient Admission

Logical Tool Layer:
- Patient Management System
- Communication Server
- CPOE System
- Medical Documentation System
- Radiologoy Information System


CPOE System:
- Task: Creation/Administration of Medication plans

Medical Documentation System:
- Task: Administration/Storage of Patientanamnesis

Radiology Information System:
- Task: Archive/Processing of Radiological Images

Object Type:
- Medication Plan
  - Uses: Administration/Storage of Patientanamnesis
  - Processed: Creation/Administration of Medication plans \\
Quelle & IS\_2022\_09\_27 \\
Anz. Antw. & 11 \\
\midrule
Frage & Wer stellt die "elektronische Patientenakte (ePA)" nach § 341 SGB V zur Verfügung? Wie wird der Zugriff auf die enthaltenen medizinischen Daten geregelt? \\
Umformuliert & Who provides the "electronic patient file (ePA)" according to § 341 SGB V? How is access to the medical data it contains regulated? \\
Antwort & - provided and managed by the health insurance companies
- Technical and organizational measures to ensure only authorized access
- Access rights, role-based access control, documentation of access \\
Quelle & IS\_2022\_09\_27 \\
Anz. Antw. & 2 \\
\midrule
Frage & Was ist IHE und welchen Nutzen hat es? Erläutern Sie den Zusammenhang zwischen IHE und Interoperabilitätsstandards. \\
Umformuliert & What is IHE and what are its benefits? Explain the relationship between IHE and interoperability standards. \\
Antwort & IHE stands for Integrating the Healthcare Enterprise. It is an initiative by healthcare professionals and industry to improve the way computer systems in healthcare share information. IHE promotes the coordinated use of established standards such as DICOM and HL7 to address specific clinical needs in support of optimal patient care \\
Quelle & IS\_2022\_09\_27 \\
Anz. Antw. & 2 \\
\midrule
Frage & Was wird standardisiert? \\
Umformuliert & What are common standards used in health information systems? \\
Antwort & HL7v2
CDA
HL7 FHIR
DICOM
ISO/IEEE 11073
CCOW (Clinical Context Object Workgroup)
EDIFACT (Electronic Data Interchange for Administration, Commerce and Transport)
SNOMED, LOINC
openEHR
CDISC (Clinical Data Interchange Standards Consortium) \\
Quelle & IS\_2022\_09\_27 \\
Anz. Antw. & 11 \\
\midrule
Frage & Was sind Daten, Informationen, Wissen? \\
Umformuliert & What are data, information, knowledge? \\
Antwort & Data are characters, discrete numbers, or continuous signals to be processed in information systems.
Information is a context-specific fact about entities such as events, things, persons, processes, ideas, or concepts. Information is represented by data.
Knowledge is general information about concepts in a certain (scientific or professional) domain (e.g., knowledge about diseases or therapeutic methods) at a certain time. \\
Quelle & A\_2021 \\
Anz. Antw. & 3 \\
\midrule
Frage & Was sind System, soziotechnisches System \\
Umformuliert & What are system and socio-technical system? \\
Antwort & A system is a set of persons, things, events, and their relationships forming an integrated whole.
If a (human-made) system consists of both human and technical components, it can be called a socio-technical system. \\
Quelle & A\_2021 \\
Anz. Antw. & 2 \\
\midrule
Frage & Wie geht 3LGM\^2 genauer? \\
Umformuliert & How dows 3LGM\^2 work? \\
Antwort & 3LGM\^2 is a three-layer graph-based metamodel for modeling (health) information systems.
It combinsed function, techincal and organisational aspects with certain aspects of data dn process metamodels.
It distinguishes between the following layers:
- Domain Layer (activities in a health care setting)
- Logical Tool Layer (application components)
- Physical Tool Layer (Physical data processing systems and their data transmission links) \\
Quelle & A\_2021 \\
Anz. Antw. & 3 \\
\midrule
Frage & Was ist eine datenverarbeitende Aufgabe (enterprise function)? Bitte nennen Sie Beispiele \\
Umformuliert & What is a data processing task (enterprise function)? Please give examples \\
Antwort & Enterprise functions mainly emphasize the contribution of activities to business goals.
Examples are Administrative Admission or Patient Care. \\
Quelle & A\_2021 \\
Anz. Antw. & 2 \\
\midrule
Frage & Bitte erläutern Sie die Aufgabe "Patientenaufnahme" \\
Umformuliert & please explain the task "patient admission" \\
Antwort & Patient admission updates and uses the entity type patient
Consists of refined subfunctions such as Nursing Admission.
Example Activity of Patient Admission is "Physician admits Patient \\
Quelle & A\_2021 \\
Anz. Antw. & 2 \\
\midrule
Frage & Nennen Sie Beispiele für Objekttypen in einem Krankenhausinformationssystem \\
Umformuliert & Give examples of object types in a hospital information system \\
Antwort & Patient
Doctor
Medicine Plan
Medical Record \\
Quelle & A\_2021 \\
Anz. Antw. & 3 \\
\midrule
Frage & Bitte skizzieren Sie eine typische LWE \\
Umformuliert & Describe the logical tool layer. \\
Antwort & Consists of Application Systems, or in broader sense, application component as the center of interest.
Represent certain application software products on a certain computer system.

Connected via Message Oriented Communicaton using commmunication interfaces.
Or Service-Oriented Communication by providing features to other application systems. \\
Quelle & A\_2021 \\
Anz. Antw. & 4 \\
\midrule
Frage & Bitte erläutern Sie die Begriffe Integration, Interoperabilität, Integrität. \\
Umformuliert & Please explain the terms integration, interoperability, integrity. \\
Antwort & Integration means that the application systems are put together in such a way that the resulting information system - as opposed to its parts - displays a new quality.
Interoperability in general is the ability of two application systems to exchange information with each other and to use the information that has been exchanged.
Data integrity means that data are consistent, that object identity is maintained, and that relationships between entities are correct (referential integrity) \\
Quelle & A\_2021 \\
Anz. Antw. & 3 \\
\midrule
Frage & Welche Arten von Integrität haben wir diskutiert? \\
Umformuliert & What types of integrity exist in the context of health information systems? \\
Antwort & Object Identity, Referential Integrity, Consistency \\
Quelle & A\_2021 \\
Anz. Antw. & 3 \\
\midrule
Frage & Welche Typen der Integration haben wir diskutiert? \\
Umformuliert & What types of integration exist in the context of health information systems? \\
Antwort & Data-Integration, Semantic Integration, User-Interface Integration, Context Integration, Functional Integration, Process Integration \\
Quelle & A\_2021 \\
Anz. Antw. & 6 \\
\midrule
Frage & Nennen Sie zwei Kommunikationsstandards für KIS? \\
Umformuliert & Name two communication standards for HIS? \\
Antwort & HL7, DICOM \\
Quelle & A\_2021 \\
Anz. Antw. & 2 \\
\midrule
Frage & Wie hängen 3LGM\^2, IHE zusammen \\
Umformuliert & How are 3LGM\^2 and IHE related? \\
Antwort & 3LGM\^2 is used to model Health Information Systems, while IHE describes Standards for Health Information Systems. \\
Quelle & A\_2021 \\
Anz. Antw. & 2 \\
\midrule
Frage & Wie hängen HL7, DICOM zusammen \\
Umformuliert & How are HL7, DICOM related? \\
Antwort & Both are Communication Standards commonly used in a Health Information System \\
Quelle & A\_2021 \\
Anz. Antw. & 2 \\
\midrule
Frage & Wie hängen IHE - Objektidentität - tHIS zusammen? \\
Umformuliert & How are IHE - object identity - tHIS related? \\
Antwort & IHE provides standard for Health Information Systems to improve communication between HIS. tHIS describes a System that includes multiple HIS. \\
Quelle & A\_2021 \\
Anz. Antw. & 2 \\
\midrule
Frage & Welche Bedeutung hat die Softwareentwicklung in einem Krankenhaus? \\
Umformuliert & What is the importance of software development in a hospital? \\
Antwort & While software development of fully integrated application components are prohibited by law due to missing certificates, adjusting software to the Needs of each Hospital results in better communication, usage and therefore results during operation. \\
Quelle & A\_2021 \\
Anz. Antw. & 2 \\
\midrule
Frage & Was ist Transaktionsmanagement? Wie wird das in KIS organisiert? \\
Umformuliert & What is transaction management? How is it organised in HIS? \\
Antwort & Trancations management ensures ACID properties during communication. This is managed by the Communication Server. Each data entity is administered by one application component and changes are broadcast to all application components that use it. \\
Quelle & A\_2021 \\
Anz. Antw. & 2 \\
\midrule
Frage & Nennen Sie drei Anwendungssysteme und die jeweils unterstützten Aufgaben! \\
Umformuliert & Name three application systems and the tasks each supports! \\
Antwort & Patient Administration System, Patient Data Management System, Laboratory Information System, Radiology Information System, etc. \\
Quelle & A\_2021 \\
Anz. Antw. & 3 \\
\midrule
Frage & Wie sorgen Sie für Ausfallsicherheit in einem KIS? \\
Umformuliert & How do you ensure fail-safety in a HIS? \\
Antwort & Redundancy across. Multiple Hardware Components that mirror each other. Separated Infrastructure for energy. \\
Quelle & A\_2021 \\
Anz. Antw. & 3 \\
\midrule
Frage & Wie kann man Krankenhausinformationssysteme vergleichen? \\
Umformuliert & How can hospital information systems be compared? \\
Antwort & (Referenzmodelle, Taxonomy) \\
Quelle & A\_2021 \\
Anz. Antw. & 2 \\
\midrule