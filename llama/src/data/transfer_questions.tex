Look at the functions presented in Sect. 3.3.2. Now imagine a small hospital (e.g., 350 beds) and a large university medical center (e.g., 1500 beds). What are the differences between these hospitals with regard to their functions? Explain your answer. & Book & 2 & Imagine a small hospital (e.g., 350 beds) and a large university medical center (e.g., 1500 beds). What are the differences between these hospitals with regard to their functions to be performed by health care professionals and other staff in health care facilities? Explain your answer. & - & A typical hospital needs all functions to function as expected. The functions to be performed by health care professionals are mostly similar in all health care facilities, independent of their size. Only some functions may differ. For example, not all health care facilities are involved in clinical research, thus their information will not need to support the function research and education. \\
Auf einer Hersteller-Webseite heißt es: „i.s.h.med ist das einzige vollständig in SAP for Healthcare integrierte Krankenhausinformationssystem“. Welches Begriffsverständnis liegt hier zugrunde? & IS_2022_07_18 & 1 & On a manufacturer's website it says: "i.s.h.med is the only hospital information system fully integrated in SAP for Healthcare". What is the underlying understanding of this term? & - & This manufacturer sees the hospital information system as software product, while in reality a HIS includes Software, Hardware and Actors. \\
Wann ist ein Modell gut? & IS_2022_09_27 & 3 & When is a model good in the context of health information systems? & - & A good models should be able to help understand and predict the behavior of the system or process. It should also be able to help design and evaluate health information systems. A reference architecture can be used to support the design of a proper HIS architecture that meets the various stakeholder concerns of HISs. This architecture should be able to show the HIS from a different angle, suitable for various stakeholders. \\
Was hat das Krankenhausinformationssystem mit einer ganzheitlichen Sicht auf den Patienten zu tun? & IS_2022_09_27 & 2 & What does the hospital information system have to do with a holistic view of the patient? & - & A hospital information system must provide the right information about patients in the right place to the right people at the right time. Ideally, this means that information about the patient is also taken into account holistically across departmental and case boundaries, e.g. for the optimal treatment of multimorbid patients and for the avoidance of side effects that can occur due to known allergies and multimedication and, in the worst case, lead to death. Additional costs, effort and patient treatment due to superfluous multiple examinations are avoided. \\
Welche Kommunikation erfolgt unmittelbar nach der Aufnahme des Patienten Alfred Winter? & IS_2022_09_27 & 3 & What communication takes place immediately after the admission of the patient Alfred Winter? & - & Admission is divided into administrative, mecial and nursing admission. Via the (possibly existing) communication server the changed/newly added infiroations are then broadcasted to related application components such as the patient managements system. \\
Look at the functions listed in Sect. 3.3.2. Look at the relationships between the functions and the different health care professional groups (physicians, nurses, administrative staff, others) working in hospitals and medical offices. Select one health care professional group and describe which functions are most important for this group. & Book & 3 & Look at the relationships between functions to be performed by health care professionals and other staff in health care facilities and the different health care professional groups (physicians, nurses, administrative staff, others) working in hospitals and medical offices. Select one health care professional group and describe which functions are most important for this group. & - & Physicians: Important functions are medical admission, decision-making and patient information, planning and organization of patient treatment, order entry, execution of diagnostic and therapeutic procedures, coding of diagnoses and procedures, and medical discharge and medical discharge summary writing. Nurses: Important functions are nursing admission, decision-making and patient information, planning and organization of patient treatment, order entry, execution of nursing procedures, and nursing discharge and nursing discharge summary writing. Administrative staff: Important functions are patient identification, administrative admission, and administrative discharge and billing. \\
Read Examples 5.5.1 and determine which methods for collecting data (as described in Sects. 5.4.3 and 5.4.4) have been used. & Book & 1 & Given the Example in the Context, determine which methods for collecting data have been used. & ### 5.5.1 Unintended Effects of a Computerized Physician Order Entry Nearly Hard-Stop Alert
The introduction of application systems may have unintended effects.
The careful evaluation of impact and unintended effects of application systems is thus an important task of management of information systems.
We will now have a look at an example of an evaluation study that showed some unintended effects of CPOE systems.
Table 5.1 presents the abstract of an RCT on automatic alerts in a CPOE system.
The authors analyzed whether the so-called hard-stop alert can reduce unwanted drug-drug interactions.
Such a “hard-stop alerts” appears on the screen to alert the physician about potential problems associated with a particular prescription and blocks the clinician's order from further execution to avert potentially serious reactions.

Table 5.1: Abstract from “Unintended Effects of a Computerized Physician Order Entry Nearly Hard-Stop Alert” [5]
- Background: The effectiveness of CPOE systems has been modest, largely because clinicians frequently override electronic alerts
- Methods: To evaluate the effectiveness of a nearly “hard-stop” CPOE system prescribing alert intended to reduce concomitant orders for warfarin and trimethoprim-sulfamethoxazole, a randomized clinical trial was conducted at two academic medical centers in Philadelphia, Pennsylvania. A total of 1981 clinicians were assigned to either an intervention group receiving a nearly hard-stop alert or a control group receiving the standard practice. The study duration was August 9, 2006, through February 13, 2007
- Results: The proportion of desired responses (i.e., not reordering the alert-triggering drug within 10 min of firing) was 57.2\% (111 of 194 hard-stop alerts) in the intervention group and 13.5\% (20 of 148) in the control group (adjusted odds ratio, 0.12; 95\% confidence interval, 0.045-0.33). However, the study was terminated early because of four unintended consequences identified among patients in the intervention group: a delay of treatment with trimethoprim-sulfamethoxazole in two patients and a delay of treatment with warfarin in another two patients
- Conclusions: An electronic hard-stop alert as part of an inpatient CPOE system seemed to be extremely effective in changing prescribing habits. However, this intervention precipitated clinically important treatment delays in four patients who needed immediate drug therapy. These results illustrate the importance of formal evaluation and monitoring for unintended consequences of programmatic interventions intended to improve prescribing habits
 
The study was designed as a quantitative, explanatory field study that was conducted as an RCT.
The study found that these alerts can help to reduce the number of alert-triggering orders.
But it also found that the hard-stop alert led to clinically important treatment delays in four patients. & Study “Unintended Effects of a Computerized Physician Order Entry Nearly Hard-stop Alert”: The effectiveness of a nearly “hard-stop” alert was evaluated in a field study. The data was collected via analysis of the prescriptions in the CPOE systems. The overall data collection method is thus a quantitative observation of available data. \\
Read Examples 5.5.2 and determine which methods for collecting data (as described in Sects. 5.4.3 and 5.4.4) have been used. & Book & 1 & Given the Example in the Context, determine which methods for collecting data have been used. & ### 5.5.2 Clinical Decision Support for Worker Health: A Five-Site Qualitative Needs Assessment in Primary Care Setting
Besides evaluating the effect of an intervention, evaluation may also try to understand reasons for successful or unsuccessful implementation of an application system.
For these kinds of questions, qualitative studies are often chosen.
Table 5.2 presents the abstract of such a qualitative study.
The authors analyzed need, barriers, and facilitators for clinical decision support (CDS) in primary care.
The study was performed as a qualitative, exploratory field study.

Table 5.2: Abstract from “Clinical Decision Support for Worker Health: A Five-Site Qualitative Needs Assessment in Primary Care Settings.” [6]

- Background: Although patients who work and have related health issues are usually first seen in primary care, providers in these settings do not routinely ask questions about work. Guidelines to help manage such patients are rarely used in primary care. Electronic health record systems (EHRS) with worker health CDS tools have potential for assisting these practices
- Objective: This study aimed to identify the need for and barriers and facilitators related to implementation of CDS tools for the clinical management of working patients in a variety of primary care settings
- Methods: We used a qualitative design that included analysis of interview transcripts and observational field notes from 10 clinics in five organizations
- Results: We interviewed 83 providers, staff members, managers, informatics and IT experts, and leaders and spent 35 h observing. We identified eight themes in four categories related to CDS for worker health (operational issues, usefulness of proposed CDS, effort and time-related issues, and topic-specific issues). These categories were classified as facilitators or barriers to the use of the CDS tools. Facilitators related to operational issues include current technical feasibility and new work patterns associated with the coordinated care model. Facilitators concerning usefulness include users' need for awareness and evidence-based tools, appropriateness of the proposed CDS for their patients, and the benefits of population health data. Barriers that are effort-related include the additional time the proposed CDS might take as well as other pressing organizational priorities. Barriers that are topic-specific include sensitive issues related to health and work and the complexities of information about work
- Conclusion: We discovered several themes not previously described that can guide future CDS development: technical feasibility of the proposed CDS within a commercial electronic health record (EHR), the sensitive nature of some CDS content, and the need to assist the entire health care team in managing worker health

The authors found several factors that may hinder or foster the use of CDS in primary care.
The results of this multi-center study can now be used to implement CDS in commercial application software products for primary care. & Study “Clinical Decision Support for Worker Health: A Five-Site Qualitative Needs Assessment in Primary Care Setting”: data were collected via interviews and qualitative observations. \\
In which of the health care settings above will the function medical admission need to be supported? & Book & 4 & In which health care settings will the function medical admission need to be supported? & - & The function “medical admission” is relevant in several health care and research settings.
It comprises the provision of forms for documenting medical history, documenting diagnoses, and scanning documents from referring physician and other sources of information about the medical history.
It is obvious that this function needs to be supported in hospitals, nursing homes, ambulatory nursing organizations, and medical offices.
Yet it is often also necessary in research settings, for example, when a person is recruited for a clinical trial and their data are entered into an EDC system.
Furthermore, therapeutic offices need this function for documentation purposes, as do rehabilitation facilities and—to a limited extent—wellness or sports facilities.
For personal environments, medical admission also plays a role, especially in telecare situations or when prevention measures are conducted, respectively. \\
An welcher Stelle im Szenario aus Aufgabe 2 könnte der Dienst „KIM“ aus der Telematikinfrastruktur Abhilfe schaffen? & IS_2022_07_18 & 1 & At which point in the scenario from the context could the "KIM" service from the telematics infrastructure provide a remedy? & The administrative patient admission is done with the patient management system. The administrative patient data, represented by the object type "patient", are sent from the patient management system via the communication server to the CPOE system, the medical documentation system and the laboratory information system. Preliminary findings brought in on paper are scanned during administrative patient admission and stored in the medical documentation system. There also exists the CPOE-System, medical documentation system and the laboratory information system. & Preliminary findings brought in on paper replaced by the KIM service, which provides secure, digital transmission. \\
Was ist ein Krankenhausinformationssystem?
        2.1 Warum ist diese Definition wichtig? & IS_2022_09_27 & 1 & Why is the definition of a hospital information system important? & - & In most cases, human actors are not included in this definition while being essential for a working HIS. \\
Consider a recent health-related situation you were involed in. Which life situation does it correspond to and what was your role in this life situation? List some of the requirements you had in this role and in this life situation. & Book & 5 & Consider a recent health-related situation one could be involed in. Which life situation does it correspond to and what was your role in this life situation? List some of the requirements you had in this role and in this life situation. & - & My father was admitted to the hospital after suddenly showing symptoms of numbness in the left arm, confusion, and trouble seeing while at home. We called the ambulance, and after a short examination, the ambulance team took him to the nearest hospital for further diagnosis and treatment.My father was admitted to the hospital after suddenly showing symptoms of numbness in the left arm, confusion, and trouble seeing while at home. We called the ambulance, and after a short examination, the ambulance team took him to the nearest hospital for further diagnosis and treatment. This situation corresponds to an emergency life situation. I participated in this situation as a close relative. My urgent requirements were to know which hospital my father was taken to and to obtain more information on the suspected diagnosis (here: stroke) and the next steps of diagnosis and therapy. \\
Imagine that a physician is given the following information about his patient, Mr. Russo: “Diagnosis: hypertension. Last blood pressure measurement: 160/100 mmHg.” Use this example to discuss the difference between “data,” “information,” and “knowledge”! & Book & 3 & Imagine that a physician is given the following information about his patient, Mr. Russo: “Diagnosis: hypertension. Last blood pressure measurement: 160/100 mmHg.” Use this example to discuss the difference between “data,” “information,” and “knowledge”! & - & “160,” “100,” “hypertension,” and “blood pressure” represent data that cannot be interpreted without knowledge about the context.
The information is that Mr. Russo has been diagnosed with hypertension and that his last blood pressure is 160/100 mmHg.
The medical knowledge embedded in this example is that a blood pressure of 160/100 mmHg indicates hypertension that should be treated. \\
Consider the requirements of various stakeholders when it comes to health information systems supporting various life situations.
Can you imagine situations where the requirements of two stakeholder groups differ or even contradict each other? What does this imply when building health information systems? & Book & 2 & Consider the requirements of various stakeholders when it comes to health information systems supporting various life situations.
Can you imagine situations where the requirements of two stakeholder groups differ or even contradict each other? What does this imply when building health information systems? & - & While a patient is being treated for an acute disease, the requirements of the treating physicians and nurses as well as of the patient and relatives may differ.
For example, patient and relatives want to be kept informed of ongoing diagnostic outcomes (e.g., lab values) as soon as possible.
However, physicians and nurses may want to discuss the findings with the patient in person to avoid causing unnecessary confusion and stress in the patient.
Therefore, the health information system must be able to provide detailed information to physicians and nurses, but it must be able to only present confirmed information to the patient (e.g., via a patient portal). \\
Look up some information on the nervous system of the human body.
Then try to identify subsystems of the nervous system.
In the same way, can you also describe subsystems of the system “hospital”? & Book & 2 & Whe comparing the nervous system of the humand body to the system "hospital" the following subsystems can be identified and described:\n & - & The nervous system comprises two main categories of cells: neurons and glial cells.
Neurons communicate with each other via synapses and thus form their own subsystem.
Glial cells form another subsystem that provides support and nutrition to the neurons.

The hospital can be understood as a system comprising at least two subsystems: the subsystem where clinical care takes place and the subsystem where management takes place.
The clinical subsystem can again be split into several subsystems, such as inpatient area, outpatient area, and specialized diagnostic or therapeutic areas.
The inpatient area itself can be divided into various subsystems, each represented by one ward.
The way I define the subsystems of a hospital depends on the questions or intentions I have. \\
Imagine a situation in which a physician speaks with Mr. Russo at the patient's bedside.
The physician looks up Mr. Russo's recent blood pressure measurement and ongoing medication, decides to increase the level of one medication, and explains this to Mr. Russo.
Use this example to discuss the meaning of “information and knowledge logistics.” What in this example indicates the right information, the right place, the right people, the right form, and the right decision? What could happen if an information system does not support high-quality information and knowledge logistics? & Book & 2 & Imagine a situation in which a physician speaks with Mr. Russo at the patient's bedside.
The physician looks up Mr. Russo's recent blood pressure measurement and ongoing medication, decides to increase the level of one medication, and explains this to Mr. Russo.
Use this example to discuss the meaning of “information and knowledge logistics.” What in this example indicates the right information, the right place, the right people, the right form, and the right decision? What could happen if an information system does not support high-quality information and knowledge logistics? & - & The physician wants to have access to the right information (the most recent blood pressure) at the right time (when talking to Mr. Russo) at the right place (at the patient's bedside) in the right form (hopefully the blood pressure is provided in an easy-to-grasp, visual way) so that he can make the right decision (here: to decide on the level of a certain medication).
If the information system does not support this, the physician may obtain an incorrect or outdated blood pressure measurement, or he may misinterpret it, thereby coming to a decision that is suboptimal for the patient. \\
During a night shift, a nurse uses the patient administration system to conduct the administrative patient admission.
The nurse then uses the NMDS to plan nursing care.
Now consider the types of integration presented in Sect. 3.8 and discuss how this nurse would recognize a high (or low) level of data integration, semantic integration, user interface integration, context integration, feature integration, and process integration. & Book & 6 & During a night shift, a nurse uses the patient administration system to conduct the administrative patient admission.
The nurse then uses the NMDS to plan nursing care.
Now consider the types of integration and discuss how this nurse would recognize a high (or low) level of data integration, semantic integration, user interface integration, context integration, feature integration, and process integration. & - & Data integration would be considered high when the nurse documents patient administrative data only once in the patient administration system and then can use this data in the NMDS.

Semantic integration would be considered high when the nurse documents a nursing diagnosis using a standardized terminology (such as NANDA) and when this standardized diagnosis is then understood by the NMDS that may, for example, suggest a standard nursing care plan for this patient based on this diagnosis.

User interface integration would be considered high when the user interfaces of both application systems look sufficiently similar, which reduces the risk of data entry or data interpretation errors.
For example, in both application systems, the names of the patients are always displayed at the same place, the birthdates are presented in standardized form, and colors that are used to highlight important information are used in the same way.

Context integration would be considered high when the user context and the patient context is preserved when the nurse shifts from one application system to the other.
The nurse thus would not have to repeat user login or the selection of the patient in the second application system.

Feature integration would be considered high when only the patient administration system offers the needed administrative features (such as documentation of patient address). The nurse would be able to call up these features from within the NMDS.

Process integration would be considered high if both application systems work together in a highly integrated way so that the process of patient admission and nursing care planning from the point of view of the nurse is supported in an efficient way. \\
Imagine you are the CIO of a hospital in which almost no computer-based tools are used.
One of the hospital's goals is to support health care professionals in their daily tasks by offering up-to-date patient information at their workplace.
Which main goals for management of information systems could you define based on this information? Which functions should be prioritized to be supported by new application systems? What could a strategic project portfolio and a migration plan for the next 5 years look like? & Book & 5 & Imagine you are the CIO of a hospital in which almost no computer-based tools are used.
One of the hospital's goals is to support health care professionals in their daily tasks by offering up-to-date patient information at their workplace.
Which main goals for management of information systems could you define based on this information? Which functions should be prioritized to be supported by new application systems? What could a strategic project portfolio and a migration plan for the next 5 years look like? & - & Goals: efficient and high-quality information logistics to support patient care.

Functions: patient administration and all functions related to patient care (Sect. 3.3.2.1).
Project portfolio and migration plan:
- Year 1: Introduction of a patient administration system.
- Year 2: Introduction of a CIS, an LIS and an RIS.
- Year 3: Introduction of a DAS and a PACS.
- Year 4: Introduction of an OMS and of a PDMS.
- Year 5: Introduction of a DWS and of a patient portal.

Please note: This is a simplified solution. Other solutions may be valid, too. In case the different application systems are meant to come from different vendors, an integration technology such as a communication server needs to be implemented. \\
Imagine you are the CIO and have to select the three most relevant indicators for the quality of your information system at your hospital: Which would you select? You can look at the examples in Sect. 4.8.2 to get ideas.
Explain your choice. & Book & 3 & Imagine you are the CIO and have to select the three most relevant indicators for the quality of your information system at your hospital: Which would you select?
Explain your choice. & - & Several solutions are possible here. One possible solution:
1. HIS user per mobile IT tool: Efficient information logistics everywhere (e.g., at the patient's bedside) requires enough mobile IT tools.
2. Number of application systems: I would strive for an integrated information system and reduce the number of application systems in the long run in order to reduce integration problems.
3. HIS budget in relation to the overall hospital budget: Sufficient funding is the precondition for high-quality and well-integrated information system and the necessary competent IT staff. \\
Information systems managers can be partly compared to architects.
Read the following statement and discuss similarities and differences between information system architects and building architects [8]:

“We are architects.
[…] We have designed numerous buildings, used by many people.
[…] We know what users want.
We know their complaints: buildings that get in the way of the things they want to do.
[…] We also know the users' joy of relaxing, working, learning, buying, manufacturing, and worshipping in buildings which were designed with love and care as well as function in mind.
[…] We are committed to the belief that buildings can help people to do their jobs or may impede them and that good buildings bring joy as well as efficiency.” & Book & 6 & Information systems managers can be partly compared to architects.
Read the following statement and discuss similarities and differences between information system architects and building architects:

“We are architects.
[…] We have designed numerous buildings, used by many people.
[…] We know what users want.
We know their complaints: buildings that get in the way of the things they want to do.
[…] We also know the users' joy of relaxing, working, learning, buying, manufacturing, and worshipping in buildings which were designed with love and care as well as function in mind.
[…] We are committed to the belief that buildings can help people to do their jobs or may impede them and that good buildings bring joy as well as efficiency.” & - & Health information managers can indeed be compared with architects.
Health information managers design information systems that are used by many different user groups.
Health information managers regularly monitor the quality of information systems to obtain feedback and to improve the information system.
Health information managers understand that information systems support many different functions for many different user groups within health care facilities.
Health information managers make sure that the application systems are user-friendly and support working processes in an efficient way.
Health information managers understand that an information system serves the overall goal of a health care facility and ultimately serves the need of the patients. \\
A clinical researcher at Ploetzberg Hospital has won a grant to set up a register for patients who have received a knee endoprosthesis.
Disease registers are research databases for collecting data about a specific disease, aiming for full coverage of the respective patient collective.
The aim of a knee endoprosthesis registry is to collect longitudinal data to find out which type of endoprosthesis works best over time.
The researcher wants to integrate data from patient-reported outcome questionnaires, findings from inpatient or outpatient visits at the hospital, and results from laboratory examinations.
Which entity types need to be integrated and from which application components do they come? Devise a plan how you would set up a sustainable research architecture, i.e., an architecture that also could be used in other research settings and for different disease or research entities, considering Sect. 6.6. & Book & 3 & A clinical researcher at Ploetzberg Hospital has won a grant to set up a register for patients who have received a knee endoprosthesis.
Disease registers are research databases for collecting data about a specific disease, aiming for full coverage of the respective patient collective.
The aim of a knee endoprosthesis registry is to collect longitudinal data to find out which type of endoprosthesis works best over time.
The researcher wants to integrate data from patient-reported outcome questionnaires, findings from inpatient or outpatient visits at the hospital, and results from laboratory examinations.
Which entity types need to be integrated and from which application components do they come? Devise a plan how you would set up a sustainable research architecture, i.e., an architecture that also could be used in other research settings and for different disease or research entities, considering Sect. 6.6. & - & The following entity types have to be integrated: patient, person, diagnosis, finding, health record, medical procedure, patient record, self-gathered symptoms, material, medical device, classification, nomenclature.

Application components to be integrated depend on local settings and implementation but will likely include: patient administration system, MDMS, LIS, OMS, PDMS, and self-diagnosis systems (e.g., an app for collecting patient-reported outcome data) or patient portals.

A research architecture for setting up multiple registries might include a DWS for research that is fed via ETL processes from the above-mentioned application components and can be tapped for data in different use cases or research scenarios.
Finally, an open platform architecture would enable reuse of patient data in various research contexts. \\
Was ist an dieser Projektbeschreibung falsch?
            ◦ Problem: Auf dem Software-Markt existiert eine Vielzahl unterschiedlicher Informationssysteme, welche im Bereich der Krankenhaushygiene eingesetzt werden können. Der Funktionsumfang und der Einsatzbereich der Informationssysteme ist sehr vielfältig.
            Ziel: Ziel ist eine Systematik zur Erfassung von Informationssystemen und eine systematische Auflistung aller am Markt verfügbarer Informationssysteme im Bereich der Krankenhaushygiene. & A_2021 & 1 & What is wrong with this project description?
Problem: There are many different information systems on the software market which can be used in the field of hospital hygiene. The range of functions and the field of application of the information systems is very diverse.
Goal: The goal is to create a system for recording information systems and a systematic listing of all information systems available on the market in the field of hospital hygiene. & - & Information Systems are not equal to Software products. The underlaying definition here is wrong and does not include human actors as an essential part. \\
Nennen Sie ein Beispiel für fehlende Datenintegration in der Pflege & A_2021 & 1 & Give an example of lack of data integration in care & - & Patient Admission, Daily Checks and Physical Documentation lead to redundant data management. \\