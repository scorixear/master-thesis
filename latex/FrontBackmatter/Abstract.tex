%*******************************************************
% Abstract
%*******************************************************
\pdfbookmark[1]{Abstrakt}{Abstrakt}
\chapter*{Abstrakt}
\addcontentsline{toc}{chapter}{Abstrakt}
%Der Abstrakt soll kurz gehalten werden und dazu dienen, den Leser zum Lesen der kompletten Arbeit zu motivieren. Er ist optional.
Die Extraktion von Wissen aus Büchern ist essentiell und komplex.
Besonders in der Medizin ist ein einfacher und vollständiger Zugang zu Wissen wichtig.
In dieser Arbeit wurde ein vortrainiertes Sprachmodell verwendet, um den Inhalt des Buches \enquote{Health Information Systems} von \citet{bb} effizienter und einfacher zugänglich zu machen.
Während des Trainings wurde die Qualität des Modells zu verschiedenen Zeitpunkten evaluiert.
Dazu beantwortete das Modell Prüfungsfragen aus dem Buch und aus Modulen der Universität Leipzig, die inhaltlich auf dem Buch aufbauen.
Abschließend wurde ein Vergleich zwischen den Trainingszeitpunkten, dem nicht weiter trainierten Modell und dem \ac{sota} Modell GPT4 durchgeführt.
Die Ergebnisse zeigen eine deutliche Leistungssteigerung durch diesen Ansatz und bieten eine Grundlage für Erweiterungen.
\vfill
