%*******************************************************
% Summary
%*******************************************************
\pdfbookmark[1]{Zusammenfassung}{summary}
\chapter*{Zusammenfassung}
\addcontentsline{toc}{chapter}{Zusammenfassung}
%In der Zusammenfassung wird vor allem zusammenfassend dargestellt, welche Ergebnisse zu den formulierten Zielen erreicht wurden.
%Bei der Formulierung sollten wichtige Schlüsselbegriffe verwendet werden.
%Wurden Fragen formuliert, können auch diese hier beantwortet werden.
%So soll es möglich sein, dass ein Leser von der Arbeit lediglich die Einleitung und die Zusammenfassung lesen und doch die Ergebnisse der Arbeit erfassen kann.
%
%Checkliste:
%\begin{enumerate}
%\item Kurze, klare Darstellung zu Gegenstand, Fragestellung und Zielsetzung der Arbeit
%\item Angewandte Methodik
%\item Vorstellung der wesentlichen Ergebnisse und daraus resultierende Schlussfolgerungen.
%\end{enumerate}
%Für wissenschaftliche Abschlussarbeiten sind etwa 200 bis 250 Wörter ausreichend.

Die Wissensbeschaffung in der Medizininformatik ist ein komplexer und wichtiger Bestandteil von Forschung, Lehre und Praxis.
Umso wichtiger ist die Effizienz und Qualität der Wissensbeschaffung.
Informationssysteme stellen dabei eine grundlegende Architektur in diesem Bereich dar.
Das Buch \enquote{Health Information Systems} von \citet{bb} enthält eine umfassende Einführung in die Thematik der Informationssysteme in der Medizin.
Die Extraktion von Informationen und Wissen aus dem Buch erweist sich jedoch als schwierig.
In dieser Arbeit wurde ein vortrainiertes Sprachmodell verwendet, um den Inhalt des Buches effizienter und leichter zugänglich zu machen.
Ziel ist es, Wissen aus dem Buch zu extrahieren.
Dieses Ziel wurde anhand von Prüfungsfragen aus dem Buch und Modulen der Universität Leipzig, die inhaltlich auf dem Buch aufbauen, evaluiert.\\

Um diese Ziele zu erreichen, wurden verschiedene Sprachmodelle auf ihre Eignung evaluiert.
Die Wahl fiel auf das Sprachmodell Llama von MetaAI \citep{llama2}.
Dieses vortrainierte Sprachmodell wurde mit Hilfe des Continual Pretrainings weiter auf das Buch trainiert.
Während des Trainings wurde die Qualität des Modells zu verschiedenen Zeitpunkten evaluiert.
Diese Bewertung basierte auf den Kriterien Korrektheit, Erklärbarkeit, Fragenverständnis und Robustheit.
Abschließend wurde ein Vergleich zwischen den Trainingszeitpunkten, dem nicht weiter trainierten Modell und dem \ac{sota} Modell GPT4 durchgeführt.\\

Die Ergebnisse zeigen, dass Modelle mit Hilfe von Continual Pretraining auf ein spezifisches Buch trainiert werden können und in fast allen Kriterien bessere Ergebnisse erzielen als das nicht weiter trainierte Modell.
Aufgrund des Größenunterschieds zwischen GPT4 und dem hier verwendeten Modell Llama 2 7B wird die Leistung von GPT4 jedoch nicht erreicht.
Das nicht weiter trainierte Modell erreichte nur einen MakroF1-Wert von \num{0.1} und konnte durch das hier durchgeführte Training auf \num{0.33} gesteigert werden.
Dies zeigt das Potential dieser Methode.
Allerdings sind die Modelle noch nicht mit dem \ac{sota} Modell GPT4 vergleichbar, das einen MakroF1-Wert von \num{0.7} erreichte.
Die hier trainierten Modelle stellen keinen für die Praxis nutzbaren Zustand dar, lösen aber die gestellte Zielstellung \enquote{Machbarkeit der Beantwortung von Fragen mit Hilfe von Sprachmodellen}.\\

Um die Leistungsfähigkeit von Sprachmodellen zu erhöhen, können sowohl die Modellgröße als auch die Größe des Datensatzes erhöht werden.
Auch neue Entwicklungen im Bereich der Adapter (\cref{sec:adapter-training}) zeigen ressourceneffiziente Ansätze, um die Performanz von Sprachmodellen zu steigern.
Insbesondere der Einsatz von Retrieval-Augmented Generation bieten vielversprechende Erweiterungsmöglichkeiten.\\

Zusammenfassend kann gesagt werden, dass die Leistung von Sprachmodellen für die Beantwortung von Klausurfragen durch Continual Pretraining gesteigert werden kann, auch wenn die Endleistung noch keinen praxistauglichen Zustand erreicht.\\

\vfill
