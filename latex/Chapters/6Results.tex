%*****************************************
\chapter{Ergebnisse}\label{ch:results}
%*****************************************

- Ergebnisse der einzelnen Tests
- Besseres Ergebnis wenn End of Answer Token trained (Human Reinforcment training)

Fragen beantwortet:
- GPT 4 deutlich besten Ergebnisse
- LLaMA untrainiert ungefähr 1/3 der Fragen beantwortet
- LLaMA 1 Epoche sehr schwach
- LLaMA 3 Epochen erlernt wieder, kommt nicht ganz an LLaMA untrainiert heran

Stärken von GPT4
Multi-Fakt Fragen

Schwächen von GPT4
Transfer und Single-Fakt Fragen

Stärken von LLaMA untrainiert
Multi-Fakt Fragen

Schwächen von LLaMA untrainiert
Single Fakt Fragen

Stärken von LLaMA 3 Epochen
Transfer Fragen

Schwächen von LLaMA 3 Epochen
Single Fakt Fragen


Vergleich LLaMA mit GPT4 durchgängig bessere Ergebnisse bei allen Fragenarten
Vergleich LLaMA 1 Epoche mit anderen Modellen durchgängig schlechtere Ergebnisse bei allen Fragenarten
Vergleich LLaMA 3 mit LLaMA 0:
- Minimale Verbesserung bei Transfer Fragen
- Wesentlich mehr unbeantwortete Fragen bei Singulär
- Wesentlich mehr unbeantwortete Fragen bei Multi-Fakt

Analyse der Korrektheit
- Betrachtung der MakroF1 Werte bestätigt totale Anzahl der Fragen Ergebnisse
- GPT4 erreicht 0.7 MakroF1, vergleichbar mit ERgebnissen aus GPT4 Paper (Medical Knowledge)
- LLaMA auch im untrainierten Zustand deutlich unter normaler Leistung
- MakroF1 auch im Transfer Bereich besser durch traininng (hier klarerer Unterschied)
- Leichte Verbesserung im Single Bereich zu erkennen
- Deutlich schlechter in Multi-Fakt Fragen

Begründung der Ergebnisse
- GPT4 durch Größe unschlagbar
- hier nur mit LLaMA 7B verglichen, zu erwarten bei größere Modelle deutlich bessere Leistung
- Verlauf des Trainings einschätzbar: 1 Epoche deutlich zu wenig
- temporäres Overfitting an Formattierungsarten (Überschriften, Aufzählungen)
    hier Beispiel von LLaMA 1 einfügen
- LLaMA 3 Epoche deutlich bessere Leistung als LLaMA 1 Epoche, mögliche Steigerung durch weiteres Training
- generell sehr wenig Trainigsdaten (600kB), Gefahr des overfittings nach 3 Epochen existent
- weitertrainieren nicht möglich, beschrieben in Kapitel 5
- LLaMA 3 erlernt keine falschen Fakten, deutlich mehr Fragen unbeantwortet, während LLaMA 0 mehr Fragen falsch beantwortet

- Tatsächliche Steigerung bei Transfer-Fragen
Ergebnisse enthalten jedoch gravierende Formattierungs- und Wiederholungsfehler

Modelle im aktuellen Zustand nicht nutzbar
    hier Beispiel von LLaMA 3 einfügen


Analyse Erklärbarkeit
- GPT 4 auch hier unschlagbar, nahezu jede Antwort enthält Erklärung
- LLaMA unbeeinflusst, 50\% hier mit Erklärung
- Verbesserung durch Human Reinforcement Training möglich, um Erklärungen zu belohnen (nicht existent im Trainingsdatensatz)

Analyse Fragenverständnis
- Fragenverständnis nicht äquivalent zu "Fragen beantwortet"
- Fragenverständis bei LLaMA 0 deutlich höher als bei LLaMA 3
- GPT4 versteht so gut wie jede Frage

Analyse Robustheit
- Rechtschreibfehler führen zu falschen Antworten
- Multi am wenigsten beeinflusst
- Single großer Abfall bei LLaMA 3
- Transfer Leistung halbiert

- Auch GPT4 schwierigkeiten bei Transfer
- andere Fragenarten nicht beeinflusst

- LLaMA 3 deutlich schlechtere Robustheit insgesammt
- LLaMA 0 profitiert etwas durch Fehler, erhöhter MakroF1 Wert (vollständigere Antworten)

- Rechtschreibfehler in Fachbegriffen erschweren Faktenwiedergabe, da Tokenization Wörter anders
unterteilt (teilweise auch in einzelne Buchstaben)
- schlechtere Leistungen zu erwarten gegenüber ohne Fehler
- zeigt generelles Verständnis von Fakten unabhängig von Textrepräsentation
- LLaMA 0 enthaltenes Wissen durch größere Datenmengen fundierter
- LLaMA 3 durch einzelnens Buch (wenig Daten) erst oberflächliche Wissensaneignung
- LLaMA 3 scheint teile des LLaMA 0 Wissens ersetzt / versteckt zu haben

Gesamtanalyse
- LLaMA 3 kommt nicht an Leistungen von GPT4 oder LLaMA 0 heran
- Zeigt durch Vergleich mit LLaMA1 jedoch Steigende Verbesserung, die möglicherweise fortgesetzt wird durch längeres Training
- LLaMA 3 zeigt bessere Ergebnisse bei Transfer Fragen und präferiert keine Antwort über einer falschen Antwort
- Erklärbarkeit und Fragenverständnis sind nur in Teilen gegeben, folgen jedoch ähnlichen Ergebnissen wie LLaMA 0
    - keine zusätzliche Verbeserung durch Trainingsdaten erreicht
- Robustheit zeigt, dass Wissen oberflächlich erlernt wurde, jedoch nicht fundiert verankert ist
    - auch hier Steigerung durch längeres Training zu erwarten

Leistungssteigerung möglich durch:
- Längeres Training (mehr Epochen)
    - hier gefahr des overfittings erkennen (bei 3 Epochen noch nicht erreicht)
- Größere Modelle generelle Verbesserung zu erwarten (LLaMA 13B, 33B)
- Mehr Trainingsdaten
    - sinkt Gefahr des overfittings
    - Wissen wird fundierter erlernt, da in unterschiedlicher Formulierung enthalten
- Leistungssteigerung durch Human Reinforcement Learning
    - bessere Erklärbarkeit zu erwarten


