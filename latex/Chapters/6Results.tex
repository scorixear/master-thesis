%*****************************************
\chapter{Ergebnisse}\label{ch:results}
%*****************************************

% - Ergebnisse der einzelnen Tests
% - Besseres Ergebnis wenn End of Answer Token trained (Human Reinforcment training)

% Fragen beantwortet:
% - GPT 4 deutlich besten Ergebnisse
% - LLaMA untrainiert ungefähr 1/3 der Fragen beantwortet, liefert Antwort zu Großteil der Fragen
% - LLaMA 1_v100: sehr schwach, so gut wie kein Verständis
% - LLaMA 1_a30: vergleichbar mit 0e, beantwortet leicht weniger Fragen
% - LLaMA 3_v100:  erlernt wieder, kommt nicht ganz an LLaMA untrainiert heran, vergleichbar mit LLaMA 1_a30
% - LLaMA 3_a30:  deutliche Leistungssteigerung über 0e und anderen Modellen,
% beantwotet fast gleich viele Fragen wie 0e, beantwortet 1/3 Fragen mehr als llama 0e richtig
% - LLaMA 5e_a30: keine Leistungssteigerung, Entwicklung in Richtung keine Antwort gegenüber falscher Antwort
% - LLaMA 10e_a30: Resultate scheinen finales Ergebnis erreicht zu haben, keine Leistungssteigerung, Entwicklung in Richtung keine Antwort gegenüber falscher Antwort

% Stärken von GPT4
% Keine Präferenz in Fragentyp
% Kann Fragen des Buches deutlich besser beantworten 93%

% Schwächen von GPT4
% keine eindeutige Schwäche bei Fragentyp
% IS_2022_07_18: 55% (wenig fragen)
% IS_2022_09_27: 70% (wenig fragen)

% Stärken von LLaMA untrainiert
% Multi-Fakt Fragen (2x mehr als andere Typen, 39\%)
% IS 2022_07_18 55% richtig, aber wenig Fragen - keine genauen Angaben möglich
% A_2021: 27%

% Schwächen von LLaMA untrainiert
% Single  20\%  & Transfer 21% -richtig beantwortet
% Andere Sources:
% IS_2022_09_27: 25%
% Book_ 24%

% Stärken von LLaMA v100 1 Epochen
% generell Schlecht

% Schwächen von LLaMA v100 1 Epochen
% ---

% Stärken von LLaMA a30 1 Epochen
% Multi 36%
% A2021: 40%

% Schwächen von LLaMA a30 1 Epochen
% Single: 5%
% IS_2022_09_27: 12%


% Stärken von LLaMA v100 3 Epochen
% Transfer Fragen: 26%
% IS_2022_09_27: 22%

% Schwächen von LLaMA v100 3 Epochen
% Single Fakt Fragen: 17%
% A_2021: 13%

% Stärken von LLaMA a30 3 Epochen
% Transfer: 52%
% A_2021: 54%

% Schwächen von LLaMA a30 3 Epochen
% Single: 32%
% IS_2022_09_27: 32%

% Stärken von LLaMA a30 5 Epochen
% Multi: 52%
% Book: 63%

% Schwächen von LLaMA a30 5 Epochen
% Single: 26%
% IS_2022_09_27: 19%

% Stärken von LLaMA a30 10 Epochen
% Multi: 47%
% Book: 60%

% Schwächen von LLaMA a30 10 Epochen
% Transfer: 26%
% IS_2022_09_27: 25% (IS_2022_07_18: 22% (wenig fragen))

% Vergleich LLaMA mit GPT4 durchgängig bessere Ergebnisse bei allen Fragenarten
% Vergleich LLaMA v100 1 Epoche mit anderen Modellen durchgängig schlechtere Ergebnisse bei allen Fragenarten
% Vergleich LLaMA a30 1 mit LLaMA 0:
% - minimal Verbesserung transfer
% - Vergleichbar Multi
% - Schlechter Single
% - generell mehr Fragen unbeantwortet
% - Besser bei A_2021
% - Schlechter bei IS_2022_09_27, IS_2022_07_18 und Book
% - Deutlich mehr unbeantwortet bei IS_2022_09_27 und IS_2022_07_18

% Vergleich LLaMA v100 3 mit LLaMA 0:
% - Minimale Verbesserung bei Transfer Fragen
% - Wesentlich mehr unbeantwortete Fragen bei Singulär
% - Wesentlich mehr unbeantwortete Fragen bei Multi-Fakt
% - Vergleichbar bei IS_2022_09_27 und Book
% - Schlechter bei A_2021 und IS_2022_07_18
% - A2021, IS_2022_09_27 wesentlich mehr unbeantwortet

% Vergleich LLaMA a30 3 mit LLaMA 0:
% - Verbesserung bei Transfer und Single
% - kleine Verbesserung bei Multi
% - Alle Fragen bei transfer verstanden
% - Leicht weniger Fragen beantwotet bei single und Multi
% - Verbesserung Book und A2021
% - Leichte Verbesserung bei IS_2022_09_27
% - Verschlechterung bei IS_2022_07_18
% - Mehr beantwortet bei Book
% - Weniger beantwortet bei A_2021

% Vergleich a30 3 mit a30 1:
% - Verbesserung bei allen Typen
% - Verbesserung bei allen Sources außer IS_2022_07_18
% - mehr Fragen beantwotet bei allen Typen
% - Mehr fragen beantwotet bei allen Sources außer A_2021

% Vergleich a30 3 mit v100 3:
% - Verbesserung bei allen Typen
% - mehr Fragen beantwortet bei allen Typen
% - Verbesserung bei allen Sources
% - Mehr Fragen beantwortet bei allen Sources außer IS_2022_07_18

% Vergleich a30 5 mit Llama 0:
% - Verbesserung bei allen Typen
% - Weniger Fragen Beantwotet bei allen Typen
% - Verbesserung bei A_2021 und Book
% - Verschlechterung bei IS_2022_09_27 und IS_2022_07_18
% - Mehr fragen beantwortet bei Book
% - Weniger Fragen beantwortet bei allen Sources außer Book

% Vergleich a30 5 mit a30 3:
% - Verbesserung bei Multi
% - Verschlechterung bei Single, Transfer
% - Weniger Fragen beantwortet bei allen Typen
% - Verbesserung bei Book
% - Verschlechterung bei allen Sources außer Book
% - Weniger Fragen beantwotet bei allen Sources außer Book

% Vergleich a30 10 mit Llama 0:
% - Verbesserung bei allen Typen
% - Weniger Fragen beantwortet bei allen Typen
% - Verbesserung bei A_2021 und Book
% - Verschlechterung bei IS_2022_07_18
% - Weniger Fragen beantwortet bei allen Sources

% Vergleich a30 10 mit a30 3:
% - Vergleichbar bei allen Typen
% - Mehr Fragen beantwortet bei Multi
% - Weniger Fragen beantwortet bei Transfer
% - Besser bei Book
% - Schlechter bei IS_2022_09_27, IS_2022_07_18
% - Weniger Fragen beantwortet bei allen Sources außer A_2021

% Vergleich a30 10 mit a30 5:
% - Besser bei single
% - Mehr bei Multi
% - Weniger bei Transfer, Single
% - Besser bei IS_2022_09_27
% - Mehr bei A2021
% - Weniger bei allen außer A2021

% Ranking - Richtig Beantwortet:
% - GPT4 immer besser
% - 10e, 5e, 3e a30 besten Ergebnisse
% - v100 1e schlechteste Ergebnisse

% Ranking Beantwortet:
% - GPT4, llama0, llama 3e
% - v100 1e schlechteste Ergebnisse

% Ranking MakroF1:
% - GPT4 besstes
% - A30 3e, 10e und 5e sehr gut
% - 5e und 10e durch Book
% - v100 1e schlechteste Ergebnisse


% Analyse der Korrektheit
% - Betrachtung der MakroF1 Werte bestätigt totale Anzahl der Fragen Ergebnisse
% - GPT4 erreicht 0.7 MakroF1, vergleichbar mit ERgebnissen aus GPT4 Paper (Medical Knowledge)
% - LLaMA auch im untrainierten Zustand deutlich unter normaler Leistung
% - Doppelte Leistung von untrainiert durch 5e, 10e und 3e erreicht
% - 1e vergleichbar mit 0e
% - insbesondere Transfer-Fragen besser
% - synthetische gute Ergebnisse bei 5e, 10e durch overfitting auf Buchfragen
% - 3e erreicht gute Ergebnisse bei A2021 -> kann dadurch mithalten

% Begründung der Ergebnisse
% - GPT4 durch Größe unschlagbar
% - hier nur mit LLaMA 7B verglichen, zu erwarten bei größere Modelle deutlich bessere Leistung
% - Verlauf des Trainings einschätzbar: 1 Epoche deutlich zu wenig, so gut wie keine Leistungssteigerung
% - temporäres Overfitting an Formattierungsarten (Überschriften, Aufzählungen)
%     hier Beispiel von LLaMA 1 einfügen
% - Overfitting durch identische Antwortimitierung bei 5e und 10e
%     hier Beispiel 10e einfügen
% - Maximale Leistung bei 3e/5e erreicht, 10e keine Verbesserung
% - generell sehr wenig Trainigsdaten (600kB), Gefahr des overfittings nach 3 Epochen existent
% - weitertrainieren von v100 nicht möglich, beschrieben in Kapitel 5
% - V100 FP16 training führt zu deutlich schlechteren Leistungen
% - LLaMA erlernt keine falschen Fakten, deutlich mehr Fragen unbeantwortet, während LLaMA 0 mehr Fragen falsch beantwortet
%     - steigert sich mit steigender Epoche -> anzeichen für Overfitting

% - Tatsächliche Steigerung bei Transfer-Fragen
% Ergebnisse enthalten jedoch gravierende Formattierungs- und Wiederholungsfehler

% Modelle im aktuellen Zustand nicht nutzbar, da kein Endtoken für Antworten vorhanden
%     hier Beispiel von LLaMA 5e einfügen


% Analyse Erklärbarkeit
% - GPT 4 auch hier unschlagbar, nahezu jede Antwort enthält Erklärung, 97%
% - LLaMA 0, V100 3, a30 1 unbeeinflusst bei 50% der Fragen
% - LLama 3e, 5e, 10e deutlich besser 85% der Fragen
% - Ergebnisse unabhängig von Source oder Type
% - Verbesserung durch Human Reinforcement Training möglich, um Erklärungen zu belohnen (nicht existent im Trainingsdatensatz)

% Analyse Fragenverständnis
% - Fragenverständnis nicht äquivalent zu "Fragen beantwortet"
% - Fragenverständis bei LLaMA 0 nur durch a30 3 geschlagen (64%, 72%)
% - Ansonsten starker abfall, V100 3 nur noch bei 40%
% - GPT4 versteht so gut wie jede Frage (97%)
% - Multi Fragen bessere Leistung von Llama0 (81%), Andere Modelle bei 63%, begründet durch Leistungsabfall bei A_2021 Fragen
% - IS Fragen zeigen deutliche Leistungssteigerung und abnahme (zu wenig vs zu viel trainiert)


% Analyse Robustheit
% - Rechtschreibfehler führen zu falschen Antworten
% - LLaMA 0 profitiert etwas durch Fehler, erhöhter MakroF1 Wert (vollständigere Antworten)
% - Auch GPT4 schwierigkeiten bei Transfer
% - Andere Modelle Leistungsabfall um ungefähr 50%

% - Rechtschreibfehler in Fachbegriffen erschweren Faktenwiedergabe, da Tokenization Wörter anders
% unterteilt (teilweise auch in einzelne Buchstaben)
% - schlechtere Leistungen zu erwarten gegenüber ohne Fehler
% - zeigt generelles Verständnis von Fakten unabhängig von Textrepräsentation
% - LLaMA 0 enthaltenes Wissen durch größere Datenmengen fundierter
% - LLaMA 3e, 5e, 10e durch einzelnens Buch (wenig Daten) erst oberflächliche Wissensaneignung
% - scheint teile des LLaMA 0 Wissens ersetzt / versteckt zu haben

% Gesamtanalyse
% - LLaMA kommt nicht an Leistungen von GPT4
% - Trainingsart entscheident (v100 schlechter als Llama 0)
% - Beste Leistungen zwischen a30 3 Epochen und a30 5 Epochen
% - Ab 5 Epochen Overfitting Status -> Überaus gut bei Buchfragen, andere Fragen schlechter
% - Präferenz von Keiner Antwort gegenüber einer Antwort wird mit steigender Epoche deutlicher

% - Erklärbarkeit und Fragenverständnis ebenso unter GPT4-
%     - deutliche Verbesserung ab a30 3 Epochen erreicht
%     - zeigt wahre Leistung von 3 Epochen gegenüber 5 und 10 Epochen
% - Robustheit zeigt, dass Wissen oberflächlich erlernt wurde, jedoch nicht fundiert verankert ist
%     - Wird nicht besser mit längerem Training
%     - Overfitting wirkt entgegen Robustheit

% Leistungssteigerung möglich durch:
% - Größere Modelle generelle Verbesserung zu erwarten (LLaMA 13B, 33B)
% - Mehr Trainingsdaten
%     - sinkt Gefahr des overfittings
%     - Wissen wird fundierter erlernt, da in unterschiedlicher Formulierung enthalten
% - Leistungssteigerung durch Human Reinforcement Learning
%     - bessere Erklärbarkeit zu erwarten


