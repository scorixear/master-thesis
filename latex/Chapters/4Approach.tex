%*****************************************
\chapter{Lösungsansatz}\label{ch:approach}
%*****************************************
\section*{Vorgehen}
- Rerchnernetz Aufbau und Konfiguration (warum reicht nicht eigener Computer)

- Auswahl Autoregressives Modell
- Datenkuration Blaues Buch
- Unüberwacht Weitertrainiert = Continual Pretraining
    - Nutzung von HuggingFace Trainer
    - Parallelisierung auf Rechennetz

- Auswahl Beispielklausuren
- Beantwortung von Fragen

- Modellvergleich
    - Omar folgende Kriterien
    - Precision, Recall, F-Score als Benchmark
- Modellvergleich GPT4
    - Omar folgende Kriterien

\section{Auswahl von Sprachmodellen}

- Modelle mit speziellen Eigenschaften
    - Transformer-Architektur
    - Decoder-basiert
    - Autoregressiv
    - Gewichte sind verfügbar

- Modelle mit gewünschten Eigenschaften
    - gute Zero-Shot Performance
    - ohne Fine-Tuning nutzbar
    - überschreitet nicht Kapazität des Rechnernetzes

- Modelle zur Auswahl
    - GPT-2
    - GPT-3
    - GPT-4
    - GPT-J
    - GPT-NeoX
    - LLaMa

- Modelle die nicht zur Auswahl stehen
    - BERT-basierend (Text Klassifikation): BERT, RoBERTa, DistillBERT
    - ZeroShot Classification: BART
    - Übersetzung: T5, Opus
    - Zusammenfassung: Pegasus
    - Dialogbasierend: DialoGPT, Blenderbot
    - Text-2-Text: FLan-T5


- Auswahlentscheidung
    - GPT-2: klein, einfach zu trainieren, öffentlich verfügbar, keine gute ZeroShot Performance
    - GPT-3 \& GPT-4: groß, sehr gute Zeroshot Performance, nicht öffentlich verfügbar, überschreitet Kapazität des Rechnernetzes
    - GPT-J: zwischen GPT-2 und GPT-3, öffentlich verfügbar, keine gute ZeroShot Performance
    - GPT-NeoX: groß, sehr gute Zeroshot Performance, öffentlich verfügbar, überschreitet Kapazität des Rechnernetzes
    - LLaMa: verschiedene Größen - optimale Größe auswählbar, sehr gute Zeroshot Performance, öffentlich verfügbar*, wird ausgewählt

\section{Datenkuration}

- Formate die das Modell nicht verarbeiten kann
    - Bilder

Textpassagen ohne Wissen oder Kontext:
    - Bild-Beschreibun
    - Seitenzahlen
    - Vorwort
    - Autorenvorstellung

Optionale Textentfernung mit potentiell besserer Leistung:
    - Inhaltsverzeichnis
    - Literaturverzeichnis
    - Stichwortverzeichnis
    - Überschriften

\section{Unüberwachtes Weitertrainieren}
- HuggingFace Trainer
- Parallelisierung auf verschiedenen GPUS
- Kommunikation zwischen GPUS mit Paket

- Alternativen zu Trainer
    - PyTorch
    - PyTorchLighning
    - Tensorflow

- Alternativen zu SLURM: keine, vorgegeben

\section{Ausführen des Training-Programme}
- Nutzung von SLURM
- Kommunikation zwischen GPUS und Nodes
- Long-Time Nodes
- Speicherung von Modellen mit Checkpoints
- Größe der Modelle = Anzahl notwendiger GPUs

\section{Klausurfragen}
- Klausurfragen Antwortenerstellung
- Umformulierung zu Prompt (da Finetuning nicht vorhanden)

\section{Modellvergleich}
- Berechnung von Precision, Recall, F-Score
- Aggregation der Ergebnisse
- Vergleich der Modelle
- Vergleich mit SOTA Modell
    - Nutzung von Kontext als Input?

