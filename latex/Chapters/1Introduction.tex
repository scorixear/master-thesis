%************************************************
\chapter{Einleitung}\label{ch:introduction}
%************************************************
\section{Gegenstand}
Die effiziente und effektive Informationsgewinnung ist in der Medizin ein klarer Grundbaustein einer guten klinischen Praxis. Jegliche medizinische Handlung nutzt und produziert eine große Menge an Informationen, sei es Publikationen als Basis für eine Diagnose oder Dokumentation über den Behandlungsprozess. Diese Informationen strukturiert und klassifiziert abzuspeichern und wiederzugeben ist ein fortlaufender Entwicklungsprozess und Teil aktueller Forschung.

Die Digitalisierung der Medizin ist ein enormer Themenbereich mit wachsender Notwendigkeit. Die Medizinische Informatik beschreibt  \enquote{die Wissenschaft der systematischen Erschließung, Verwaltung, Aufbewahrung, Verarbeitung und Bereitstellung von Daten, Informationen und Wissen in der Medizin und im Gesundheitswesen. [...]}\citep{gmds}.

In der Lehre wird die Praxis der medizinischen Informatik durch umfassende Literatur wie zum Beispiel in \citet{bb} unterstützt. Zur Strukturierung dieser Literatur existiert die Ontologie SNIK \citep{snikgraphposter}, eine semantisches Netzwerk klassifiert in das SNIK Metamodell. Der öffentliche Zugriff auf dieses Netzwerk 


Situation:
Global:

Universität und Institut

das Blaue Buch
\section{Problemstellung}


\section{Motivation}

\section{Zielsetzung}\label{sec:zielsetzung}


\section{Aufgabenstellung}

\section{Aufbau der Arbeit}
