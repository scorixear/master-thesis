%************************************************
\chapter{Einleitung}\label{ch:introduction}
%************************************************
\section{Gegenstand}\label{sec:gegenstand}
Eine effektive und effiziente Informationsbeschaffung bildet einen fundamentalen Bestandteil einer qualitativ hochwertigen klinischen Praxis in der Medizin. 
Bei jeder medizinischen Handlung werden große Mengen an Informationen genutzt und erzeugt, sei es als Grundlage für eine Diagnose oder zur Dokumentation des Behandlungsprozesses. 
Die strukturierte und klassifizierte Speicherung und Wiedergabe dieser Informationen ist ein kontinuierlicher Entwicklungsprozess und Gegenstand aktueller Forschung.\\

Die Digitalisierung der Medizin ist ein weites Themenfeld mit stetig wachsendem Bedarf. 
Die Medizinische Informatik beschreibt dabei~\enquote{die Wissenschaft der systematischen Erschließung, Verwaltung, Aufbewahrung, Verarbeitung und Bereitstellung von Daten, Informationen und Wissen in der Medizin und im Gesundheitswesen. [$\dots$]}\footnote{\raggedright{}\ac{gmds} (2023). Definition Medizinische Informatik.\ \url{https://www.gmds.de/aktivitaeten/medizinische-informatik/} (abgerufen am 07. 03. 2023).}.
Vor diesem Hintergrund gewinnt die Entwicklung und Implementierung effizienter Informationssysteme und Technologien zur Unterstützung der klinischen Praxis zunehmend an Bedeutung.\\

In der Lehre wird die Praxis der Medizinischen Informatik durch umfangreiche Literatur, z.B. in \citet{bb}, unterstützt.
Zur Strukturierung von Fachbegriffen und Rollen des Informationsmanagements im Krankenhaus existiert die Ontologie SNIK \citep{semantischesnetz}, ein semantisches Netz, kategorisiert in der Metaontologie SNIK und Teil des Projekts SNIK des Instituts für Medizinische Informatik, Statistik und Epidemiologie%
\footnote{\raggedright{}Institut für Medizinische Informatik, Statistik und Epidemiologie.\ \url{https://www.imise.uni-leipzig.de/Institut} (besucht am 09. 03. 2023).} an der Universität Leipzig.
Die Nutzung dieses Netzes ermöglicht eine systematische Darstellung von Rollen, Entitäten und Funktionen des Informationsmanagements im Krankenhaus, unabhängig von der Definition der zugrunde liegenden Literaturquellen.\\

Die Bedeutung von maschinellem Lernen, Deep Learning und Sprachmodellen ist in der heutigen Zeit sehr präsent. 
Diese Technologien werden in vielen Bereichen, von der Automobilindustrie bis hin zu medizinischen Anwendungen, eingesetzt, um neue Methoden der Informationsgewinnung und -verarbeitung zu ermöglichen. 
Verschiedene KI-Modelle können in vielen Bereichen, wie z.B. der Erkennung seltener Krankheiten \citep{rare_diseases}, der personalisierten Medizin \citep{precision_med} oder dem Marketing \citep{ai_marketing}, weitreichende Auswirkungen auf zukünftige Arbeitsprozesse haben.
Sprachmodelle wie das \ac{gpt}-3-Modell von OpenAI \citep{gpt3} können dazu beitragen, Texte in verschiedenen Sprachen automatisch zu übersetzen und sogar kreatives Schreiben zu ermöglichen. 
Deep Learning, das auf künstlichen neuronalen Netzen basiert, ermöglicht eine noch tiefere und komplexere Verarbeitung von Daten. 
In der Medizin kann Deep Learning beispielsweise zur Diagnose von Krankheiten und zur Analyse medizinischer Bilder eingesetzt werden \citep{skincancer}.

\section{Problemstellung}\label{sec:problemstellung}
Informationssysteme im Gesundheitswesen sind hochkomplex, abhängig von den Anforderungen der jeweiligen Organisation und unterliegen einer ständigen Weiterentwicklung.
Die Anforderungen sind vielfältig und umfassen beispielsweise die Speicherung und Verarbeitung von Patientendaten, die Dokumentation von Behandlungsprozessen, die Unterstützung von Diagnose- und Therapieprozessen sowie die Unterstützung von Forschungsaktivitäten.
Ausgehend von diesen Anforderungen gibt es eine Fülle von Literatur, die sich mit der Definition, Implementierung und Wartung von Informationssystemen im Gesundheitswesen befasst. 
Das Buch \citetitle{bb} von \citeauthor{bb} beschäftigt sich umfassend mit diesen Anforderungen und ist im März 2023 in der 3. Auflage erschienen \citep{bb}.\\

Das Management von Informationssystemen ist eine anspruchsvolle Aufgabe, die sich nicht nur auf die Anwendung durch das Krankenhauspersonal beschränkt.
Es ist auch von großer Bedeutung für Studierende, um ein besseres Verständnis der bestehenden Systeme zu erlangen, und für Wissenschaftler, um diese Systeme zu erweitern oder neue Managementmethoden zu entwerfen.
Eine konkrete und konsistente Wissensbasis ist sowohl für die Anwendung als auch für Lehre und Forschung von großer Bedeutung.\\

Eine weitere Herausforderung bei der Auseinandersetzung mit der Literatur zu Informationssystemen ist die Komplexität der Übertragung von theoretischen Konzepten auf praktische Anwendungsfälle.
Insbesondere für Studierende und Praktiker erfordert die praktische Anwendung ein tiefes Verständnis der Zusammenhänge und der Anwendbarkeit der vorgestellten Konzepte auf konkrete Arbeitsumgebungen. 
Da die verfügbare Literatur oft sehr umfangreich und ihre Definitionen fragmentiert sind, stellt die Identifikation relevanter Informationen für spezifische Problemstellungen eine weitere Herausforderung dar.
Die Fragmentierung von Definitionen bezieht sich hier auf die Erläuterung eines Fachbegriffs oder Konzepts innerhalb einer Literaturquelle.
Diese Definitionen werden häufig nicht zusammenfassend aufgelistet, sondern ergeben sich im Laufe der Texte und Kapitel.
Dies führt dazu, dass große Teile eines Buches gelesen werden müssen, um einzelne Konzepte und ihre Beziehungen zu anderen Themen vollständig zu erfassen.\\

Der Umfang der Literaturquellen und die Fragmentierung bei der Definition von Fachbegriffen
erschweren eine schnelle Informationsbeschaffung, insbesondere für Studierende, die grundlegende Konzepte richtig verstehen wollen.\\

\begin{itemize}
  \item Problem: Schwierigkeiten bei der Informationsbeschaffung aufgrund des Umfangs von \citet{bb} und der Fragmentierung von Definitionen
\end{itemize}

\section{Motivation}

Eine Strukturierung von Informationen zum Management von Krankenhausinformationssystemen ist bereits Bestandteil des Projekts SNIK \citep{semantischesnetz, domaene}, 
ein semantisches Netz des Instituts für Medizinische Informatik, Statistik und Epidemiologie der Universität Leipzig.
Auf Basis dieses Netzes wurden bereits verschiedene Methoden zur Informationsextraktion untersucht.\\

Die BeLL mit dem Titel \enquote{Question Answering on SNIK} \citep{hannesbell} erweiterte den Zugang zu diesem Netz durch die Verwendung natürlicher englischer Sprache. 
Die Ergebnisse der BeLL wurden in einem \ac{qas} QAnswer \citep{qanswer} umgesetzt, zeigen aber Defizite in der Erklärbarkeit und Verständlichkeit komplexerer Fragen. 
Im Gegensatz dazu untersuchte \citet{chatgpt_qas} auf einem anderen Datensatz die Leistung einer Konversations-KI (in diesem Fall ChatGPT) im Vergleich zu einer herkömmlichen, auf einem Wissensgraphen basierenden \ac{qas} (verwendet wurde $KGQA_N$).
Die Ergebnisse zeigten, dass ChatGPT im Vergleich zum verwendeten $KGQA_N$ erstaunlich stabile und verständliche Antworten lieferte, jedoch bei der Ausgabe korrekter Antworten deutlich schlechter abschnitt. Hier steht zu erwarten, dass durch ein besseres Feintuning die Anzahl der falschen Antworten deutlich reduziert werden kann.\\

Es ist daher notwendig zu untersuchen, ob der Einsatz einer Konversations-KI die Anwendungsschwierigkeiten in QAnswer beheben kann und ob die Wiedergabe mit niedrigem Wahrheitswert durch Feintuning verbessert werden kann.

\section{Zielsetzung}\label{sec:zielsetzung}

Dem in~\ref{sec:problemstellung} gezeigten Problem werden folgende Ziele dieser Arbeit zugeordnet.
\begin{itemize}
  \item Ziel Z1: Beantwortung von Fragen zu Informationssystemen im Gesundheitswesen in natürlicher Sprache durch eine Konversation-KI mit Hilfe von \citet{bb}
  \item Ziel Z2: Lösung einer Beispielklausur des Moduls \enquote{Architektur von Informationssystemen im Gesundheitswesen}\footnote{\raggedright{}einem Modul des Masterstudiengangs Medizinische Informatik an der Universität Leipzig, das inhaltlich auf \citet{bb} aufbaut} mit Hilfe einer Konversations-KI.\@
  Das Ergebnis wird kein produktives System darstellen, sondern lediglich die Machbarkeit der Beantwortung von Fragen mit Hilfe einer Konversations-KI aufzeigen.
\end{itemize}

\section{Bezug zu ethischen Leitlinien der GMDS}

Die ethischen Leitlinien der \ac{gmds} \citep{gmds_eth} geben \enquote{sowohl den tragenenden Gesellschaften als Institution als auch dem einzelnen Mitglied eine Orientierung,
welche ethischen Forderungen in ihrem bzw.\ seinem jeweiligen Aufgaben- und Verantwortungsbereicht relevant sein können.}\footnote{\label{ft:gmds}\citet{gmds_eth}}
Aufgeteilt in 16 Artikel werden hier verschiedene Kompetenzen und Verantwortlichkeiten definiert, unter die auch das hier beschriebene System fällt.
Da die zu entwickelnde Konversations-KI (im folgenden KI genannt) Informationen über medizinisches Wissen, insbesondere über Informationssysteme im Gesundheitswesen, bereitstellt,
unterliegt sie in ihrer Existenz als \enquote{Informationsbasis} einer Vielzahl von Artikeln.
Sie wirkt unterstützend in den Artikeln 1 \enquote{Auftrag}, 2 \enquote{Fachkompetenz}, 3 \enquote{Kommunikative Kompetenz}, 
kann aber durch unsachgemäßen Gebrauch und ihr zugrundeliegendes Wesen entgegen den Artikeln 1 \enquote{Auftrag}, 2 \enquote{Fachkompetenz}, 3 \enquote{Kommunikative Kompetenz},
4 \enquote{Medizinethische Kompetenz}, 6 \enquote{Soziale Verantwortung} und 13 \enquote{Forschung} handeln.\\

In Artikel 1 \enquote{Auftrag} heißt es unter anderem, dass 
\enquote{Die Würde des Menschen und das Persönlichkeitsrecht [$\dots$] dabei vorrangig geachtet und geschützt werden [müssen]}\footref{ft:gmds}.
Dazu gehört insbesondere die \enquote{Allgemeine Erklärung der Menschenrechte}\footnote{\url{https://www.un.org/depts/german/menschenrechte/aemr.pdf} (abgerufen am 24.04.2023) },
in der das \enquote{Verbot der Diskriminierung z.B. nach Geschlecht oder Rasse (Art. 2, 7)} verankert ist und 
der \enquote{Schutz des (geistigen) Eigentums (Art. 17, 27)} und die \enquote{Gedanken-, Gewissens-, Religions- und Meinungsfreiheit (Art. 18, 19)} beschrieben werden.
Die von der KI zu generierenden Antworten ergeben sich aus einer Vielzahl von zuvor genutzten Datensätzen aus dem Internet, die nicht notwendigerweise diesen ethischen Leitlinien folgen.
Eine inhaltliche Garantie für die Einhaltung dieser Leitlinien ist daher ohne zusätzliche Filterung der Antworten zu gestellten Fragen nicht möglich. 
Durch eine optimale Filterung können Antworten, die den Leitlinien widersprechen, ausgeschlossen werden. 
Diese Filterung ist jedoch aufgrund ihrer Komplexität sowohl in zeitlicher als auch in finanzieller Hinsicht nicht Bestandteil dieser Arbeit.
Es muss daher davon ausgegangen werden, dass es durchaus Antworten geben kann, die gegen die Leitlinien verstoßen.\\

Artikel 1 wird durch die KI gefördert, in dem sie Medizinische Versorgungseinrichtungen durch effiziente und schnelle Informationsbeschaffung in ihren Fähigkeiten unterstützt, 
\enquote{ihre Leistungen qualitativ und quantitativ nachweisen, überwachen und sicherstellen [zu] können.}\footref{ft:gmds}.\\

Artikel 2 \enquote{Fachkompetenz} definiert, dass das Mitglied seine \enquote{Fachkompetenz nach dem Stand der Wissenschaft und Technik erwirbt [$\dots$][und] Maßnahmen zur Fehlervermeidung ergreift}\footref{ft:gmds}. 
Dies ist teilweise durch die KI gegeben, da die extrahierten Informationen auf dem Stand des zugrundeliegenden Buches \citet{bb} sind. 
Eine Fehlervermeidung ist hier jedoch nicht vorgesehen. 
Wie in \citet{chatgpt_qas} gezeigt, weist ChatGPT als \ac{gpt}-Modell eine geringe Wahrheitsquote auf, 
wodurch die fehlerfreie Beantwortung von Fragen zwar angestrebt, aber nicht garantiert werden kann.\\

Artikel 3 \enquote{Kommunikative Kompetenz} beschreibt die Fähigkeit 
\enquote{Recht, Interessen [und] Konventionen der verschiedenen von [dem Mitglied] seiner Arbeit Betroffenen zu verstehen und zu berücksichtigen [$\dots$][und]
Wissenschaftliche Erkentnisse in verständlicher Form der Öffentlichkeit zugänglich [zu machen][$\dots$]}\footref{ft:gmds}.
Die KI unterstützt dabei den/die Fragende/n durch eine Antwort in möglicherweise verständlicherer Form der wissenschaftlichen Erkenntnisse in \citet{bb}, 
jedoch ist hier nicht gegeben, dass die Betroffenen in der Antwort berücksichtigt oder verstanden wurden.\\

Artikel 4 \enquote{Medizinethische Kompetenz} legt fest, dass das Mitglied 
\enquote{ethische Prinzipien der Medizin [$\dots$] bei seinem beruflichen Handeln beachtet}\footref{ft:gmds}.
Zu den ethischen Prinzipien\footref{ft:gmds} gehören auch die \enquote{Achtung vor dem Menschen zu wahren} und die \enquote{Würde des Invidiums zu schützen}.
Beides kann nicht allein durch die KI gewährleistet werden und muss durch mögliche Filter ergänzt werden.\\

Artikel 6 \enquote{Soziale Verantwortung} definiert die Verantwortungen des Mitglied 
\enquote{gesellschaftliche Auswirkungen [zu] berücksichtigen [$\dots$] [und die] Allgemeine Erklärung der Menschenrechte und ethische[n] Prinzipien der Medizin [zu beachten]}\footref{ft:gmds}. 
Wie bereits zu den Artikeln 1 und 4 ausgeführt, ist es der KI aufgrund der Datengrundlage nicht möglich, diesen Artikel, insbesondere die hier genannten Punkte, in ihrer Antwort zu berücksichtigen.\\

Artikel 13 \enquote{Forschung} definiert, dass das Mitglied \enquote{gute Wissenschaftliche Arbeit, inbesondere Offenheit und Transparenz [und] Akzeptanz von Kritik}\footref{ft:gmds}
einhalten soll. 
Gute wissenschaftliche Arbeit wird weiter definiert als 
\enquote{die strikte Ehrlichkeit im Hinblick auf die Beiträge von Partnern, Konkurenten, Vorgängern zu wahren [und] weder Fälschung oder Plagiate [zu] benutzen}\footref{ft:gmds}.
Da die KI auf dem Inhalt eines Buches trainiert wird, in ihrer Datenbasis aber bereits eine Vielzahl von Texten, darunter auch Plagiate und Fälschungen, gelernt hat, 
ist es nicht möglich, die Antworten als wissenschaftliche Quelle zu verwenden, da sowohl Plagiate als auch Fälschungen ohne Annotation vorhanden sein können.
Die Antworten sollten als reine Informationsextraktion und nicht als Quelle für wissenschaftliche Arbeiten betrachtet werden.

\section{Aufgabenstellung}

Die in~\ref{sec:zielsetzung} genannten Ziele $Z_i$ werden durch die hier aufgeführten Aufgaben $A_i$ gelöst.

\begin{itemize}
  \item Aufgabe zu Ziel Z1
  \begin{itemize}
    \item Aufgabe A1.1: Es sollen aktuelle Sprachmodelle verglichen werden. Dabei sind die Einschränkungen der Verfügbarkeit und Verwendbarkeit zu berücksichtigen. 
    \ac{gpt}-Modelle basieren auf großen Datenmengen und enthalten mehrere Milliarden Parameter. 
    Ein eigenes Training eines Modells würde sowohl den zeitlichen als auch den finanziellen Rahmen übersteigen, weshalb auf ein vortrainiertes Modell zurückgegriffen werden muss. 
    Dazu muss das Modell frei verfügbar sein und unter einer Open-Source-Lizenz stehen. Außerdem muss das Modell am Rechenzentrum der Universität Leipzig geladen und trainiert werden können. Aufgrund der großen Anzahl an Parametern sind auch hier Grenzen des Arbeitsspeichers und damit der Größe des Modells gesetzt.
    \item Aufgabe A1.2: Für ein effizientes und erfolgreiches Feintuning der Konversations-KI ist eine Datenkuration von \citet{bb} notwendig. 
    Abschnitte wie Literaturverzeichnis, Buchcover oder Grafiken müssen vor der Verwendung des Textes entfernt oder umgeschrieben werden.
    \item Aufgabe A1.3: Die trainierte Konversations-KI wird dann zur Beantwortung von Fragen zu \citet{bb} verwendet. 
    Dabei wird während des Trainings der in \citet{gpt3} beschriebene \enquote{Zero-Shot}-Ansatz verfolgt.
    Dies bedeutet, dass das Verständnis der Fragestellung und die Bewertung wichtiger Informationen allein durch das Modell erfolgt und nicht durch vordefinierte Fragen im Datensatz dem Modell beigebracht wird.
  \end{itemize}
  \item Aufgabe zu Ziel Z2
  \begin{itemize}
    \item Aufgabe A2.1: Die in dieser Arbeit erstellte Konversations-KI wird vor und nach dem Training hinsichtlich ihrer Fähigkeit, Fragen korrekt zu beantworten, evaluiert.
    Dies ermöglicht eine Aussage über die Effektivität des Trainings und die Leistungssteigerung der Konversations-KI.\@
    Ebenso wird ein Vergleich mit dem aktuellen GPT-4 Modell \citep{gpt4} durchgeführt, um die Notwendigkeit eines Feintunings zu ermitteln.
    \item Aufgabe A2.2: Bewertung der Antwortoptionen von Klausurfragen nach gleichen Kriterien wie in \citet{chatgpt_qas}
  \end{itemize}
\end{itemize}

\section{Aufbau der Arbeit}
Kapitel 1 beschreibt das grundlegende Umfeld dieser Arbeit, formuliert exitierende Probleme und Anforderungen, bietet Ziele zur Lösung dieser Probleme an und gibt Aufgaben, zur Umsetzung dieser Ziele. 
In Kapitel 2 werden Grundlagen gelegt zum Verständnis der in dieser Arbeit verwendeten Technologie, während in Kapitel 3 der aktuelle Stand der Forschung zusammengefasst wird. 
Kapitel 4 umfasst Lösungsstrategien des in~\ref{sec:problemstellung} formulierten Problem mit einer anschließenden Beschreibung der Umsetzung dieser Lösungen in Kapitel 5. 
Die Ergebnisse dieser Arbeit werden in Kapitel 6 präsentiert und in Kapitel 7 zusammengefasst diskutiert. 
Zusätzlich gibt Kapitel 7 einen Ausblick dieser Arbeit.
