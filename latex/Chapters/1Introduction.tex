%************************************************
\chapter{Einleitung}\label{ch:introduction}
%************************************************
\section{Gegenstand}\label{sec:gegenstand}
Eine effektive und effiziente Informationsgewinnung bildet einen fundamentalen Bestandteil einer qualitativ hochwertigen klinischen Praxis in der Medizin. 
Bei jeder medizinischen Handlung werden große Mengen an Informationen genutzt und generiert, sei es zur Grundlage einer Diagnose oder zur Dokumentation des Behandlungsprozesses. 
Die strukturierte und klassifizierte Speicherung sowie Wiedergabe dieser Informationen stellt einen fortwährenden Entwicklungsprozess dar und ist Gegenstand aktueller Forschung.\\

Die Digitalisierung der Medizin ist ein ausgedehntes Themenfeld mit stetig wachsender Notwendigkeit. 
Die Medizinische Informatik beschreibt passend dazu \enquote{die Wissenschaft der systematischen Erschließung, Verwaltung, Aufbewahrung, Verarbeitung und Bereitstellung von Daten, Informationen und Wissen in der Medizin und im Gesundheitswesen. [...]}\citep{gmds}. 
In diesem Kontext wird die Bedeutung der Entwicklung und Implementierung effektiver Informationssysteme und Technologien zur Unterstützung der klinischen Praxis immer größer.\\

In der Lehre wird die Praxis der medizinischen Informatik durch umfassende Literatur wie zum Beispiel in \citet{bb} unterstützt. 
Zur Strukturierung dieser Literatur existiert die Ontologie SNIK \citep{snikgraphposter}, ein semantisches Netzwerk klassifiert in das SNIK Metamodell und Projekt des Instituts für Medizinische Informatik, Statistik und Epidemiologie \citep{imise} an der Universität Leipzig.
Durch die Verwendung dieses Netzwerkes wird eine systematische Darstellung von Rollen, Entitäten und Funktionen des Informationsmanagements in Krankenhäusern ermöglicht, losgelöst von den Definition der zugrunde liegenden Literaturquellen.\\

Die Bedeutung von maschinellem Lernen, Deep Learning und Sprachmodellen nimmt in der heutigen Zeit immer mehr zu. Diese Technologien werden in vielen Bereichen eingesetzt, von der Automobilindustrie bis hin zu medizinischen Anwendungen, um neue Methoden der Informationsgewinnung und -verarbeitung zu bieten. 
Eine Studie von \citet{mckinsey} zeigt, dass maschinelles Lernen und künstliche Intelligenz in den nächsten Jahren in der Wirtschaft zunehmend eingesetzt werden und dabei helfen werden, die Effizienz und Produktivität zu steigern.
Sprachmodelle wie beispielsweise das GPT-3-Modell von OpenAI \citep{gpt3} können dazu beitragen, Texte in verschiedenen Sprachen automatisch zu übersetzen und sogar kreative Schreibarbeiten zu erledigen. 
Deep Learning, das auf künstlichen neuronalen Netzwerken basiert, ermöglicht eine noch tiefere und komplexere Verarbeitung von Daten. 
In der Medizin kann Deep Learning zum Beispiel bei der Diagnose von Krankheiten und bei der Analyse von medizinischen Bildern eingesetzt werden \citep{skincancer}.

\section{Problemstellung} \label{sec:problemstellung}
Das Management von Informationssystemen ist eine komplexe Aufgabe, die sich nicht nur auf die Anwendung durch Mitarbeiter in Krankenhäusern beschränkt. Es ist auch wichtig für Studierende, um ein besseres Verständnis der vorhandenen Systeme zu erlangen, sowie für Wissenschaftler, um diese Systeme zu erweitern oder neue Methoden zum Management zu entwerfen.
Eine konkrete und konsistente Wissensbasis ist sowohl für die Anwendung als auch die Lehre und Forschung von großer Bedeutung. Die Forschung liefert hierfür eine Fülle von Literatur zu verschiedenen Aspekten der Medizinischen Informatik und dem Management von Informationssystemen.\\

Zwischen verschiedenen Literaturquellen herrscht jedoch oft Inkonsistenz bei der Definition und Benennung von Fachbegriffen. 
Zum Beispiel wird der Begriff \enquote{Krankenhausinformationssystem (engl. Hospital Information System)} in \citet{kis_winter} als das sozio-technische Subsystem eines Krankenhauses beschrieben,
das aus allen assoziierten menschlichen und technischen Akteuren, den damit verbundenen informationsverarbeitenden Rollen und allen Prozessen zur Verarbeitung dieser Informationen besteht.
Im Gegensatz dazu schließt die Definition in \citet{kis_italy} keine menschlichen Akteure ein.\\

Die Vielzahl an verfügbaren Literaturquellen und die Seitenanzahl der einzelnen Quellen erschweren eine schnelle Informationsbeschaffung, insbesondere für Studierende, die grundlegende Konzepte korrekt verstehen möchten. 
Darüber hinaus führen semantische Unterschiede bei der Definition von Fachbegriffen zu zusätzlichen Hindernissen.\\

\begin{itemize}
  \item Problem P1: Schwierigkeiten bei der Informationsbeschaffung aufgrund der Fülle an Literatur und Seitenanzahl
  \item Problem P2: Inkonsistenz bei der Definition von Fachbegriffen zwischen verschiedenen Literaturquellen
\end{itemize}

\section{Motivation}

Allgemeines Wissen, Informationen und Daten werden in heutiger Zeit über Google und Wikipedia gefunden. 
Die Suche nach Fragen durch Google und die korrekte Beantwortung jener durch einen kurzen Ausschnitt aus der Definition des Begriffs in Wikipedia bieten dem normalen Nutzer eine einfache und schnelle Möglichkeit, seine/ihre Fragen in natürlicher Sprache zu formulieren und Antworten in natürlicher Sprache zu erhalten.\\

Dies erfordert jedoch enorme Rechenleistungen und große Datensätze um konkretes und korrektes Wissen wiederzugeben. 
Weitergehend sind Wissen und Informationen über die Medizinische Informatik selten im selben Detailgrad in diesen Ressourcen vorhanden, wie sie in den Literaturquellen von SNIK vorliegen. 
Zur Unterstützung von Lehrenden und Studierenden soll diese Arbeit eine Möglichkeit bieten, ähnlich dem Prozess der gerade beschriebenen normalen Suche im Internet, Wissen und Informationen aus den Literaturquellen zu extrahieren, welche auch als Basis für die Erstellung des SNIK-Projektes genutzt wurden.\\

Dies ermöglicht jenen Nutzenden die Problematiken bei dem Umgang mit SNIK zu umgehen und bietet eine weitere Methodik um digitale Bildung besser in den Arbeitsfluss einer Universität zu integrieren.

\section{Zielsetzung}\label{sec:zielsetzung}

Den in \ref{sec:problemstellung} gezeigten Problemen $P_i$ werden Ziele $Z_i$ dieser Arbeit zugeordnet.

\begin{itemize}
  \item Ziel Z1: Zugriff auf Literaturquellen von SNIK mit Hilfe eines ChatBots
  \item Ziel Z2: Wiedergabe von Wissen aus den Literaturquellen von SNIK in natürlicher Sprache beispielhaft gezeigt an Hand von \citet{bb} als Wissensressource
\end{itemize}

\section{Aufgabenstellung}

Die in \ref{sec:zielsetzung} genannten Ziele $Z_i$ werden durch die hier aufgeführten Aufgaben $A_i$ gelöst.

\begin{itemize}
  \item Aufgabe zu Ziel Z1
  \begin{itemize}
    \item Aufgabe A1.1: Implementierung einer Webseite, welche natürlich gestellte Fragen entgegen nimmt
    \item Aufgabe A1.2: Nutzung eines General Pre-trained Transformers (GPT) zur Bewertung, Verständis und Beantwortung der gegeben Fragen
  \end{itemize}
  \item Aufgabe zu Ziel Z2
  \begin{itemize}
    \item Aufgabe A2.1: Erstellung eines Fragenkatalog aus der Literaturquelle
    \item Aufgabe A2.2: Fein-Tuning des GPT-Modells auf die Literaturquelle
    \item Aufgabe A2.3: Analyse und Bewertung der ausgegebenen Daten
  \end{itemize}
\end{itemize}

\section{Aufbau der Arbeit}
Kapitel 1 beschreibt das grundlegende Umfeld dieser Arbeit, formuliert exitierende Probleme und Anforderungen, bietet Ziele zur Lösung dieser Probleme an und gibt Aufgaben, zur Umsetzung dieser Ziele. 
In Kapitel 2 werden Grundlagen gelegt zum Verständnis der in dieser Arbeit verwendeten Technologie, während in Kapitel 3 der aktuelle Stand der Forschung zusammengefasst wird. 
Kapitel 4 umfasst Lösungsstrategien der in \ref{sec:problemstellung} formulierten Probleme mit einer anschließenden Beschreibung der Umsetzung dieser Lösungen in Kapitel 5. 
Die Ergebnisse dieser Arbeit werden in Kapitel 6 präsentiert und in Kapitel 7 zusammengefasst diskutiert. Zusätzlich gibt Kapitel 7 einen Ausblick dieser Arbeit.