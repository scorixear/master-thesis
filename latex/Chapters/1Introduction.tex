%************************************************
\chapter{Einleitung}\label{ch:introduction}
%************************************************
\section{Gegenstand}\label{sec:gegenstand}
Eine effektive und effiziente Informationsgewinnung bildet einen fundamentalen Bestandteil einer qualitativ hochwertigen klinischen Praxis in der Medizin. 
Bei jeder medizinischen Handlung werden große Mengen an Informationen genutzt und generiert, sei es zur Grundlage einer Diagnose oder zur Dokumentation des Behandlungsprozesses. 
Die strukturierte und klassifizierte Speicherung sowie Wiedergabe dieser Informationen stellt einen fortwährenden Entwicklungsprozess dar und ist Gegenstand aktueller Forschung.\\

Die Digitalisierung der Medizin ist ein ausgedehntes Themenfeld mit stetig wachsender Notwendigkeit. 
Die Medizinische Informatik beschreibt passend dazu~\enquote{die Wissenschaft der systematischen Erschließung, Verwaltung, Aufbewahrung, Verarbeitung und Bereitstellung von Daten, Informationen und Wissen in der Medizin und im Gesundheitswesen. [$\dots$]}
\footnote{\raggedright{}GMDS (2023). Definition Medizinische Informatik.\ https://www.gmds.de/aktivitaeten/medizinische-informatik/ (besucht am 07. 03. 2023).}.
In diesem Kontext wird die Bedeutung der Entwicklung und Implementierung effektiver Informationssysteme und Technologien zur Unterstützung der klinischen Praxis immer größer.\\

In der Lehre wird die Praxis der medizinischen Informatik durch umfassende Literatur wie zum Beispiel in \citet{bb} unterstützt.
Zur Strukturierung von Fachbegriffen und Rollen des Informationsmanagements im Krankenhaus existiert die Ontologie SNIK \citep{semantischesnetz}, ein semantisches Netz, kategorisiert in das SNIK Metamodell, und Projekt des Instituts für Medizinische Informatik, Statistik und Epidemiologie 
\footnote{\raggedright{}Institut für Medizinische Informatik, Statistik und Epidemiologie.\ https://www.imise.uni-leipzig.de/Institut (besucht am 09. 03. 2023).} an der Universität Leipzig.
Durch die Verwendung dieses Netzes wird eine systematische Darstellung von Rollen, Entitäten und Funktionen des Informationsmanagements in Krankenhäusern ermöglicht, losgelöst von den Definition der zugrunde liegenden Literaturquellen.\\

Die Bedeutung von maschinellem Lernen, Deep Learning und Sprachmodellen nimmt in der heutigen Zeit immer mehr zu. 
Diese Technologien werden in vielen Bereichen eingesetzt, von der Automobilindustrie bis hin zu medizinischen Anwendungen, um neue Methoden der Informationsgewinnung und -verarbeitung zu bieten.
Ein Bericht von \citet{hwwi} zeigt, dass maschinelles Lernen und künstliche Intelligenz in den nächsten Jahren zunehmend eingesetzt werden und deren Einfluss auf die Wirtschaft und Gesellschaft kontinuierlich zunehmen wird.
Sprachmodelle wie beispielsweise das GPT-3-Modell von OpenAI \citep{gpt3} können dazu beitragen, Texte in verschiedenen Sprachen automatisch zu übersetzen und sogar kreative Schreibarbeiten zu erledigen. 
Deep Learning, das auf künstlichen neuronalen Netzwerken basiert, ermöglicht eine noch tiefere und komplexere Verarbeitung von Daten. 
In der Medizin kann Deep Learning zum Beispiel bei der Diagnose von Krankheiten und bei der Analyse von medizinischen Bildern eingesetzt werden \citep{skincancer}.

\section{Problemstellung}\label{sec:problemstellung}
Das Management von Informationssystemen ist eine komplexe Aufgabe, die sich nicht nur auf die Anwendung durch Mitarbeiter in Krankenhäusern beschränkt.
Es ist auch wichtig für Studierende, um ein besseres Verständnis der vorhandenen Systeme zu erlangen, sowie für Wissenschaftler, um diese Systeme zu erweitern oder neue Methoden zum Management zu entwerfen.
Eine konkrete und konsistente Wissensbasis ist sowohl für die Anwendung als auch die Lehre und Forschung von großer Bedeutung.
Die Forschung liefert hierfür eine Fülle von Literatur zu verschiedenen Aspekten der Medizinischen Informatik und dem Management von Informationssystemen.\\

Ein weiteres Herausforderungsfeld in der Auseinandersetzung mit dieser Literatur besteht in der Komplexität der Übertragung von theoretischen Konzepten auf praktische Anwendungsfälle. 
Insbesondere für Studierende und Praktiker erfordert die Umsetzung von theoretischem Wissen in eine angemessene praktische Anwendung ein tiefes Verständnis der Zusammenhänge und der Anwendbarkeit der gezeigten Konzepte auf konkrete Arbeitsumgebungen. 
Da die verfügbare Literatur oft sehr umfangreich ist und ihre Definitionen fragmentiert sind, stellt die Identifikation von relevanten Informationen für spezifische Problemstellungen eine weitere Herausforderung dar.\\

Die Vielzahl an verfügbaren Literaturquellen, deren Umfang und Fragmentierung bei der Definition von Fachbegriffen
erschweren eine schnelle Informationsbeschaffung, insbesondere für Studierende, die grundlegende Konzepte korrekt verstehen möchten. \\

\begin{itemize}
  \item Problem P1: Schwierigkeiten bei der Informationsbeschaffung aufgrund des Umfangs der Literatur und der Fragmentierung von Definitionen
\end{itemize}

\section{Motivation}

Eine Strukturierung von Informationen über das Management von Informationssystemen in Krankenhäusern ist bereits Teil des Projektes SNIK \citep{semantischesnetz}, 
ein semantischen Netz des Instituts für Medizinische Informatik, Statistik und Epidemiologie der Universität Leipzig.
Auf Basis dieses Netzes wurden bereits verschiedene Methoden untersucht, wie Informationen extrahiert werden können.\\

Die BeLL-Arbeit mit dem Titel \enquote{Question Answering auf SNIK} \citep{hannesbell} erweiterte den Zugang zu diesem Netz durch die Nutzung natürlicher englischer Sprache. 
Im Kontrast dazu untersuchten \citet{chatgpt_qas} die Leistung von Konversations-KI (in diesem Fall ChatGPT) im Vergleich zu einem herkömmlichen Question-Answering-Systemen, die auf Wissensgraphen basieren (genutzt wurde $KGQA_N$).
Die Ergebnisse zeigten, dass ChatGPT erstaunlich stabile und nachvollziehbare Antworten im Vergleich zum genutzten $KGQA_N$ lieferte.\\

Die Ergebnisse der BeLL-Arbeit wurden im Projekt QAnswer \citep{qanswer} umgesetzt, zeigen jedoch bei der Erklärbarkeit und dem Verständis von komplexeren Fragen mangelnde Resultate. 
Es ist daher notwendig zu untersuchen, ob die Verwendung einer Konversations-KI unter denselben Schwierigkeiten leidet und ob sich dabei neue Herausforderungen ergeben.

\section{Zielsetzung}\label{sec:zielsetzung}

Dem in~\ref{sec:problemstellung} gezeigten Problem P1 werden folgende Ziele dieser Arbeit zugeordnet.
\begin{itemize}
  \item Ziel Z1: Beantwortung von Fragen zu \citet{bb} in natürlicher Sprache durch eine Konversation-KI
  \item Ziel Z2: Lösung einer Beispielklausur des Moduls \enquote{Architektur von Informationssystemem im Gesundheitswesen} mit Hilfe einer Konversations-KI
\end{itemize}

\section{Aufgabenstellung}

Die in~\ref{sec:zielsetzung} genannten Ziele $Z_i$ werden durch die hier aufgeführten Aufgaben $A_i$ gelöst.

\begin{itemize}
  \item Aufgabe zu Ziel Z1
  \begin{itemize}
    \item Aufgabe A1.1: Vergleich von aktuell verfügbaren Sprachmodellen und ihrer Nutzbarkeit
    \item Aufgabe A1.2: Datenkuration von \citet{bb}
    \item Aufgabe A1.3: Nutzung einer Sprachmodells zur Bewertung, Verständis und Beantwortung der gegebenen Frage
  \end{itemize}
  \item Aufgabe zu Ziel Z2
  \begin{itemize}
    \item Aufgabe A2.1: Evaluierung der Konversations-KI vor und nach dem Training
    \item Aufgabe A2.2: Bewertung der Antwortoptionen von Klausurfragen
  \end{itemize}
\end{itemize}

\section{Aufbau der Arbeit}
Kapitel 1 beschreibt das grundlegende Umfeld dieser Arbeit, formuliert exitierende Probleme und Anforderungen, bietet Ziele zur Lösung dieser Probleme an und gibt Aufgaben, zur Umsetzung dieser Ziele. 
In Kapitel 2 werden Grundlagen gelegt zum Verständnis der in dieser Arbeit verwendeten Technologie, während in Kapitel 3 der aktuelle Stand der Forschung zusammengefasst wird. 
Kapitel 4 umfasst Lösungsstrategien der in~\ref{sec:problemstellung} formulierten Probleme mit einer anschließenden Beschreibung der Umsetzung dieser Lösungen in Kapitel 5. 
Die Ergebnisse dieser Arbeit werden in Kapitel 6 präsentiert und in Kapitel 7 zusammengefasst diskutiert. 
Zusätzlich gibt Kapitel 7 einen Ausblick dieser Arbeit.
