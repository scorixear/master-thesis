%************************************************
\chapter{Einleitung}\label{ch:introduction}
%************************************************
\section{Gegenstand}\label{sec:gegenstand}
Eine effektive und effiziente Informationsgewinnung bildet einen fundamentalen Bestandteil einer qualitativ hochwertigen klinischen Praxis in der Medizin. 
Bei jeder medizinischen Handlung werden große Mengen an Informationen genutzt und generiert, sei es zur Grundlage einer Diagnose oder zur Dokumentation des Behandlungsprozesses. 
Die strukturierte und klassifizierte Speicherung sowie Wiedergabe dieser Informationen stellt einen fortwährenden Entwicklungsprozess dar und ist Gegenstand aktueller Forschung.\\

Die Digitalisierung der Medizin ist ein ausgedehntes Themenfeld mit stetig wachsender Notwendigkeit. 
Die Medizinische Informatik beschreibt passend dazu \enquote{die Wissenschaft der systematischen Erschließung, Verwaltung, Aufbewahrung, Verarbeitung und Bereitstellung von Daten, Informationen und Wissen in der Medizin und im Gesundheitswesen. [...]}\citep{gmds}. 
In diesem Kontext wird die Bedeutung der Entwicklung und Implementierung effektiver Informationssysteme und Technologien zur Unterstützung der klinischen Praxis immer größer.\\

In der Lehre wird die Praxis der medizinischen Informatik durch umfassende Literatur wie zum Beispiel in \citet{bb} unterstützt. 
Zur Strukturierung dieser Literatur existiert die Ontologie SNIK \citep{snikgraphposter}, ein semantisches Netzwerk klassifiert in das SNIK Metamodell und Projekt des Instituts für Medizinische Informatik, Statistik und Epidemiologie \citep{imise} an der Universität Leipzig.
Durch die Verwendung dieses Netzwerkes wird eine systematische Darstellung von Rollen, Entitäten und Funktionen des Informationsmanagements in Krankenhäusern ermöglicht, losgelöst von den Definition der zugrunde liegenden Literaturquellen.\\

Die Bedeutung von maschinellem Lernen, Deep Learning und Sprachmodellen nimmt in der heutigen Zeit immer mehr zu. Diese Technologien werden in vielen Bereichen eingesetzt, von der Automobilindustrie bis hin zu medizinischen Anwendungen, um neue Methoden der Informationsgewinnung und -verarbeitung zu bieten. 
Eine Studie von \citet{mckinsey} zeigt, dass maschinelles Lernen und künstliche Intelligenz in den nächsten Jahren in der Wirtschaft zunehmend eingesetzt werden und dabei helfen werden, die Effizienz und Produktivität zu steigern.
Sprachmodelle wie beispielsweise das GPT-3-Modell von OpenAI \citep{gpt3} können dazu beitragen, Texte in verschiedenen Sprachen automatisch zu übersetzen und sogar kreative Schreibarbeiten zu erledigen. 
Deep Learning, das auf künstlichen neuronalen Netzwerken basiert, ermöglicht eine noch tiefere und komplexere Verarbeitung von Daten. 
In der Medizin kann Deep Learning zum Beispiel bei der Diagnose von Krankheiten und bei der Analyse von medizinischen Bildern eingesetzt werden \citep{skincancer}.

\section{Problemstellung} \label{sec:problemstellung}
Das Management von Informationssystemen ist eine komplexe Aufgabe, die sich nicht nur auf die Anwendung durch Mitarbeiter in Krankenhäusern beschränkt. Es ist auch wichtig für Studierende, um ein besseres Verständnis der vorhandenen Systeme zu erlangen, sowie für Wissenschaftler, um diese Systeme zu erweitern oder neue Methoden zum Management zu entwerfen.
Eine konkrete und konsistente Wissensbasis ist sowohl für die Anwendung als auch die Lehre und Forschung von großer Bedeutung. Die Forschung liefert hierfür eine Fülle von Literatur zu verschiedenen Aspekten der Medizinischen Informatik und dem Management von Informationssystemen.\\

Zwischen verschiedenen Literaturquellen herrscht jedoch oft Inkonsistenz bei der Definition und Benennung von Fachbegriffen. 
Zum Beispiel wird der Begriff \enquote{Krankenhausinformationssystem (engl. Hospital Information System)} in \citet{kis_winter} als das sozio-technische Subsystem eines Krankenhauses beschrieben,
das aus allen assoziierten menschlichen und technischen Akteuren, den damit verbundenen informationsverarbeitenden Rollen und allen Prozessen zur Verarbeitung dieser Informationen besteht.
Im Gegensatz dazu schließt die Definition in \citet{kis_italy} keine menschlichen Akteure ein.\\

Die Vielzahl an verfügbaren Literaturquellen, deren Umfang, unterschiedliche Perspektiven und semantische Unterschiede bei der Definition von Fachbegriffen
erschweren eine schnelle Informationsbeschaffung, insbesondere für Studierende, die grundlegende Konzepte korrekt verstehen möchten. \\

\begin{itemize}
  \item Problem P1: Schwierigkeiten bei der Informationsbeschaffung aufgrund der Fülle an Literatur und deren Umfang
  \item Problem P2: Inkonsistenz bei der Definition von Fachbegriffen zwischen verschiedenen Literaturquellen
\end{itemize}

\section{Motivation}

Eine Strukturierung von Informationen über das Management von Informationssystemen in Krankenhäusern ist bereits Teil des Projektes SNIK \citep{snikposter}, 
ein semantischen Netz des Instituts für Medizinische Informatik, Statistik und Epidemiologie der Universität Leipzig \citep{imise}. SNIK nutzt als Basis verschiedene Literaturquellen um eine Ontologie basierend auf dem SNIK Metamodell aufzubauen. 
Der Vorteil ist eine Umgehung von semantischen Unterschieden in den Literaturquellen durch eine Abstrahierung von Grundbegriffen in die Übergruppen Rolle, Entität und Funktion.
Auf Basis dieses Netzwerkes wurden bereits verschiedene Methoden untersucht, wie Informationen extrahiert werden können.\\

Die BeLL-Arbeit mit dem Titel \enquote{Question Answering auf SNIK} \citep{bell_snik} erweiterte den Zugang zu diesem Netzwerk durch die Nutzung natürlicher englischer Sprache. 
Im Kontrast dazu untersuchten \citet{chatgpt_qas} die Leistung von Konversations-KI (in diesem Fall ChatGPT) im Vergleich zu herkömmlichen Question-Answering-Systemen, die auf Wissensgraphen basieren (hier als $KGQA_N$ bezeichnet). 
Die Ergebnisse zeigten, dass ChatGPT erstaunlich stabile und nachvollziehbare Antworten im Vergleich zum genutzten $KGQA_N$ lieferte.

Die Ergebnisse der BeLL-Arbeit wurden im Projekt QAnswer \citep{qanswer} umgesetzt, zeigen jedoch bei der Erklärbarkeit und dem Verständis von komplexeren Fragen mangelnde Resultate. 
Es ist daher notwendig zu untersuchen, ob die Verwendung einer Konversations-KI unter denselben Schwierigkeiten leidet und ob sich dabei neue Herausforderungen ergeben.

\section{Zielsetzung}\label{sec:zielsetzung}

Den in \ref{sec:problemstellung} gezeigten Problemen $P_i$ werden Ziele $Z_i$ dieser Arbeit zugeordnet.

\begin{itemize}
  \item Ziel Z1: Wiedergabe von Wissen aus \citet{bb} in natürlicher Sprache generiert durch eine Konversation-KI
  \item Ziel Z2: Verständnis von semantischer Bedeutung von Begriffen erklärt in \citet{bb} durch die Konversations-KI
\end{itemize}

\section{Aufgabenstellung}

Die in \ref{sec:zielsetzung} genannten Ziele $Z_i$ werden durch die hier aufgeführten Aufgaben $A_i$ gelöst.

\begin{itemize}
  \item Aufgabe zu Ziel Z1
  \begin{itemize}
    \item Aufgabe A1.1: Implementierung einer Webseite, welche natürlich gestellte Fragen entgegen nimmt
    \item Aufgabe A1.2: Nutzung eines General Pre-trained Transformers (GPT) zur Bewertung, Verständis und Beantwortung der gegeben Fragen
  \end{itemize}
  \item Aufgabe zu Ziel Z2
  \begin{itemize}
    \item Aufgabe A2.1: Erstellung eines Fragenkatalog aus der Literaturquelle
    \item Aufgabe A2.2: Fein-Tuning des GPT-Modells auf die Literaturquelle
    \item Aufgabe A2.3: Analyse und Bewertung der ausgegebenen Daten
  \end{itemize}
\end{itemize}

\section{Aufbau der Arbeit}
Kapitel 1 beschreibt das grundlegende Umfeld dieser Arbeit, formuliert exitierende Probleme und Anforderungen, bietet Ziele zur Lösung dieser Probleme an und gibt Aufgaben, zur Umsetzung dieser Ziele. 
In Kapitel 2 werden Grundlagen gelegt zum Verständnis der in dieser Arbeit verwendeten Technologie, während in Kapitel 3 der aktuelle Stand der Forschung zusammengefasst wird. 
Kapitel 4 umfasst Lösungsstrategien der in \ref{sec:problemstellung} formulierten Probleme mit einer anschließenden Beschreibung der Umsetzung dieser Lösungen in Kapitel 5. 
Die Ergebnisse dieser Arbeit werden in Kapitel 6 präsentiert und in Kapitel 7 zusammengefasst diskutiert. Zusätzlich gibt Kapitel 7 einen Ausblick dieser Arbeit.