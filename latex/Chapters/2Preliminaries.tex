%*****************************************
\chapter{Grundlagen}\label{ch:preliminaries}
%*****************************************

\section*{Inhalte des Kapitels}
- Taxonomie von Ammus
- Transformer Architektur
- Neuronales Netz
- Activation Functions
- Backpropagation
- Training eines Modells
    - Batching
    - Optimizer
    - Learning Rate
- Feed-Forward Netze
- Multi-Head Attention
- Encoder und Decoder
- Input embeddings
- Positional Embeddings
- Byte Pair Encoding
- Transformer
- Autoregressive Modelle
    - Self-supervised Learning
    - Pretraining
    - continual Pretraining
    - Decoder-Based
    - ZeroShot / FewShot
- Weiterentwicklungen
    - Deep Residual Connections
    - DropOut


\section{Transformer}\label{sec:transformer}
%\todo{
%Grundlagen:
%- das was du schon drin hast mit "General Pretrained Transformer - Neuronal Networks - Zero Shot Ansatz - Finetuning - Datenkuration"
%- aber alles was mit GPT zu tun hat nicht, das ist ein spezieller Ansatz und das sollte zu state of the art
%- alles was im Titel steht :-) Question Answering, unsupervised und supervised training und so
%- Definitionen Daten, Informationen und Wissen nach Winter passt hier denke ich auch rein
%
%State of the Art:
%- aktuelle Paper
%- Vergleich verschiedener Systeme und Modelle
%- und das was du da schon hast, also das meiste ist ja schon am richtigen Ort
%}
Die erste schriftliche Erwähnung des Transformer-Modells und zusätzlich auch die Einführung der beiden Teilmodelle Encoder und Decoder findet sich in \citet{attention}.
Die hier beschriebene bidirektionale Architektur bildet die Grundlage für alle darauf aufbauenden Modelle und Weiterentwicklungen. 
Die grundlegende Architektur wurde für verschiedene Anwendungen stark modifiziert. 
Seit 2017 gibt es grundlegende Unterschiede in den Modellen und deren Möglichkeiten. 
Aus diesem Grund haben \citet{ammus} eine Taxonomie der Transformer-basierten vortrainierten Sprachmodelle eingeführt. 
Diese Taxonomie wird hier zur Beschreibung weiterer Architekturen und Methodiken verwendet.\\

Neben dem Grundbaustein eines Transformers - dem Attention-\ac{nn} - sind zwei wichtige Modifikationen gegenüber normalen neuronalen Netzen in Transformer eingeflossen. 
Residuale Verbindungen als Level-Normalisierung, auf Englisch \enquote{Deep Residual Connections}, verändern das Ziel eines \ac{nn}, behalten aber durch ihre Level-Normalisierung die gleichen Ausgaben bei. 
Dieses Konzept wurde erstmals in \citet{deep_residual} eingeführt und liefert die Lösung für ein grundlegendes Problem von großen, aus mehreren Ebenen bestehenden Transformer-Modellen. 
Bereits 2016 wurde im Bereich der Bilderkennung festgestellt, dass sich die Korrektheit von Modellen mit zunehmender Tiefe sättigt und dann schnell verschlechtert, wenn dieses Modell weiter trainiert wird. 
Dies setzte eine praktische Grenze für die Tiefe von \ac{nn}s und verhinderte somit die Lösung komplexerer Probleme mit größeren Modellen. 
\citet{deep_residual} beschreiben eine Lösung durch die genannten Residuen, die normale \ac{nn} simple ersetzen können, und zeigen ebenfalls die Wirksamkeit dieser Methode.\\

Die zweite wichtige Änderung ist die Einführung von Dropout. 
Dropout ist eine Methode, die die Trainingszeit von \ac{nn}s verkürzt und die Generalisierung verbessert. 
\citet{dropout} beschreiben die Methode als das zufällige Aussetzen von Neuronen in einem \ac{nn}. 
Diese Aussetzung hängt nicht von der Eingabe ab. 
Durch das Aussetzen von Neuronen wird das \ac{nn} gezwungen, sich nicht auf andere Neuronen zu verlassen und somit eine bessere Generalisierung zu erreichen. 
Das Aussetzen erfolgt nur während des Trainings und nicht während der Inferenz. 
Die Methode wurde 2014 eingeführt und ist seitdem ein fester Bestandteil von \ac{nn}s.\\

\section{Tokenization}\label{sec:tokenization}
Transformer-Modelle können Eingaben nicht ohne zusätzliche Umwandlung verarbeiten.
Neben der Erzeugung von Kodierungsvektoren muss die Eingabe zunächst in kleinere Einheiten, sogenannte Tokens, zerlegt werden.
Verschiedene ältere Modelle verwenden dazu Wörter oder Symbolunterteilungen.
Dies ist jedoch problematisch.\\

Durch die Zerlegung der Eingaben in Symbole ist zwar das Vokabular kleiner, welches zu schnelleren Trainingsdurchläufen führt, jedoch muss das Modell vor dem Erlernen von Wortzusammenhängen, Satzstrukturen und Sachverhalten zunächst die Bedeutung der Wörter und deren Zusammensetzung aus Symbolen erlernen.
Dies führt dazu, dass ein großer Teil der Trainingszeit für das Erlernen der Sprache verloren geht, was die endgültige Leistungsfähigkeit der Modelle massiv einschränkt \citep{bpe}.
Eine logische Schlussfolgerung wäre hier die Verwendung von Wörtern oder sogar Satzphrasen als Tokens.
Mit zunehmender Größe der Datensätze, die zum Training der Modelle verwendet werden, wächst hier das Vokabular immens an.
Dies führt zu einer starken Verlangsamung der Trainingsläufe und zu sehr großen Modellen ohne Vorteil in ihrer Leistungsfähigkeit.
Wörter mit gleichem Wortstamm oder ähnlicher Bedeutung aufgrund grammatikalischer Regeln (Plural, Genus, Tempora) müssen vom Modell erst als \enquote{gleiches Wort} gelernt werden.
Daher hat sich die Unterteilung von Wörtern in Teilwörter als Standard durchgesetzt.

\subsection{Byte-Pair-Encoding}\label{subsec:bpe}
\citet{bpe} schlugen zu diesem Zweck die Verwendung von \ac{bpe} vor.
Die Unterteilung von Wörtern in Untergruppen von Wörtern hat bereits bei der Übersetzung von Sätzen zu erheblichen Verbesserungen geführt. Sie hat sich aber auch in anderen Bereichen und Aufgaben wie der Textgenerierung, der Textklassifikation und der Analyse von Emotionen durchgesetzt.
Die Unterteilung von Wörtern ist hier eher als das Zusammenfügen kleinerer Teilwörter zu verstehen.
Ausgehend von einem Vokabular, das aus allen Symbolen eines Alphabets besteht, wird dieses durch das Zusammenführen (engl. \enquote{Merge}) von Symbolen erweitert, deren Kombination im Datensatz am häufigsten vorkommt.
Dieser Vorgang wird solange wiederholt, bis die gewünschte Anzahl von Teilwörtern erreicht ist.
Die Anzahl der Teilwörter ist dabei ein Hyperparameter, der je nach Modell und Datensatz variiert.\\

Die Unterteilung von Wörtern in Teilwörter hat den Vorteil, dass die Größe des Vokabulars nicht mit der Größe des Datensatzes wächst.
Dies führt zu einer schnelleren Eingabeverarbeitung und einer besseren Generalisierung der Modelle.
Die Unterteilung von Wörtern in Teilwörter hat jedoch auch Nachteile. Sie ist nicht eindeutig, d.h. ein Wort kann in unterschiedliche Mengen von Teilwörtern zerlegt werden.
Dies führt zu einer größeren Anzahl möglicher Eingaben, die das Modell lernen muss.
Ein weiterer Nachteil ist, dass die Zerlegung von Wörtern in Teilwörter nicht immer sinnvoll ist.
So kann es vorkommen, dass ein Wort in Teilwörter zerlegt wird, die in der Sprache nicht existieren.
Dies wiederum minimiert die Verallgemeinerbarkeit der Modelle.
Ein Beispiel hierfür ist das Wort \enquote{Datensatz}.
Eine sinnvolle Unterteilung wäre hier \enquote{Daten} und \enquote{satz}, aber durch den Aufbau des Vokabulars aus den Symbolen des Datensatzes kann es vorkommen, dass das Teilwort \enquote{Daten} nicht die notwendige Häufigkeit besitzt und somit nicht im Vokabular vorhanden ist.
Daher muss auch dieses Wort zerlegt werden, z.B. in \enquote{Da} und \enquote{ten}.
Beide Teilwörter haben in der deutschen Sprache keine Bedeutung, werden aber durch das Modell mit Bedeutung belegt und in Beziehung zu anderen Wörtern gesetzt.
Dies führt zu einer unverständlichen Bedeutungsannotation von Teilwörtern und verschlechtert sowohl die Leistung als auch die Nachvollziehbarkeit des Modells und erschwert die Forschung an den Modellen. 

